\documentclass[]{article}
\usepackage[latin1]{inputenc}
\usepackage{graphicx}
\usepackage[left=1.00in, right=1.00in, top=1.10in, bottom=1.00in]{geometry}

\usepackage{dirtytalk}
\usepackage[normalem]{ulem}
\usepackage{tikz-cd}
\usepackage{units}
\usepackage{algorithm}
\usepackage{algpseudocode}
\usepackage{alltt}
\usepackage{mathrsfs}
\usepackage{amssymb}
\usepackage{amsmath}
\DeclareMathOperator\cis{cis}

% (font shortcuts)
\usepackage{amsfonts}
\newcommand{\mb}[1]{\mathbb{#1}}
\newcommand{\mc}[1]{\mathcal{#1}}
\newcommand{\ms}[1]{\mathscr{#1}}
\newcommand{\mf}[1]{\frak{#1}}

% (arrow shortcuts)
\newcommand{\ra}{\rightarrow}
\newcommand{\lra}{\longrightarrow}
\newcommand{\la}{\leftarrow}
\newcommand{\lla}{\longleftarrow}
\newcommand{\Ra}{\Rightarrow}
\newcommand{\Lra}{\Longrightarrow}
\newcommand{\La}{\Leftarrow}
\newcommand{\Lla}{\Longleftarrow}
\newcommand{\lr}{\leftrightarrow}
\newcommand{\llr}{\longleftrightarrow}
\newcommand{\Lr}{\Leftrightarrow}
\newcommand{\Llr}{\Longleftrightarrow}

% (match parenthesis)
\newcommand{\mlr}[1]{\left|#1\right|}
\newcommand{\plr}[1]{\left(#1\right)}
\newcommand{\blr}[1]{\left[#1\right]}

% (exponent shortcuts)
\newcommand{\inv}{^{-1}}
\newcommand{\nrt}[2]{\sqrt[\leftroot{-2}\uproot{2}#1]{#2}}

% (annotation shortcuts)
\newcommand{\conj}[1]{\overline{#1}}
\newcommand{\ol}[1]{\overline{#1}}
\newcommand{\ul}[1]{\underline{#1}}
\newcommand{\os}[2]{\overset{#1}{#2}}
\newcommand{\us}[2]{\underset{#1}{#2}}
\newcommand{\ob}[2]{\overbrace{#2}^{#1}}
\newcommand{\ub}[2]{\underbrace{#2}_{#1}}
\newcommand{\bs}{\backslash}
\newcommand{\ds}{\displaystyle}

% (set builder)
\newcommand{\set}[1]{\left\{ #1 \right\}}
\newcommand{\setc}[2]{\left\{ #1 : #2 \right\}}
\newcommand{\setm}[2]{\left\{ #1 \, \middle| \, #2 \right\}}

% (group generator)
\newcommand{\gen}[1]{\langle #1 \rangle}

% (functions)
\newcommand{\im}[1]{\text{im}(#1)}
\newcommand{\range}[1]{\text{range}(#1)}
\newcommand{\domain}[1]{\text{domain}(#1)}
\newcommand{\dist}[1]{(#1)}
\newcommand{\sgn}{\text{sgn}}

% (Linear Algebra)
\newcommand{\mat}[1]{\begin{bmatrix}#1\end{bmatrix}}
\newcommand{\pmat}[1]{\begin{pmatrix}#1\end{pmatrix}}
%\newcommand{\dim}[1]{\text{dim}(#1)}
\newcommand{\rnk}[1]{\text{rank}(#1)}
\newcommand{\nul}[1]{\text{nul}(#1)}
\newcommand{\spn}[1]{\text{span}\,#1}
\newcommand{\col}[1]{\text{col}(#1)}
%\newcommand{\ker}[1]{\text{ker}(#1)}
\newcommand{\row}[1]{\text{row}(#1)}
\newcommand{\area}[1]{\text{area}(#1)}
\newcommand{\nullity}[1]{\text{nullity}(#1)}
\newcommand{\proj}[2]{\text{proj}_{#1}\left(#2\right)}
\newcommand{\diam}[1]{\text{diam}\,#1}

% (Vectors common)
\newcommand{\myvec}[1]{\vec{#1}}
\newcommand{\va}{\myvec{a}}
\newcommand{\vb}{\myvec{b}}
\newcommand{\vc}{\myvec{c}}
\newcommand{\vd}{\myvec{d}}
\newcommand{\ve}{\myvec{e}}
\newcommand{\vf}{\myvec{f}}
\newcommand{\vg}{\myvec{g}}
\newcommand{\vh}{\myvec{h}}
\newcommand{\vi}{\myvec{i}}
\newcommand{\vj}{\myvec{j}}
\newcommand{\vk}{\myvec{k}}
\newcommand{\vl}{\myvec{l}}
\newcommand{\vm}{\myvec{m}}
\newcommand{\vn}{\myvec{n}}
\newcommand{\vo}{\myvec{o}}
\newcommand{\vp}{\myvec{p}}
\newcommand{\vq}{\myvec{q}}
\newcommand{\vr}{\myvec{r}}
\newcommand{\vs}{\myvec{s}}
\newcommand{\vt}{\myvec{t}}
\newcommand{\vu}{\myvec{u}}
\newcommand{\vv}{\myvec{v}}
\newcommand{\vw}{\myvec{w}}
\newcommand{\vx}{\myvec{x}}
\newcommand{\vy}{\myvec{y}}
\newcommand{\vz}{\myvec{z}}
\newcommand{\vzero}{\myvec{0}}

%\usepackage[active,tightpage]{preview}
\setlength\PreviewBorder{7.77pt}
\usepackage{varwidth}
\AtBeginDocument{\begin{preview}\begin{varwidth}{\linewidth}}
\AtEndDocument{\end{varwidth}\end{preview}}


\author{Book: Erwin Kreyszig, Presenter: Richard Fabiano, Notes by Michael Reed}
\title{Functional Analysis}
%date{}

\begin{document}
\maketitle

%\begin{abstract}
%\end{abstract}

Chapter 1 - Metric spaces:
skip but will refer back to some examples.

\ul{Chapter 2}

\begin{definition}
	A vector space over a field $K$ is nonempty set $X$ of elements (called vectors) together with algebraic operations of vector addition and scalar multiplication which satisfy axioms (p. 50-51).
\end{definition}
\begin{note}
	In this course, $K$ is always either $\mb R$ or $\mb C$.
\end{note}
\begin{example}
	[$\mb R^2$ and $\mb R^3$] These we can \say{visualize} vectors as directed line segments and we have some \say{intuition.}
\end{example}
\begin{example}
	[$X = \mb R^n$] $x = (\xi_1,\dots,\xi_n)$, $\xi_i\in\mb R$.
\end{example}
\begin{example}
	[$X=\mb C^n$] $x = (\xi_1,\dots,\xi_n)$, $\xi_i\in\mb C$.
\end{example}
\begin{example}
	[$X=\ell^\infty$] , vectors are sequences $x=(\xi_1,\xi_2,\dots)$ satisfying $\sup\setc{|\xi_i|}{i=1,2,\dots}<\infty$.
\end{example}
\begin{example}
	$X = C[a,b]$ vectors $x=x(t)$ are continuous functions on interval $[a,b]$.
\end{example}

\begin{definition}
	A \ul{subspace} of a vector space $X$ is a nonempty subset $Y$ such that $\forall x,y\in Y$ and $\forall \alpha,\beta\in K$, then $\alpha x+\beta y\in Y$.
\end{definition}
\begin{note}
	A subspace $Y$ is itself a vector space.
\end{note}
\begin{definition}
	A \ul{linear combination} of vectors $x_1,\dots,x_n$ in $X$ is a vector of the form $\alpha_1x_1+\dots+\alpha_nx_n$, with $\alpha_n\in K$.
\end{definition}
\begin{definition}
	If $M\subset X$ is a subset of $X$, the set of all linear combinations of vectors of vectors in $M$ is called span of $M$, denoted $\spn M$.
\end{definition}
\begin{note}
	$\spn M$ is a subspace.
\end{note}
\begin{definition}
	Consider a finite set $M = \set{x_1,\dots,x_n}$ and the equation $\alpha_1x_1 + \dots + \alpha_nx_n = 0$ $(*)$.
	If $(*)$ holds only for $\alpha_1=\alpha_2=\dots=\alpha_n = 0$, then $M$ is linearly independent. otherwise dependent.
\end{definition}

An infinite set $M$ is linearly independent if every finite subset is linearly independent.

\begin{definition}
	If $X$ is a vector space and $\mc B$ is a linear independent subset such that $\spn \mc B = X$, then $\mc B$ is a basis for $X$ (Hamel basis).
\end{definition}

\begin{definition}
	A vector space $X$ is \ul{finite dimensional} if there is a natural number $n$ such that $X$ contains a set of $n$ linearly independent vectors, whereas any set of $n+1$ vectors is linearly dependent.
	In this case, $n$ is the \ul{dimension} of $X$.
	If $X$ is not finite dimensional, it is \ul{infinite dimensional}.
\end{definition}
\begin{note}
	$X = \set0$ has dimension 0.
\end{note}
\begin{corollary}
	Every finite dimensional vector space has a basis.
\end{corollary}

\begin{theorem}
	Every vector space has a basis (requires axiom of choice).
\end{theorem}

\ul{HW} 2.1: 3,4,6,7,10.

Last time: vector spaces and their (algebraic) properties.

To motivate definition of a norm on a vector space, consider $\mb R^2$.

\begin{definition}
	A normed vector space is a vector space $X$ with a norm $||\cdot||$ defined on it.
	A \ul{norm} $||\cdot||$ on a vector space $X$ is a real-valued function on $X$, with values denoted by $||x||$, which satisfies:
	\begin{itemize}
		\item[N1)] $||x||\geq 0$
		\item[N2)] $||x||=0$ if and only if $x=0$
		\item[N3)] $||\alpha x|| = |\alpha|\,||x||$ for all $x\in X$, for all $\alpha\in K$
		\item[N4)] $||x+y||\leq ||x||+||y||$ for all $x,y\in X$ (Triangle inequality).
	\end{itemize}
\end{definition}
\begin{note}
	The norm defines a metric on $X$ by $d(x,y) = ||x-y||$.
\end{note}
\begin{example}
	$\mb R^3$ with Euclidean norm: for $x=(\xi_1,\xi_2,\xi_3)$, $||x||=\sqrt{\xi_1^2+\xi_2^2+\xi_3^2}$.
\end{example}
\begin{example}
	$\mb R^n$ for $x=(\xi_1,\dots,\xi_n)$, define $||x||_2 = \sqrt{\sum_{i=1}^n \xi_i^2}$.
\end{example}
\begin{example}
	$X = C[a,b]$ for $x=x(t)$, define $\displaystyle||x|| = \max_{a\leq t\leq b}|x(t)|$.
	Notation: this is also denoted $||\cdot||_\infty$.
\end{example}
\begin{example}
	$X = \ell^\infty$ for $x = (\xi_i)_{i=1}^\infty$, define norm $||x|| = \sup\setc{|\xi_i|}{i=1,\dots}$.
\end{example}

\begin{definition}
	A sequence of vectors $(x_n)_{n=1}^\infty$ in a normed vector space $X$ is \ul{convergent} if there exists $x\in\mb X$ such that $\lim_{n\ra\infty}||x_n-x||=0$.
	Notation: we write $x_n\ra x$.
\end{definition}

\begin{recall}
	$\displaystyle\lim_{n\ra\infty} ||x_n-x||=0$ means: for every $\epsilon>0$, there exists $N>0$ such that if $n\geq N$, then $||x_n-x||<\epsilon$.
\end{recall}

\begin{definition}
	A sequence of vectors $(x_n)_{n=1}^\infty$ is \ul{Cauchy} if for every $\epsilon>0$, there exists $N>0$ such that if $m,n>N$, then $||x_m-x_n||<\epsilon$.
\end{definition}
\begin{definition}
	A normed vector space is \ul{complete} if every Cauchy sequence in $X$ is convergent in $X$.
\end{definition}
\begin{definition}
	A complete normed vector space is called a \ul{Banach space}.
\end{definition}

\begin{example}
	$X = \mb R^n$, with Euclidean norm $||\ ||_2$. For $x=(\xi_1,\dots,\xi_n)$, $||x|| = \sqrt{\sum_{i=1}^n \xi_i^2}$. $||\cdot||_2$ is a norm. N1-N3 easy.
	For $x,y=(\eta_1,\dots,\eta_n)$.
	To show $||x+y|| \leq ||x||+||y||$, show $||x+y||^2 \leq (||x||+||y||)^2 = ||x||^2 + ||y||^2 + 2 ||x||\,||y||$.
	\begin{align*}
		||x+y||^2 &= \sum_{i=1}^n (\xi_i+\eta_i)^2 
				  = \sum_{i=1}^n (\xi_i^2 + \eta_i^2 + 2\xi_i\eta_i) \\
				  &= \sum_{i=1}^n \xi_i + \sum_{i=1}^n \eta_i + 2\sum_{i=1}^n \xi_i\eta_i
				  = ||x||^2 + ||y||^2 + 2\sum_{i=1}^n \xi_i\eta_i \\
				  &\leq ||x||^2 + ||y||^2 + 2\sum_{i=1}^n |\xi_i\eta_i| \qquad \text{Cauchy-Schwarz ineqaulity}\\
				  &\leq ||x||^2 + ||y||2 + 2\sqrt{\sum_{i=1}^n|\xi_i|^2} \sqrt{\sum_{i=1}^n|\eta_i|^2} \\
				  &= ||x||^2 + ||y||^2 + 2||x||\cdot||y|| = (||x||+||y||)^2
	\end{align*}
	To show $\mb R^n$ is complete with this norm:
	Let $(x_m)_{m=1}^\infty$ be a Cauchy sequence in $\mb R^n$.
	Notation: $x_m = (\xi_1^m,\dots,\xi_n^m)$.
	Let $\epsilon>0$. There $\exists N>0$ such that if $m,r>N$, then $||x_m-r_r||<\epsilon$.
	So $||x_m-x_r||^2 < \epsilon^2$.
	So $\sum_{i=1}^n (\xi_i^m-\xi_i^r)^2 < \epsilon^2$ $(*)$.
	For each $i$, $(\xi_i^m-\xi_i^r)^2<\epsilon^2$, so $|\xi_i^m-\xi_i^r|<\epsilon$.
	So $(\xi_i^m)_{m=1}^\infty$ is a Cauchy sequence of real numbers, here convergent since $\mb R$ is complete.
	Thus $\lim_{m\ra\infty} \xi_i^m = \xi_i$ for each $i$.
	Define $x = (\xi_1,\dots,\xi_n)$.
	Let $r\ra\infty$ in $(*)$ to get $\sum_{i=1}^n(\xi_i^m-\xi_i)^2\leq\epsilon^2$ and $||x^m-x||^2 \leq \epsilon^2$ and $||x^m-x||\leq \epsilon$.
	Thus $x_m\ra x$.
\end{example}

\ul{HW} 2.2: 6,10

\end{document}
