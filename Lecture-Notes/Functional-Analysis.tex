\documentclass[]{article}
\usepackage[latin1]{inputenc}
\usepackage{graphicx}
\usepackage[left=1.00in, right=1.00in, top=1.10in, bottom=1.00in]{geometry}

\usepackage{dirtytalk}
\usepackage[normalem]{ulem}
\usepackage{tikz-cd}
\usepackage{units}
\usepackage{algorithm}
\usepackage{algpseudocode}
\usepackage{alltt}
\usepackage{mathrsfs}
\usepackage{amssymb}
\usepackage{amsmath}
\DeclareMathOperator\cis{cis}

% (font shortcuts)
\usepackage{amsfonts}
\newcommand{\mb}[1]{\mathbb{#1}}
\newcommand{\mc}[1]{\mathcal{#1}}
\newcommand{\ms}[1]{\mathscr{#1}}
\newcommand{\mf}[1]{\frak{#1}}

% (arrow shortcuts)
\newcommand{\ra}{\rightarrow}
\newcommand{\lra}{\longrightarrow}
\newcommand{\la}{\leftarrow}
\newcommand{\lla}{\longleftarrow}
\newcommand{\Ra}{\Rightarrow}
\newcommand{\Lra}{\Longrightarrow}
\newcommand{\La}{\Leftarrow}
\newcommand{\Lla}{\Longleftarrow}
\newcommand{\lr}{\leftrightarrow}
\newcommand{\llr}{\longleftrightarrow}
\newcommand{\Lr}{\Leftrightarrow}
\newcommand{\Llr}{\Longleftrightarrow}

% (match parenthesis)
\newcommand{\mlr}[1]{\left|#1\right|}
\newcommand{\plr}[1]{\left(#1\right)}
\newcommand{\blr}[1]{\left[#1\right]}

% (exponent shortcuts)
\newcommand{\inv}{^{-1}}
\newcommand{\nrt}[2]{\sqrt[\leftroot{-2}\uproot{2}#1]{#2}}

% (annotation shortcuts)
\newcommand{\conj}[1]{\overline{#1}}
\newcommand{\ol}[1]{\overline{#1}}
\newcommand{\ul}[1]{\underline{#1}}
\newcommand{\os}[2]{\overset{#1}{#2}}
\newcommand{\us}[2]{\underset{#1}{#2}}
\newcommand{\ob}[2]{\overbrace{#2}^{#1}}
\newcommand{\ub}[2]{\underbrace{#2}_{#1}}
\newcommand{\bs}{\backslash}
\newcommand{\ds}{\displaystyle}

% (set builder)
\newcommand{\set}[1]{\left\{ #1 \right\}}
\newcommand{\setc}[2]{\left\{ #1 : #2 \right\}}
\newcommand{\setm}[2]{\left\{ #1 \, \middle| \, #2 \right\}}

% (group generator)
\newcommand{\gen}[1]{\langle #1 \rangle}

% (functions)
\newcommand{\im}[1]{\text{im}(#1)}
\newcommand{\range}[1]{\text{range}(#1)}
\newcommand{\domain}[1]{\text{domain}(#1)}
\newcommand{\dist}[1]{(#1)}
\newcommand{\sgn}{\text{sgn}}

% (Linear Algebra)
\newcommand{\mat}[1]{\begin{bmatrix}#1\end{bmatrix}}
\newcommand{\pmat}[1]{\begin{pmatrix}#1\end{pmatrix}}
%\newcommand{\dim}[1]{\text{dim}(#1)}
\newcommand{\rnk}[1]{\text{rank}(#1)}
\newcommand{\nul}[1]{\text{nul}(#1)}
\newcommand{\spn}[1]{\text{span}\,#1}
\newcommand{\col}[1]{\text{col}(#1)}
%\newcommand{\ker}[1]{\text{ker}(#1)}
\newcommand{\row}[1]{\text{row}(#1)}
\newcommand{\area}[1]{\text{area}(#1)}
\newcommand{\nullity}[1]{\text{nullity}(#1)}
\newcommand{\proj}[2]{\text{proj}_{#1}\left(#2\right)}
\newcommand{\diam}[1]{\text{diam}\,#1}

% (Vectors common)
\newcommand{\myvec}[1]{\vec{#1}}
\newcommand{\va}{\myvec{a}}
\newcommand{\vb}{\myvec{b}}
\newcommand{\vc}{\myvec{c}}
\newcommand{\vd}{\myvec{d}}
\newcommand{\ve}{\myvec{e}}
\newcommand{\vf}{\myvec{f}}
\newcommand{\vg}{\myvec{g}}
\newcommand{\vh}{\myvec{h}}
\newcommand{\vi}{\myvec{i}}
\newcommand{\vj}{\myvec{j}}
\newcommand{\vk}{\myvec{k}}
\newcommand{\vl}{\myvec{l}}
\newcommand{\vm}{\myvec{m}}
\newcommand{\vn}{\myvec{n}}
\newcommand{\vo}{\myvec{o}}
\newcommand{\vp}{\myvec{p}}
\newcommand{\vq}{\myvec{q}}
\newcommand{\vr}{\myvec{r}}
\newcommand{\vs}{\myvec{s}}
\newcommand{\vt}{\myvec{t}}
\newcommand{\vu}{\myvec{u}}
\newcommand{\vv}{\myvec{v}}
\newcommand{\vw}{\myvec{w}}
\newcommand{\vx}{\myvec{x}}
\newcommand{\vy}{\myvec{y}}
\newcommand{\vz}{\myvec{z}}
\newcommand{\vzero}{\myvec{0}}

% Theorems and Propositions
\usepackage{amsthm}
\newtheorem{theorem}{Theorem}
\newtheorem{proposition}{Proposition}

\theoremstyle{definition}
\newtheorem{definition}{Definition}

\theoremstyle{remark}
\newtheorem*{remark}{Remark}
\newtheorem{example}{Example}
\newtheorem*{recall}{Recall}
\newtheorem*{note}{Note}
\newtheorem*{observe}{Observe}
\newtheorem*{question}{\underline{Question}}
\newtheorem*{fact}{Fact}
\newtheorem{corollary}{Corollary}
\newtheorem*{lemma}{Lemma}
\newtheorem{xca}{Exercise}

%\usepackage[active,tightpage]{preview}
\setlength\PreviewBorder{7.77pt}
\usepackage{varwidth}
\AtBeginDocument{\begin{preview}\begin{varwidth}{\linewidth}}
\AtEndDocument{\end{varwidth}\end{preview}}


\author{Presenter: Richard Fabiano, Notes by Michael Reed, Book: Erwin Kreyszig}
\title{Functional Analysis}
%date{}

\begin{document}
\maketitle

%\begin{abstract}
%\end{abstract}

Chapter 1 - Metric spaces:
skip but will refer back to some examples.

\ul{Chapter 2}

\begin{definition}
	A vector space over a field $K$ is nonempty set $X$ of elements (called vectors) together with algebraic operations of vector addition and scalar multiplication which satisfy axioms (p. 50-51).
\end{definition}
\begin{note}
	In this course, $K$ is always either $\mb R$ or $\mb C$.
\end{note}
\begin{example}
	[$\mb R^2$ and $\mb R^3$] These we can \say{visualize} vectors as directed line segments and we have some \say{intuition.}
\end{example}
\begin{example}
	[$X = \mb R^n$] $x = (\xi_1,\dots,\xi_n)$, $\xi_i\in\mb R$.
\end{example}
\begin{example}
	[$X=\mb C^n$] $x = (\xi_1,\dots,\xi_n)$, $\xi_i\in\mb C$.
\end{example}
\begin{example}
	[$X=\ell^\infty$] vectors are sequences $x=(\xi_1,\xi_2,\dots)$ satisfying $\sup\setc{|\xi_i|}{i=1,2,\dots}<\infty$.
\end{example}
\begin{example}
	[$X = C{[}a,b{]}$] vectors $x=x(t)$ are continuous functions on interval $[a,b]$.
\end{example}

\begin{definition}
	A \ul{subspace} of a vector space $X$ is a nonempty subset $Y$ such that $\forall x,y\in Y$ and $\forall \alpha,\beta\in K$, then $\alpha x+\beta y\in Y$.
\end{definition}
\begin{note}
	A subspace $Y$ is itself a vector space.
\end{note}
\begin{definition}
	A \ul{linear combination} of vectors $x_1,\dots,x_n$ in $X$ is a vector of the form $\alpha_1x_1+\dots+\alpha_nx_n$, with $\alpha_n\in K$.
\end{definition}
\begin{definition}
	If $M\subset X$ is a subset of $X$, the set of all linear combinations of vectors of vectors in $M$ is called span of $M$, denoted $\spn M$.
\end{definition}
\begin{note}
	$\spn M$ is a subspace.
\end{note}
\begin{definition}
	Consider a finite set $M = \set{x_1,\dots,x_n}$ and the equation $\alpha_1x_1 + \dots + \alpha_nx_n = 0$ $(*)$.
	If $(*)$ holds only for $\alpha_1=\alpha_2=\dots=\alpha_n = 0$, then $M$ is linearly independent. otherwise dependent.
\end{definition}

An infinite set $M$ is linearly independent if every finite subset is linearly independent.

\begin{definition}
	If $X$ is a vector space and $\mc B$ is a linear independent subset such that $\spn \mc B = X$, then $\mc B$ is a basis for $X$ (Hamel basis).
\end{definition}

\begin{definition}
	A vector space $X$ is \ul{finite dimensional} if there is a natural number $n$ such that $X$ contains a set of $n$ linearly independent vectors, whereas any set of $n+1$ vectors is linearly dependent.
	In this case, $n$ is the \ul{dimension} of $X$.
	If $X$ is not finite dimensional, it is \ul{infinite dimensional}.
\end{definition}
\begin{note}
	$X = \set0$ has dimension 0.
\end{note}
\begin{corollary}
	Every finite dimensional vector space has a basis.
\end{corollary}

\begin{theorem}
	Every vector space has a basis (requires axiom of choice).
\end{theorem}

\ul{HW} 2.1: 3,4,6,7,10.

Last time: vector spaces and their (algebraic) properties.

To motivate definition of a norm on a vector space, consider $\mb R^2$.

\begin{definition}
	A normed vector space is a vector space $X$ with a norm $||\cdot||$ defined on it.
	A \ul{norm} $||\cdot||$ on a vector space $X$ is a real-valued function on $X$, with values denoted by $||x||$, which satisfies:
	\begin{itemize}
		\item[N1)] $||x||\geq 0$
		\item[N2)] $||x||=0$ if and only if $x=0$
		\item[N3)] $||\alpha x|| = |\alpha|\,||x||$ for all $x\in X$, for all $\alpha\in K$
		\item[N4)] $||x+y||\leq ||x||+||y||$ for all $x,y\in X$ (Triangle inequality).
	\end{itemize}
\end{definition}
\begin{note}
	The norm defines a metric on $X$ by $d(x,y) = ||x-y||$.
\end{note}
\begin{example}
	$\mb R^3$ with Euclidean norm: for $x=(\xi_1,\xi_2,\xi_3)$, $||x||=\sqrt{\xi_1^2+\xi_2^2+\xi_3^2}$.
\end{example}
\begin{example}
	$\mb R^n$ for $x=(\xi_1,\dots,\xi_n)$, define $||x||_2 = \sqrt{\sum_{i=1}^n \xi_i^2}$.
\end{example}
\begin{example}
	$X = C[a,b]$ for $x=x(t)$, define $\displaystyle||x|| = \max_{a\leq t\leq b}|x(t)|$.
	Notation: this is also denoted $||\cdot||_\infty$.
\end{example}
\begin{example}
	$X = \ell^\infty$ for $x = (\xi_i)_{i=1}^\infty$, define norm $||x|| = \sup\setc{|\xi_i|}{i=1,\dots}$.
\end{example}

\begin{definition}
	A sequence of vectors $(x_n)_{n=1}^\infty$ in a normed vector space $X$ is \ul{convergent} if there exists $x\in\mb X$ such that $\lim_{n\ra\infty}||x_n-x||=0$.
	Notation: we write $x_n\ra x$.
\end{definition}

\begin{recall}
	$\displaystyle\lim_{n\ra\infty} ||x_n-x||=0$ means: for every $\epsilon>0$, there exists $N>0$ such that if $n\geq N$, then $||x_n-x||<\epsilon$.
\end{recall}

\begin{definition}
	A sequence of vectors $(x_n)_{n=1}^\infty$ is \ul{Cauchy} if for every $\epsilon>0$, there exists $N>0$ such that if $m,n>N$, then $||x_m-x_n||<\epsilon$.
\end{definition}
\begin{definition}
	A normed vector space is \ul{complete} if every Cauchy sequence in $X$ is convergent in $X$.
\end{definition}
\begin{definition}
	A complete normed vector space is called a \ul{Banach space}.
\end{definition}

\begin{example}
	$X = \mb R^n$, with Euclidean norm $||\cdot ||_2$. For $x=(\xi_1,\dots,\xi_n)$, $||x||_2 = \sqrt{\sum_{i=1}^n \xi_i^2}$ is a norm. N1-N3 easy.
	For $x,y=(\eta_1,\dots,\eta_n)$.
	To show $||x+y|| \leq ||x||+||y||$, show
	\begin{align*}
		||x+y||^2 &\leq (||x||+||y||)^2 = ||x||^2 + ||y||^2 + 2 ||x||\,||y||. \\
		||x+y||^2 &= \sum_{i=1}^n (\xi_i+\eta_i)^2 
				  = \sum_{i=1}^n (\xi_i^2 + \eta_i^2 + 2\xi_i\eta_i) \\
				  &= \sum_{i=1}^n \xi_i + \sum_{i=1}^n \eta_i + 2\sum_{i=1}^n \xi_i\eta_i
				  = ||x||^2 + ||y||^2 + 2\sum_{i=1}^n \xi_i\eta_i \\
				  &\leq ||x||^2 + ||y||^2 + 2\sum_{i=1}^n |\xi_i\eta_i| \qquad \text{Cauchy-Schwarz ineqaulity}\\
				  &\leq ||x||^2 + ||y||2 + 2\sqrt{\sum_{i=1}^n|\xi_i|^2} \sqrt{\sum_{i=1}^n|\eta_i|^2} \\
				  &= ||x||^2 + ||y||^2 + 2||x||\cdot||y|| = (||x||+||y||)^2
	\end{align*}
	To show $\mb R^n$ is complete with this norm:
	Let $(x_m)_{m=1}^\infty$ be a Cauchy sequence in $\mb R^n$.
	Notation: $x_m = (\xi_1^m,\dots,\xi_n^m)$.
	Let $\epsilon>0$. There $\exists N>0$ such that if $m,r>N$, then $||x_m-r_r||<\epsilon$.
	So $||x_m-x_r||^2 < \epsilon^2$.
	So $\sum_{i=1}^n (\xi_i^m-\xi_i^r)^2 < \epsilon^2$ $(*)$.
	For each $i$, $(\xi_i^m-\xi_i^r)^2<\epsilon^2$, so $|\xi_i^m-\xi_i^r|<\epsilon$.
	So $(\xi_i^m)_{m=1}^\infty$ is a Cauchy sequence of real numbers, here convergent since $\mb R$ is complete.
	Thus $\lim_{m\ra\infty} \xi_i^m = \xi_i$ for each $i$.
	Define $x = (\xi_1,\dots,\xi_n)$.
	Let $r\ra\infty$ in $(*)$ to get $\sum_{i=1}^n(\xi_i^m-\xi_i)^2\leq\epsilon^2$ and $||x^m-x||^2 \leq \epsilon^2$ and $||x^m-x||\leq \epsilon$.
	Thus $x_m\ra x$.
\end{example}

\ul{HW} 2.2: 6,10

Last time: normed vector spaces, $\mb R^n$ is complete.

Define vector space $s = $ set of all sequences (bounded or unbounded).
$x\in s$, $x=(\xi_1,\xi_2,\dots)$ or $x=(\xi_i)_{i=1}^\infty$.
Can define a metric by $$d(x,y) = \sum_{i=1}^\infty \frac1{2^i} \frac{|\xi_i-\eta_i|}{1+|\xi_i-\eta_i|},$$ where $x = (\xi_i)_{i=1}^\infty,y=(\eta_i)_{i=1}^\infty$.
Is there a norm $||\cdot||$ on $S$ such that $d(x,y) = ||x-y||$ for all $x,y\in S$.

\begin{theorem}
	[Translation Invariance] On vector space $X$, a metric $d$ induced by a norm is translation invariant: $d(x+a,y+a)=d(x,y)$ and $d(\alpha x,\alpha y) = |\alpha|d(x,y)$ for all $x,y,a\in X$, and all scalar.
\end{theorem}
\begin{proof}
	Let $x,y,a\in X$ and $\alpha$ scalar.
	$d(x+a,y+a) = ||(x+a)-(y+a)|| = ||x-y|| = d(x,y)$ and
	$d(\alpha x -\alpha y) = ||\alpha x-\alpha y|| = ||\alpha(x-y)|| = |\alpha|\ ||x-y|| = |\alpha| d(x,y)$.
\end{proof}

HW: Show metric $d$ on $s$ is not induced by any norm.

\begin{recall}
	[$\displaystyle X = \ell^\infty = \setc{x=(\xi_i)_{i=1}^\infty}{\sup_{1\leq i\leq\infty}|\xi_i|<\infty}$]
	Can define norm by $\displaystyle||x||_\infty = \sup_i|\xi_i|$.
	$\ell^\infty$ with this norm is complete, hence a Banach space.
\end{recall}
\begin{example}
	[$X = C{[}a,b{]}$]
	Define a norm on $X$ by $\displaystyle||x(t)||_\infty = \max_{a\leq t\leq b}|x(t)|$.
	Can check this is a norm. $C[a,b]$ with $||\ ||_\infty$ is complete (HW grad).
\end{example}
\begin{example}
	[$X = C{[}a,b{]}$] Define a norm on $X$ by $||x||_1 = \int_a^b|x(t)|\,dt$.
	Can check this is a norm.
	Claim: $C[a,b]$ with $||\ ||_1$ is not complete.
	To show this, we must demonstrate a sequence of functions $(x_n)_{n=1}^\infty$ in $C[a,b]$ which is Cauchy in $||\cdot ||_1$, but $(x_n)_{n=1}^\infty$ does not converge to any $x\in X$.
Take $X = C[0,1]$. Define $x_n(t) \begin{cases} 0 & 0\leq t\leq 1/2 \\ nt-\frac n2 & \frac12 \leq t \leq \frac12+\frac1n \\ 1 & \frac12+\frac1n\leq t\leq 1 \end{cases}$.
		Consider $\displaystyle||x_n-x_m||_1 = \int_0^1|x_n(t)-x_m(t)|\,dt \leq \frac12\cdot\frac1N$ for $n,m\geq N$.
		Given $\epsilon>0$, set $N>\frac1{2\epsilon}$.
		If $n,m\geq N$, then $||x_n-x_m||\leq\frac1{2N}<\epsilon$.
		Clearly $(x_n)_{n=1}^\infty$ is Cauchy in $||\,||_1$.
		Can we have $x_n\ra x$ for some $x(t)\in C[a,b]$?
		If so, then $||x_n-x||_1\ra 0$ but $$||x_n(t)-x(t)||_1 = \int_0^\frac12 |x(t)|\, dt + \int_\frac12^{a_n} |x_n(t)-x(t)|\,dt + \int_{a_n}^1 |1-x(t)|\,dt$$ must have $\lra 0$ as $n\ra\infty$.
		This implies $\int_0^\frac12|x(t)|\,dt=0$, so $x(t)=0$ for $0\leq t<\frac12$ and $\int_\frac12^1 |1-x(t)|\, dt = 0$, so $x(t) = 1$ for $\frac12<t\leq 1$.
		Can't have $x(t)\in C[0,1]$.
\end{example}
\begin{example}
	Similar for $X=C[a,b]$ with $||\cdot||_2,||\cdot||_p,p\geq 1$.
	$\displaystyle ||x||_p = \plr{\int_a^b|x(t)|^p\,dt}^\frac1p$.
\end{example}

\begin{definition}
	A subspace $Y$ of a normed vector space $X$ is \ul{closed} whenever $(x_n)_{n=1}^\infty$ is a sequence in $Y$ with $x_n\ra x$, then $x\in Y$.
\end{definition}
\begin{theorem}
	Let $X$ be a Banach space.
	A subspace $Y$ of $X$ is complete if and only if $Y$ is closed.
\end{theorem}
\begin{proof}
	\ul{$\implies$}$|$ Suppose $Y$ is complete subspace of $X$. Let $(x_n)_{n=1}^\infty$ be a sequence in $Y$, and suppose $x_n\ra x\in X$. Since $x_n\ra x$ in $X$, then $(x_n)_{n=1}^\infty$ is Cauchy in $X$. Thus $(x_n)_{n=1}^\infty$ is Cauchy in $Y$.
	Because $Y$ is complete, $(x_n)$ converges in $Y$. Thus $x\leq y$, so $Y$ is closed.

	\ul{$\Lla$}$|$ Suppose $Y$ is closed. Let $(x_n)_{n=1}^\infty$ be a Cauchy sequence in $Y$. Then $(x_n)_{n=1}^\infty$ is Cauchy in $X$, hence converges in $X$, say $x_n\ra x\in X$. Since $Y$ is closed, we ahve $x\in Y$.
	Thus $Y$ is complete.
\end{proof}

\begin{recall}
	For $x = (x_1,x_2)\in\mb R^2$, $||x||_2 = \sqrt{x_1^2+x_2^2}$ and $||x||_\infty = \max\set{|x_1|,|x_2|}$.
	So $$S_2 = \setc{x\in\mb R^2}{||x||_2 = 1} = \setc{(x_1,x_2)}{x_1^2+x_2^2=1}$$ 
	and $$S_\infty = \setc{x\in\mb R^2}{||x||_\infty=1} = \setc{(x_1,x_2)}{|x_1|=1\text{ or }|x_2|=1}.$$
\end{recall}

\begin{definition}
	Let $X$ and $Y$ be normed spaces.
	A mapping $T:X\ra Y$ is an \ul{isometry} if it preserves length: that is, if $||Tx||_Y = ||x||_X$ for all $x\in X$.
	If $T$ is also an isomorphism, then we say $X$ and $Y$ are \ul{isomorphically isometric} i.e., they are the \say{same.}
\end{definition}

\begin{theorem}
	Let $X$ be a normed space.
	Then there exists a completion of $X$.
\end{theorem}

\begin{lemma}
	Let $\set{x_1,x_2,\dots,x_n}$ be linearly independent in a normed vector space $X$.
	Then there exists $c>0$ such that for every set of scalars $\set{\alpha_1,\dots,\alpha_n}$ we have $(*)$ $||\alpha_1x_1+\alpha_2x_2+\dots+\alpha_nx_n|| \geq c(|\alpha_1|+\dots+|\alpha_n|)$.
\end{lemma}
\begin{proof}
	Set $s = |\alpha_1|+\dots+|\alpha_n|$. If $s=0$, then $(*)$ always holds for any $c>0$.
	So let $s>0$. Instead of $(*)$ consider $(**)$ $||\beta_1x_1+\beta_2x_2+\dots+\beta_nx_n||\geq c$ for all $\sum_{j=1}^n|\beta_j| = 1$, where $\beta_j = \frac{\alpha_j}s$.

	BWOC, suppose $(**)$ not true for all scalars such that $\sum_{j=1}^n|\beta_j| = 1$.
	Thus there is exists a sequence $(y_n)_{m=1}^\infty$, $y_m = \beta_1^m x_1 + \beta_2^m x_2 + \dots + \beta_n^m x_n$, with $\sum_{j=1}^n|\beta_j^m| = 1$, and $||y_m||\ra0$ as $m\ra 0$.
	We have
	\begin{align*}
		y_1 &: \beta_1^1, \beta_2^1,\dots,\beta_n^1 \\
		y_2 &: \beta_1^2, \beta_2^2,\dots,\beta_n^2 \\
			&\vdots \\
		y_m &: \beta_1^m,\beta_2^m,\dots,\beta_n^m \\
			&\vdots
	\end{align*}
	Notice $|\beta_j^m|\leq 1$ for all $m,j$. The sequence $(\beta_1^m)_{m=1}^\infty$ is bounded, hence has a convergent subsequence, $(\beta_1^{m,1})$, and $\beta_1^{m,1}\ra\beta_1$.
	There is a subsequence $(y_{1,m})$.
	Next consider $(\beta_2^{m,1})$.
	Similarly it has convergent subsequence $\beta_2^{m,2}$ and $\beta_2^{m,2}\ra\beta_2$.
	Also $\beta_1^{m,2}\ra\beta_1$. (Bolzano-Weierstrass theorem) Repeat for each index, getting subsequence of the previous subsequence.
	Get a subsequence $(y_{n,m})_{m=1}^\infty$ of $(y_m)$ such that $y_{n,m} = \gamma_1^mx_1 + \gamma_2^mx_2 + \dots + \gamma_n^mx_n$ and $\sum_{j=1}^n|\gamma_j^m|=1$ and $\lim_{m\ra\infty}\gamma_j^m = \beta_j$.
	Define $y = \beta_1x_1+\beta_2x_2+\dots+\beta_nx_n$.
	Thus $y_{n,m}\ra y$, where $\sum_{j=1}^n |\beta_j| = 1$.
	But $||y_{n,m}||\ra0$, and $||y||\neq0$ contradiction.
\end{proof}

\begin{theorem}
	Every finite dimensional subspace $Y$ of normed space $X$ is complete and closed.
\end{theorem}
\begin{corollary}
	Every finite dimensional normed space is a Banach space.
\end{corollary}

There are things that happen in infinite dimensional spaces that don't happen in finite dimensional spaces. For example, every finite dimensional vector space is complete; but not all infinite dimensional ones.

\ul{HW 2.3}: 1,3,10, G2;
\ul{HW 2.4}: 1,2,8, G6.

Last time: Every finite dimensional normed space is complete.
Use Lemma:
If $\set{x_1,\dots,x_n}$ is linearly independent in $X$, then $\exists c>0$ such that $||\alpha_1x_1+\dots+\alpha_nx_n|| \geq c(|\alpha_1|+\dots+|\alpha_n|)$.

\begin{definition}
	Let $X$ be a vector space, and let $||\cdot||$ and $||\cdot||_0$ be two norms on $X$.
	The norms are said to be \ul{equivalent} if there are positive numbers $a,b$ such that $\forall x\in X$ we have $a||x||_0 \leq ||x|| \leq b||x||_0$.
\end{definition}
\begin{remark}
	Note that $a||x||_0 \leq ||x||\leq b||x||_0 \implies \frac1b||x||\leq ||x||_0 \leq \frac1a||x||$ and $||x_n-x||\ra0$ implies $||x_n-x||_0\ra 0$ + vice versa.
\end{remark}
\begin{theorem}
	Let $X$ be a finite dimensional vector space.
	If $||\cdot||,||\cdot||_0$ are norms on $X$, they are equivalent.
\end{theorem}
\begin{proof}
	Let $X$ be finite dimensional and let $||\cdot||,||\cdot||_0$ be norms on $X$.
	Sufficient to show $\exists$ constants $c_1,c_2>0$ such that \ul{$||x||_0\leq c_1||x||$} and $||x||\leq c_2||x||_0$ for all $x\in X$.
	Let $\set{x_1,\dots,x_n}$ be a basis for $X$.
	Set $k = \max\setc{||x_j||_0}{j=1,\dots,n}>0$ and let $c>0$ be given by the Lemma in $||\cdot||$.
	Let $x\in X$. Then $x = \alpha_1x_1+\dots+\alpha_nx_n$.
	Then 
	\begin{align*}
		||x||_0 &= ||\alpha_1x_1+\dots+\alpha_nx_n||_0 \leq |\alpha_1|\,||x_1||_0 + \dots + |\alpha_n|\,||x_n||_0 \leq k(|\alpha_1|+\dots+|\alpha_n|) \\
				&\leq \frac kc||\alpha_1x_1+\dots+\alpha_nx_n|| \leq \frac kc||x|| = c_1||x||
	\end{align*}
	Similar argument shows that $||x||\leq c_2||x||_0$.
\end{proof}
\begin{example}
	An example of normed space with two norms that are not equivalent.
	We know it must be infinite dimensional by last result.
	$X = C[a,b]$ with norms $||\cdot||_\infty$ and $||\cdot||_1$.
	Not equivalent, $X$ with $||\cdot||_\infty$ is complete (Banach space) but $X$ with $||\cdot||_1$ is not complete.
\end{example}

\newpage

\begin{definition}
	A subset $M$ of a normed space (or metric space) $X$ is \ul{compact} if every sequence in $M$ has a subsequence that converges in $M$.
\end{definition}
\begin{lemma}
	A compact set $M$ is closed and bounded.
	Converse is not true in general, but it is true for compact subsets of $\mb R$ by Bolzano-Weierstrass theorem.
\end{lemma}
\begin{theorem}
	Let $X$ be a finite dimensional normed space.
	A subset $M$ is compact if and only if $M$ is closed and bounded.
\end{theorem}
\begin{example}
	A subset $M$ of a normed space which is closed and bounded, but not compact.
	Let $X = \ell^2$ with $||\cdot||_2$.
	Set 
	\begin{align*}
		e_1 &= (1,0,0,\dots) = (\delta_{1j})_{j=1}^\infty \\
		e_2 &= (0,1,0,\dots) = (\delta_{2j})_{j=1}^\infty \\
		e_n &= (\delta_{nj})_{j=1}^\infty
	\end{align*}
	$M = \set{e_1,e_2,\dots}$ is bounded because $||e_n||=1$ for $n=1,\dots$.
	Also $||e_m-e_n|| = \sqrt2$ for all $m\neq n$.
	So $M$ is closed, but $M$ is not compact.
	That is, $(e_n)_{n=1}^\infty$ is a sequence in $M$ with no convergent subsequence.
\end{example}

Last time: compact sets,,
\begin{itemize}
	\item a compact set is closed + bounded
	\item converse is true in finite dimensional noremd space
\end{itemize}

\begin{theorem}
	[Riesz Lemma]
	Let $Y,Z$ be subspaces of normed space $X$, and suppose $Y$ is a closed proper subspace of $Z$.
	Then for every $\theta\in(0,1)$ there exists $z\in Z$ such that $||z||=1$ and $||z-y||\geq\theta$ for all $y\in Y$.
\end{theorem}
\begin{proof}
	Let $\theta\in(0,1)$. Let $v\in Z\bs Y$. Let $\ds a = \inf_{y\in Y}||v-y||$.
	We know $a>0$ because if $a=0$, there exists a sequence $(y_n)_{n=1}^\infty$ in $Y$ such that $||v-y||\ra0$. 
	Thus $y_n\ra v$, which implies $v\in Y$ because $Y$ is closed. But $v\notin Y$, so $a\neq 0$.
	There exists $y_0\in Y$ such that $\ds a<||v-y_0||\leq \frac a\theta$.
	Such a vector exists, otherwise the $\ds \inf_{y\in Y}||v-y||\geq \frac a\theta$.
	Set $z = c(v-y_0)$, where $\ds c=\frac1{||v-y_0||}$.
	Then $z\in Z$ and $||z||=1$.
	For any $y\in Y$ we have 
	\begin{align*}
		||z-y|| &= ||c(v-y_0)-y||
				= c||v-y_0-\frac1cy||
				= c||v-y_1||,
	\end{align*}
	where $\ds y_1 = y_0+\frac1cy\in Y$, so $y_1\in Y$.
	So
		$$||z-y|| = c||v-y_1|| 
				\geq c\cdot a 
				= \frac a{||v-y_0||} 
				\geq a\cdot\frac\theta a = \theta$$
	implies $||z-y||\geq\theta$.
\end{proof}
\ul{HW} 2.5: 7

\begin{theorem}
	Let $X$ be a normed space. Then $X$ is finite dimensional if and only if the closed unit ball $M = \setc{x\in X}{||x||\leq 1}$ is compact.
\end{theorem}
\begin{proof}
	\ul{$\implies$}$\mid$ Suppose that $X$ is finite dimensional.
	Then $M$ is closed and bounded, hence compact by previous result.

	\ul{$\La$}$\mid$ Suppose $M$ is compact. BWOC, suppose $X$ is infinite dimensional.
	Pick $x_1\in $ such that $||x_1||=1$, and set $X_1=\spn\set{x_1}$.
	By Riesz's lemma, there exists $x_2\in X$ such that $||x_2||=1$ and $||x_2-x||\geq\frac12$ for all $x\in X_1$.
	In particular, $||x_2-x_1||\geq\frac12$.
	Set $X_2 = \spn\set{x_1,x_2}$.
	By Riesz lemma again, there exists $x_3\in X\bs X_2$ such that $||x_3||=1$ and $||x_3-x_2||\geq\frac12$ for all $x\in X_2$.
	In particular, $||x_3-x_2||\geq\frac12$ and $||x_3-x_1||\geq\frac12$.
	Continue process by induction to get sequence $(x_n)_{n=1}^\infty$ in $M$.
	This cannot have a convergent subsequence because $||x_m-x_n||\geq\frac12$ for all $m,n$.
	Contradiction to $M$ being compact.
\end{proof}

\subsection*{Linear Operators}

\begin{definition}
	A \ul{linear operator} $T$ is a mapping such that
	\begin{enumerate}
		\item[i)] the domain $\ms D(T)$ is a vector space and the range $\ms R(T)$ lies in a vector space with same field of scalars.
		\item[ii)] for all $x,y\in \ms D(T)$, and scalars $\alpha$, $T(x+y) = Tx+Ty$ and $T(\alpha x) = \alpha Tx$.
	\end{enumerate}
\end{definition}
\begin{definition}
	The \ul{null space} of $T$, denoted $\ms N(T)$, is $N(T) = \setc{x\in \ms D(T)}{Tx=0}$.
\end{definition}
\begin{theorem}
	Let $T:\ms D(T)\subset X\ra Y$ be a linear operator from $X$ into $Y$. Then
	\begin{enumerate}
		\item[a.] $\ms R(T)$ is a subspace of $Y$.
		\item[b.] If $\dim\ms D(T)=n<\infty$, then $\dim\ms R(T)\leq n$.
		\item[c.] $\ms N(T)$ is a subspace of $X$.
	\end{enumerate}
\end{theorem}
\begin{example}
	For any vector space $X$, the identity operator $I$ and zero operator $T=0$ (that is, $Tx=0$ for all $x\in X$) are linear operators.
\end{example}
\begin{example}
	[$X=C{[}a,b{]}$] The integral operator $T:X\ra X$,
	$$(Tx)(t) = \int_a^tx(s)\,ds.$$
	The differential operator $T:\ms D(T)\subset X\ra X$,
	$$ (Tx)(t) = x'(t).$$
	But $\ms D(T) = \setc{x(t)\in C[a,b]}{x'(t)\in C[a,b]}$.
\end{example}

\begin{recall}
Last time: compactness + dimension.
Introduction to linear operators.
If $T:\ms D(T)\subset X\ra Y$, then
\begin{enumerate}
	\item[a.] $\ms R(t)$ is a vector space,
	\item[b.] If $\dim\ms D(T)=n<\infty$, then $\dim\ms R(T)\leq n$,
	\item[c.] $N(T)$ is a vector space.
\end{enumerate}
\end{recall}
\begin{proof}
	[Proof of b.]
	Assume $\dim\ms D(T)=n<\infty$.
	From part a., $\ms R(T)$ is a vector space.
	To show $\dim\ms R(T)\leq n$,
	it is sufficient to show that every set of $n+1$ vectors in $\ms R(T)$ is linearly dependent.
	Let $\set{y_1,\dots,y_{n+1}}$ be any vectors in $\ms R(T)$.
	Then $\exists x_1,\dots,x_{n+1}\in\ms D(T)$ such that $Tx_j = y_j$ for $j=1,\dots,n+1$.
	Since $\dim\ms D(T) = n$, the set $\set{x_1,\dots,x_{n+1}}$ is linearly dependent.
	Thus there exist scalars $\alpha_1,\dots,\alpha_{n+1}$, not all zero, such that $\alpha_1x_1+\dots+\alpha_{n+1}x_{n+1} = 0$.
	So 
	\begin{align*}
		T(\alpha_1x_1+\dots+\alpha_{n+1}x_{n+1}) &= T0 = 0 \\
		\alpha_1Tx_1 + \alpha_{n+1}Tx_{n+1} &= 0 \\
		\alpha_1y_1 + \dots + \alpha_{n+1}y_{n+1} &= 0
	\end{align*}
	So $\set{y_1,\dots,y_{n+1}}$ is linearly dependent.
\end{proof}

\begin{example}
	For $X=\mb R^n$ and $Y=\mb R^m$, any $m\times n$ matrix defines a linear operator from $X$ to $Y$ by matrix multiplication $Tx=Ax$.
\end{example}
\begin{definition}
	Let $T:\ms D(T)\subset X\ra Y$ be 1-1.
	Then the \ul{inverse operator} $T\inv:\ms R(T)\subset Y\ra X$ is defined by $T\inv y=x$ if $Tx=y$.
\end{definition}
\begin{theorem}
	Let $X,Y$ be vector spaces, and let $T:\ms D(T)\subset X\ra Y$ be a linear operator.
	\begin{enumerate}
		\item[a)] $T\inv$ exists if and only if $Tx=0$ implies $x=0$.
		\item[b)] If $T\inv$ exists, it is a linear operator.
		\item[c)] If $\dim\ms D(T)=n<\infty$, and $T\inv$ exists, then $\dim\ms R(T)=n$.
	\end{enumerate}
\end{theorem}
\begin{note}
	If $T:X\ra Y$ and $S:Y\ra Z$ are invertible and $ST:X\ra Z$ is defined, then $(ST)\inv = T\inv S\inv$.
	(like inverse of product of two invertible square matrices)
\end{note}
So far, the discussion has been algebraic.
Next we introduce norms + linear operators.

\newpage

\begin{definition}
	Let $X,Y$ be normed spaces, and let $T:\ms D(T)\subset X\ra Y$ be a linear operator.
	$T$ is \ul{bounded} if there exists $c\geq0$ such that $||Tx|| \leq c||x||$ for all $x\in\ms D(T)$.
\end{definition}
\begin{definition}
The \say{smallest $c$ that works} is called the \ul{norm} of $T$, denoted $||T||$.
For $x\neq0$, we have $\frac{||Tx||}{||x||}\leq c$ for all $x\in\ms D(T),x\neq0$.
So $$||T|| = \sup_{\us{x\neq0}{x\in\ms D(T)}}\frac{||Tx||}{||x||}.$$
\end{definition}
Alternatively, $$ ||T|| = \sup_{\us{x\neq0}{x\in\ms D(T)}}\frac{||Tx||}{||x||} = \sup_{\us{x\neq0}{x\in\ms D(T)}} \mlr{\mlr{T\plr{\frac x{||x||}}}} = \sup_{\us{||y||=1}{y\in\ms D(T)}}||Ty|| = \sup_{\us{||x||=1}{x\in\ms D(T)}}||Tx||.$$
\begin{note}
	It follows that if $T$ is bounded, then $||Tx|| \leq ||T||\,||x||$.
\end{note}
\begin{note}
	$||T||$ satisfies properties of  norm, and is a norm on the vector space of all bounded linear operators from $X$ to $Y$.
\end{note}
\begin{example}
	Identity operator $||I|| = 1$. zero operator $||0||=0$.
\end{example}
\begin{example}
	$T:\mb R^n\ra\mb R^m$ by $Tx=Ax$, where $A$ is $m\times n$ matrix.
	For any norms on $\mb R^m$ and $\mb R^n$, $T$ is a bounded linear operator.
\end{example}
\begin{example}
	$X = C[0,1]$ with $||\cdot||_\infty$.
	Let $K(t,\tau)$ be continuous on $[0,1]\times[0,1]$. So $\exists K_0\geq 0$ such that $|K(t,\tau)|\leq K_0$ for all $t,\tau\in[0,1]$.
	Define $T:X\ra X$ by $(Tx)(t) = \int_0^t K(t,\tau)X(\tau)\,d\tau$.
	Can show $T$ is linear. Also $T$ is bounded because 
	\begin{align*}
		|| Tx||_\infty = \sup_{0\leq t\leq1}\mlr{\int_0^t K(t,\tau)x(\tau)\,d\tau}
		&\leq \sup_{0\leq t\leq 1}\int_0^t |K(t,\tau)|\,|x(\tau)|\,d\tau \\
		&\leq \sup_{0\leq t\leq 1}\int_0^t K_0||x||\,d\tau
					   = \sup_{0\leq t\leq 1}K_0||x||t
					   = K_0||x||_0
	\end{align*}
	So $||T||\leq K_0$.
\end{example}
\begin{example}
	[$X = C{[}0,1{]}$ with $||\cdot||_\infty$]
	$\ms D(T) = C^1[0,1]$.
	$(Tx)(t) = x'(t)$.
	Consider $x_n(t) = t^n$.
	Then $||x_n||=1$ for $n=1,2,\dots$
	but $||Tx_n||_\infty = ||nt^{n-1}||=n$.
	So there is  no $c$ such that $\frac{||Tx_n||}{||x_n||}\leq c$ for all $n$.
	So $T$ is unbounded.
\end{example}
\ul{HW} 2.6: 2,3,5,11; grad 14,15.

Let $x=(x_1,\dots,x_n)\in\mb R^n$.
$$\frac1{\sqrt n} ||x||_1 \leq ||x||_2 \iff \frac1{\sqrt n}(|x|_1+\dots+|x_n|)\leq\sqrt{|x_1|^2+\dots+|x_n|^2} \iff (|x_1|+\dots+|x_n|)^2 \leq n(|x_1|^2 + \dots + |x_n|^2) $$
See problem 1.2.3.
Case $n=2$: $(|x_1|+|x_2|)^2\leq2(|x_1|^2+|x_2|^2)$.
Observe: $\pm 2ab\leq a^2+b^2$ for all $a,b\in\mb R$ because $0\leq (a+b)^2$ and $0\leq(a-b)^2$.
For $n=2$, $(|x_1|+|x_2|)^2 = |x_1|^2+|x_2|^2+2|x_1|\,|x_2| \leq 2(|x_1|^2+|x_2|^2) \checkmark$.
For general $n$, by induction.
Assume for $n=k$:
$$(|x_1|+\dots+|x_k|)^2\leq k(|x_1|^2+\dots+|x_k|^2)$$
For $n=k+1$:
\begin{align*}
	(|x_1|+\dots+|x_k|+|x_{k+1}|)^2 &= (|x_1|+\dots+|x_k|)^2 + |x_{k+1}^2| + 2(|x_1|+\dots+|x_k|)|x_{k+1}| \\
									&\leq k(|x_1|+\dots+|x_k|)+|x_{k+1}|^2 + 2|x_1|\,|x_{k+1}| + \dots + 2|x_k|\,|x_{k+1}| \\
									&\leq k(|x_1|^2+\dots+|x_k|^2)+|x_{k+1}|^2+(|x_1|^2+|x_{k+1}|^2)+\dots+(|x_k|^2+|x_{k+1}|^2) \\
									&= k(|x_1|^2+\dots+|x_k|^2)+(|x_1|^2+\dots+|x_k|^2) + (k+1)|x_{k+1}|^2 \\
									&= (k+1)(|x_1|^2+\dots+|x_k|^2) + (k+1)|x_{k+1}|^2 \\
									&= (k+1)(|x_1|^2+\dots+|x_{k+1}|^2)\,\checkmark
\end{align*}

Last time: Bounded linear operators examples, examples of an unbounded linear operator.

\begin{example}
	[$X=\ell^\infty$ with $||\cdot||_\infty$]
	Given $x=(\xi_1,\xi_2,\dots)\in\ell^\infty$, define $Tx = (\xi_2,\xi_3,\dots)$ \say{left shift operator.}
	Easy to show $T$ is linear, $\ms D(T)=X$, so $T:X\ra X$.
	Also $||Tx||_\infty = \sup\set{|\xi_i|}_{i=2}^\infty\leq||x||_\infty$.
	This implies $T$ is bounded and $||T||\leq1$.
	Actually $||T||=1$, since for example, for $x=(0,2,0,\dots)$
	$$||Tx||_\infty = ||(2,0,0,\dots)||_\infty = 2 = ||x||_\infty.$$
	Thus it can't be possible for $||T||<1$.
\end{example}

\begin{theorem}
	Let $T:\ms D(T)\subset X\ra Y$ be a linear operator. If $X$ is finite dimensional, then $T$ is bounded.
\end{theorem}
\begin{recall}
	To show $T$ is bounded, show there is a constant $C$ such that $||Tx||\leq C||x||$ for all $x\in\ms D(T)$.
\end{recall}
\begin{recall}
	$\exists c>0$ such that for any scalars $\xi_1,\dots,\xi_n$, we have $||\xi_1e_1+\dots+\xi_ne_n||\geq c(|\xi_1|+\dots+|\xi_n|)$.
\end{recall}
\begin{proof}
	Assume $T$ is linear, $X,Y$, are normed spaces, and $X$ is finite dimensional.
	Let $n$ be the dimension of $X$ and $\set{e_1,\dots,e_n}$ be a basis for $X$.
	Let $B = \max\set{||Te_1||,\dots||Te_n||}$.
	Let $x\in\ms D(T)$. Then $x=\xi_1e_1+\dots+\xi_ne_n$ for some scalars $\xi_1,\dots,\xi_n$.
	Then 
	\begin{align*}
		||Tx|| &= ||T(\xi_1e_1+\dots+\xi_ne_n)|| \\
			   &\leq ||\xi_1Te_1+\dots+\xi_nTe_n||
			   \leq |\xi_1|\,||Te_1||+\dots+|\xi_n|\,||Te_n|| \\
			   &\leq B(|\xi_1+\dots+\xi_n|)
			   \leq \frac Bc||\xi_1e_1+\dots+\xi_ne_n||
			   = \frac Bc||x||
	\end{align*}
	Thus $||Tx||\leq\frac Bc||x||$ for all $x\in\ms D(T)$.
	So $T$ is bounded.
\end{proof}

\begin{definition}
	Let $T:\ms D(T)\subset X\ra Y$ be any operator (not necessarily linear), where $X$ and $Y$ are normed vector spaces.
	We say $T$ is continuous at $x_0\in\ms D(T)$ if for all $\epsilon>0$, there exists $\delta>0$ such that $x\in\ms D(T)$ and $||x-x_0||<\delta$, then $||Tx-Tx_0||<\epsilon$.
	A continuous operator is continuous at every $x\in\ms D(T)$.
\end{definition}
\begin{theorem}
	Let $T:\ms D(T)\subset X\ra Y$ be linear.
	Then
	\begin{enumerate}
		\item[a.] $T$ is continuous if and only if $T$ is bounded.
		\item[b.] If $T$ is continuous at one point $x_0\in\ms D(T)$, then $T$ is continuous on $\ms D(T)$.
	\end{enumerate}
\end{theorem}
\begin{proof}
	[Proof of a.]
	When $T=0$ clearly true, so assume $T\neq0$, so $||T||\neq0$.

	\ul{$\Lla$}$\mid|$ Assume $T$ is bounded, and $||T||\neq0$.
	Let $x_0\in\ms D(T)$. Let $\epsilon>0$. Set $\delta=\frac\epsilon{||T||}$.
	If $x\in\ms D(T)$ and $||x-x_0||<\delta$, then 
	\begin{align*}
		||Tx-Tx_0|| &= ||T(x-x_0)||
					\leq ||T||\cdot||x-x_0||
					< ||T||\cdot\delta
					= ||T||\cdot\frac\epsilon{||T||}=\epsilon.
	\end{align*}
	Thus $T$ is continuous at $x_0$.
	Since $x_0\in\ms D(T)$ is arbitrary, $T$ is continuous for every $x\in\ms D(T)$.
	
%	\ul{$\implies$}$\mid$ Next time.
%\end{proof}
%\begin{proof}
%	[Proof of a.]
%	\ul{$\Lla$}$\mid$ last time.
	\ul{$\implies$}$\mid$ Assume $T$ is continuous.
	Fix $x_0\in\ms D(T)$, so $T$ is continuous at $x_0$. Let $\epsilon=1$.
	Then there exists $\delta>0$ such that if $||x-x_0||\leq\delta$ then $||Tx-Tx_0||\leq1$.
	Let $y\in\ms D(T)$, $y\neq0$.
	Set $x=x_0+\frac\delta{||y||}y$. Then $||x-x_0|| = \mlr{\mlr{\frac\delta{||y||}y}}=\delta$.
	Thus $||Tx-Tx_0||\leq 1$. But observe
	\begin{align*}
		||Tx-Tx_0|| = ||T(x-x_0)||
					= \mlr{\mlr{T\plr{\frac\delta{||y||}}}}
					= \frac\delta{||y||}||Ty||
	\end{align*}
	So $\frac\delta{||y||}||Ty||\leq1$, so $||Ty||\leq\frac1\delta||y||$.
\end{proof}
\begin{recall}
	$T$ bounded $\iff \exists C$ such that $||Tx||\leq C||x||$ for all $x\in\ms D(T)$.
\end{recall}
\begin{proof}
	[Proof of b.] Suppose $T$ is continuous at $x_0$. By proof of \ul{$\implies$}$\mid$ in a., this implies $T$ is bounded.
	By \ul{$\Lla$}$\mid$ in a., this implies $T$ is continuous on all of $\ms D(T)$.
\end{proof}
\ul{HW} 2.7: 2,7,9. Grad 6 
\begin{corollary}
	Let $T:\ms D(T)\subset X\ra Y$ be bounded linear operator.
	\begin{enumerate}
		\item[a.] If $x_n\ra x$ and $x_n,x\in\ms D(T)$, then $Tx_n\ra Tx$.
		\item[b.] $\ms N(T)$ is closed.
	\end{enumerate}
\end{corollary}
\begin{proof}
	[Proof of a.] Assume $x_n\ra x$, $x_n,x\in\ms (D)T$, and $T$ bounded. Then
	\begin{align*}
		||Tx_n-Tx|| = ||T(x_n-x)||
					\leq ||T||\,||x_n-x||^{\ra0}.
	\end{align*}
	So $||Tx_n-Tx||\ra0$, so $Tx_n\ra Tx$.
\end{proof}
\begin{proof}
	[Proof of b.] Assume $T$ is bounded. Let $(x_n)_{n=1}^\infty$ be a sequence in $\ms N(T)$, and furthermore $x_n\ra x$ for $x\in X$.
	Thus $Tx_n\ra Tx$. But all $Tx_n=0$, $n=1,\dots$, so $Tx=0$. Thus $x\in\ms N(T)$, so $\ms N(T)$ is closed.
\end{proof}

\begin{definition}
	A \ul{linear functional} $f$ is a linear operator with domain $\ms D(f)\subset X$, where $X$ is a vector space, and with range space the field of scalars $\mb R$ or $\mb C$.
	Notation: use $f,g,h$.
\end{definition}
\begin{note}
	A linear functional is a linear operator.
\end{note}
\begin{definition}
	A \ul{bounded} \ul{linear functional} is a linear funcitonal which is a bounded linear operator.
\end{definition}
	Thus for a bounded linear functional we have:
\begin{itemize}
	\item $\exists c\geq 0$ such that $|f(x)|\leq c||x||$,
		so $\ds ||f|| = \sup_{\us{x\neq0}{x\in\ms D(f)}} \frac{|f(x)|}{||x||}
		= \sup_{\us{||x||=1}{x\in\ms D(f)}} |f(x)|. $
		\item $|f(x)| \leq ||f||\,||x||$.
\end{itemize}
\begin{example}
	[$X=\mb R^3$ with Euclidean norm] Fix $a=(\alpha_1,\alpha_2,\alpha_3)\in\mb R^3$.
	Define $f(x)$ for $x=(\xi_1,\xi_2,\xi_3)\in\mb R^3$ by $f(x) = x\cdot a = \xi_1\alpha_1+\xi_2\alpha_2+\xi_3\alpha_3$.
	Clearly $f$ is a linear functional.
	By theorem we know $f$ is bounded. We have: $|f(x)|=|x\cdot a|\leq ||x||\cdot||a||$ by Cauchy-Schwarz.
	This implies $||f||\leq||a||$. To show $||f||=||a||$, find $x$ such that $|f(x)|=||a||\,||x||$.
	Take $x=a$: $$|f(x)|=|f(a)| = a\cdot a = ||a||^2 = ||a||\,||a|| = ||a||\,||x||.$$
	Thus $||f||=||a||$.
\end{example}
\begin{example}
	[$X= C{[}a,b{]}$ with $||\cdot||_\infty$] Define $f(x)$ for $x(t)\in X$ by $f(x) = \int_a^bx(t)\,dt$.
	Clearly $f$ is a linear functional.
	To see if $f$ is bounded,
	\begin{align*}
		|f(x)| = \mlr{\int_a^bx(t)\,dt}
			   \leq \int_a^b|x(t)|\,dt
			   \leq \int_a^b||x||_\infty\,dt
			   = \fbox{$(b-a)$}\,||x||_\infty.
	\end{align*}
	Thus $f$ is bounded, $||f||\leq b-a$.
	To see that $||f||=b-a$, consider $x(t)\equiv 1$.
	Then $||x||_\infty=1$, and $|f(x)| = \mlr{\int_a^b1\,dt} = b-a = (b-a)||x||$.
	So $||f||=b-a$.
\end{example}
\begin{example}
	$X = C[a,b]$ with $||\cdot||_\infty$. Fix $c\in[a,b]$.
	Define $f(x) = x(c)$ (point evaluation), $f$ is linear. Easy to show $f$ is bounded.
\end{example}
\begin{example}
	[$X = C{[}0,1{]}$ with $||\cdot||_1$]
	Then $||x||_1 = \int_0^1|x(t)|\,dt$. Recall this is a normed space, but not complete.
	Define linear functional by $f(x) = x(0)$. We claim $f$ is not bounded.
	(There is no constant $C$ such that $|f(x)|\leq C||x||_1$ for all $x$).
	Consider $$x_n(t) = \begin{cases} 0 & \frac1n\leq t\leq 1 \\ -nt+1 & 0\leq t\leq\frac1n \end{cases}.$$
	Then $|f(x_n)| = |x_n(0)| = 1$ for all $n=1,2,\dots$ and $||x_n||_1 = \int_0^1|x_n(t)|\,dt = \frac1{2n}$ for $n=1,2\dots$.
	There is no $C$ such that $|f(x_n)|=1\leq C\frac1{2n} = C||x_n||_1$ for all $n$.
\end{example}
\ul{HW} 2.8: 3, G6.

\newpage
Last time: examples of linear functionals.

\begin{example}
	[$X=\ell^2$] Fix $a = (\alpha_j)\in\ell^2$.
	So $\sum_{j=1}^\infty |\alpha_j|^2<\infty$. Define $f:X\ra\mb R$ or $\mb C$ by $f(x) = \sum_{j=1}^\infty \alpha_j\xi_j$, where $x=(\xi_j)$.
	Clearly $f$ is linear.
	$f$ is bounded because 
	\begin{align*}
		|f(x)| &= \mlr{\sum_{j=1}^\infty \alpha_j\xi_j}
			   \leq \sum_{j=1}^\infty |\alpha_j\xi_j| \\
		\text{Cauchy-Schwarz}	   &\leq \sqrt{\sum_{j=1}^\infty |\alpha_j|^2}\sqrt{\sum_{j=1}^\infty |\xi_j|^2} 
			   = ||a||_2\cdot ||x||_2
	\end{align*}
	Thus $f$ is bounded and $||f||\leq||a||_2$.
	Notice for $x=a$ we get $|f(a)| = ||a||_2\cdot ||a||_2$.
	So $||f||=||a||_2$. $\checkmark$
\end{example}

\begin{definition}
	Let $X$ be a vector space. The vector space $X^* = \setc{f}{f\text{ is a linear functional on }X}$ is called the \ul{algebraic dual space} of $X$.
	$X^*$ is a vector space. 
\end{definition}
	What about $(X^*)^*$? This is called the second algebraic dual space.
Fix $x\in X$. Define $g_x\in X^{**}$ by:
For any $f\in X^*$, define $g_x(f) = f(x)$.
Claim: $g_x$ is linear: for any $f_1,f_2\in X^*$, scalars $\alpha,\beta$,
\begin{align*}
	g_x(\alpha f_1+\beta f_2) = (\alpha f_1+\beta f_2)(x) 
							  = \alpha f_1(x) + \beta f_2(x) 
							  = \alpha g_x(f_1) + \beta g_x(f_2)
\end{align*}
This defines a mapping $C:X\ra X^{**}$ by $Cx = g_x$.
$C$ is a linear operator, called the canonical mapping.
Are there any elements of $X^{**}$ besides those in $\ms R(C)$?
If $X^{**} = \ms R(C)$, we say $X$ is algebraically reflexive.

Linear operators on finite dimensional vector spaces. 
Let $X,Y$ be finite dimensional vector spaces and let $T:X\ra Y$ be a linear operator.
Let $\mc E=\set{e_1,\dots,e_n}$ be a basis for $X$ and $\mc B = \set{b_1,\dots,b_r}$ be a basis for $Y$.
Let $x\in X$. Then $x = \sum_{i=1}^n \xi_ie_i$
$$ y = Tx = \sum_{j=1}^n \xi_k Te_k = \sum_{k=1}^n \xi_ky_k, $$
where $y_k= Te_k$.
On the other hand, $y$ and $y_k$ have unique representation in terms of basis $\mc B$
$$ y = \sum_{j=1}^r \eta_jb_j, \qquad y_k = Te_k = \sum_{j=1}^r \tau_{jk}bj $$
Thus:
\begin{align*}
	\sum_{j=1}^r \eta_jb_j = y = \sum_{k=1}^n \xi_ky_k 
						= \sum_{k=1}^n \xi_k\plr{\sum_{j=1}^r \tau_{jk}b_j} 
						= \sum_{j=1}^r \plr{\sum_{k=1}^n \xi_k\tau_{jk}}b_j
\end{align*}
Thus $\eta_j = \sum_{k=1}^n \xi_k\tau_{jk}$.
Define matrix $T_{EB} = (\tau_{jk})_{r\times n}$.
Set $\tilde x = \pmat{\xi_1\\\vdots\\\xi_n}$, $\tilde y = \pmat{\eta_1\\\vdots\\\eta_r}$, then $\tilde y = T_{EB}\tilde x$.

\begin{example}
	[2.3-2] Let $(x_k)$ be a sequence in $c_0$, $x_k=(\xi_j^k)_{j=1}^\infty$ where $\xi_j^k\ra0$ as $j\ra\infty$.
	Suppose $x_k\ra x$ in $\ell^\infty$, $x=(\xi_j)_{j=1}^\infty$, that is $||x_k-x||_\infty\ra0$.
	Show $x\in c_0$. That is, show $\xi_j\ra 0$.
	Let $\epsilon >0$...
\end{example}
\begin{example}
	[2.3-3]
	Consider $x_n = (1,\frac12,\dots,\frac1n,0,0,0,\dots)$.
	Clearly each $x_n\in Y$ and $x_n\ra x\in\ell^\infty$, where $x = (1\frac12,\frac13,\dots)\notin Y$ because $||x_n-x||_\infty = \frac1{n+1}$.
\end{example}
\begin{example}
	[2.3-10] Let $(e_n)_{n=1}^\infty$ be a Schauder basis.
	Set $Y = \setc{\sum_{k=1}^n q_ke_k}{q_k\in\mb Q}$. Let $x\in X$ and $\epsilon>0$.
	Then 
	\begin{align*}
		\mlr{\mlr{x-\sum_{k=1}^nq_ke_k}} &\leq \mlr{\mlr{x-\sum_{k=1}^n \alpha_ke_k}} + \mlr{\mlr{\sum_{k=1}^n(\alpha_k-q_k)e_k}}
	\end{align*}
	$\epsilon/2$.
\end{example}

\begin{example}
	[Test \#5] $X = \setc{p(x)}{p(x)\text{ is a polynomial}}$, $||p||_1 = \int_0^1 |p(x)|\,dx$ and $||p||_\infty = \max_{0\leq x\leq1}|p(x)|$.
	Consider $f_n(x) = x^n$. Then $||f_n||_\infty=1$ for all $n$ but $||f_n||_1 = \frac1{n+1}\ra0$.
	So there is no constant $c>0$ such that $||p||_\infty\leq c||p||_1$ for all $p\in X$.
\end{example}
\newpage
\ul{Last time}: Linear operators on finite dimensional spaces, matrix representations.

\begin{definition}
	[Dual Basis] Given a vector space $X$ with basis $\set{e_1,\dots,e_n}$, there is a unique set of linear functionals $\set{f_1,\dots,f_n}$ such that $f_i(e_j) = \delta_{ij}$. % = \begin{cases} 0 & i\neq j \\ 1 & i=j \end{cases}$.
	This is a basis for dual space $X^*$.
\end{definition}
Given vector spaces $X,Y$, in linear algebra we define $\ms L(X,Y) = $ vector space of all linear operators from $X$ to $Y$. 
If $X,Y$ are normed spaces, we define $B(X,Y) = $ vector space of all bounded linear operators from $X$ to $Y$.
$B(X,Y)$ is a normed space with norm $$||T|| = \sup_{\us{x\neq0}{x\in X}} \frac{||Tx||}{||x||} = \sup_{\us{||x||=1}{x\in X}}||Tx||.$$
Dual space of $X = B(X,\mb R)$ or $B(X,\mb C) = X'$.

\begin{theorem}
	$B(X,Y)$ is complete (hence a Banach space) if $Y$ is complete.
\end{theorem}
\begin{corollary}
	The dual space $X'$ is a Banach space.
\end{corollary}

\section{Hilbert spaces}

We motivate definition of inner product by considering dot product in $\mb R^3$.
\begin{itemize}
	\item geometric definition of dot product $x\cdot y = ||x||\ ||y||\cos\theta$, thus $\theta = \arccos \frac{x\cdot y}{||x||\ ||y||}$.
	\item algebraic definition of dot product of $x=(\xi_1,\xi_2,\xi_3),y = (\eta_1,\eta_2,\eta_3)$: $x\cdot y = \xi_1\eta_1 + \xi_2\eta_2 + \xi_3\eta_3$, can be easily extended to $\mb R^n$.
	\item $||x||^2 = x\cdot x$.
	\item parallelogram law: \fbox{$||x+y||^2 + ||x-y||^2 = 2(||x||^2+||y||^2)$}.
\end{itemize}
\begin{definition}
	Let $X$ be a vector space. An \ul{inner product} on $X$ is a mapping $\gen{\cdot,\cdot}:X\times X\ra K$ ($\mb R$ or $\mb C$) such that for all $x,y,z\in X$ and scalars $\alpha\in K$,
	\begin{enumerate}
		\item $\gen{x+y,z} = \gen{x,z} + \gen{y,z}$
		\item $\gen{\alpha x,y} = \alpha\gen{x,y}$.
		\item $\gen{x,y} = \conj{\gen{y,x}}$.
		\item $\gen{x,x}\geq0$ and $\gen{x,x}=0\iff x=0$.
	\end{enumerate}
	The space $X$ is then called an \ul{inner product space}.
	A Hilbert space is a complete inner product space.
\end{definition}
\begin{remark}
	$\gen{\cdot,\cdot}$ is sesquilinear. %, since $\alpha\gen{x,y} = \gen{\alpha x,y} = \gen{x,\conj\alpha y}$.
\end{remark}
\begin{itemize}
	\item An inner product defines a norm by $||x||=\sqrt{\gen{x,x}}$.
	\item If $X$ is real, then $\gen{x,y} = \gen{y,x}$.
	\item Not every norm is defined by an inner product.
	\item If $||\cdot||$ is defined by an inner product, then $||\cdot||$ satisfies the parallelogram law.
	\item If a norm satisfies the parallelogram law, then it \say{comes from} an inner product (polarization identity).
\end{itemize}
Hilbert spaces have certain important properties not found in general normed spaces or Banach spaces.
\begin{itemize}
	\item Hilbert space $H$ can be represented as $H = M\oplus M^\perp$.
	\item Hilbert spaces may have \say{orthonormal basis} (countable).
	\item Riesz Theorem characterizes all bounded linear functionals.
	\item Can define the \say{adjoint} of a linear operator.
\end{itemize}

\begin{example}
	$\mb R^n,\mb C^n$ with the \say{dot product} and Euclidean norm.
\end{example}
\newpage
\begin{example}
	$\ds \ell^2 = \setc{x = (\xi_j)_{j=1}^\infty}{\sum_{j=1}^\infty |\xi_j|^2<\infty}$, 
	$\ds ||x||_2 = \sqrt{\sum_{j=1}^\infty |\xi_j|^2}$, 
	$\ds \gen{x,y} = \sum_{j=1}^\infty \xi_j\conj\eta_j$,
	$x=(\xi_j),y=(\eta_j)$. %_{j=1}^\infty$.
\end{example}
\begin{example}
	$C[a,b]$ with $\gen{f,g} = \int_a^b f(x)\conj{g(x)}\,dx$ is an inner product space with  norm $||f||_2 = \sqrt{\int_a^b |f(x)|^2\,dx}$. This is \ul{not} complete.
	The \say{completion} is the Hilbert space $L^2(a,b)$.
\end{example}
\ul{HW} 2.9: 2,4,6,G8. 3.1: 3,4,6,7.

\end{document}
