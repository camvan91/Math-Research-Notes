\documentclass[]{article}
\usepackage[latin1]{inputenc}
\usepackage{graphicx}
\usepackage[left=1.00in, right=1.00in, top=1.10in, bottom=1.00in]{geometry}

\usepackage{dirtytalk}
\usepackage[normalem]{ulem}
\usepackage{tikz-cd}
\usepackage{units}
\usepackage{algorithm}
\usepackage{algpseudocode}
\usepackage{alltt}
\usepackage{mathrsfs}
\usepackage{amssymb}
\usepackage{amsmath}
\DeclareMathOperator\cis{cis}

% (font shortcuts)
\usepackage{amsfonts}
\newcommand{\mb}[1]{\mathbb{#1}}
\newcommand{\mc}[1]{\mathcal{#1}}
\newcommand{\ms}[1]{\mathscr{#1}}
\newcommand{\mf}[1]{\frak{#1}}

% (arrow shortcuts)
\newcommand{\ra}{\rightarrow}
\newcommand{\lra}{\longrightarrow}
\newcommand{\la}{\leftarrow}
\newcommand{\lla}{\longleftarrow}
\newcommand{\Ra}{\Rightarrow}
\newcommand{\Lra}{\Longrightarrow}
\newcommand{\La}{\Leftarrow}
\newcommand{\Lla}{\Longleftarrow}
\newcommand{\lr}{\leftrightarrow}
\newcommand{\llr}{\longleftrightarrow}
\newcommand{\Lr}{\Leftrightarrow}
\newcommand{\Llr}{\Longleftrightarrow}

% (match parenthesis)
\newcommand{\mlr}[1]{\left|#1\right|}
\newcommand{\plr}[1]{\left(#1\right)}
\newcommand{\blr}[1]{\left[#1\right]}

% (exponent shortcuts)
\newcommand{\inv}{^{-1}}
\newcommand{\nrt}[2]{\sqrt[\leftroot{-2}\uproot{2}#1]{#2}}

% (annotation shortcuts)
\newcommand{\conj}[1]{\overline{#1}}
\newcommand{\ol}[1]{\overline{#1}}
\newcommand{\ul}[1]{\underline{#1}}
\newcommand{\os}[2]{\overset{#1}{#2}}
\newcommand{\us}[2]{\underset{#1}{#2}}
\newcommand{\ob}[2]{\overbrace{#2}^{#1}}
\newcommand{\ub}[2]{\underbrace{#2}_{#1}}
\newcommand{\bs}{\backslash}
\newcommand{\ds}{\displaystyle}

% (set builder)
\newcommand{\set}[1]{\left\{ #1 \right\}}
\newcommand{\setc}[2]{\left\{ #1 : #2 \right\}}
\newcommand{\setm}[2]{\left\{ #1 \, \middle| \, #2 \right\}}

% (group generator)
\newcommand{\gen}[1]{\langle #1 \rangle}

% (functions)
\newcommand{\im}[1]{\text{im}(#1)}
\newcommand{\range}[1]{\text{range}(#1)}
\newcommand{\domain}[1]{\text{domain}(#1)}
\newcommand{\dist}[1]{(#1)}
\newcommand{\sgn}{\text{sgn}}

% (Linear Algebra)
\newcommand{\mat}[1]{\begin{bmatrix}#1\end{bmatrix}}
\newcommand{\pmat}[1]{\begin{pmatrix}#1\end{pmatrix}}
%\newcommand{\dim}[1]{\text{dim}(#1)}
\newcommand{\rnk}[1]{\text{rank}(#1)}
\newcommand{\nul}[1]{\text{nul}(#1)}
\newcommand{\spn}[1]{\text{span}\,#1}
\newcommand{\col}[1]{\text{col}(#1)}
%\newcommand{\ker}[1]{\text{ker}(#1)}
\newcommand{\row}[1]{\text{row}(#1)}
\newcommand{\area}[1]{\text{area}(#1)}
\newcommand{\nullity}[1]{\text{nullity}(#1)}
\newcommand{\proj}[2]{\text{proj}_{#1}\left(#2\right)}
\newcommand{\diam}[1]{\text{diam}\,#1}

% (Vectors common)
\newcommand{\myvec}[1]{\vec{#1}}
\newcommand{\va}{\myvec{a}}
\newcommand{\vb}{\myvec{b}}
\newcommand{\vc}{\myvec{c}}
\newcommand{\vd}{\myvec{d}}
\newcommand{\ve}{\myvec{e}}
\newcommand{\vf}{\myvec{f}}
\newcommand{\vg}{\myvec{g}}
\newcommand{\vh}{\myvec{h}}
\newcommand{\vi}{\myvec{i}}
\newcommand{\vj}{\myvec{j}}
\newcommand{\vk}{\myvec{k}}
\newcommand{\vl}{\myvec{l}}
\newcommand{\vm}{\myvec{m}}
\newcommand{\vn}{\myvec{n}}
\newcommand{\vo}{\myvec{o}}
\newcommand{\vp}{\myvec{p}}
\newcommand{\vq}{\myvec{q}}
\newcommand{\vr}{\myvec{r}}
\newcommand{\vs}{\myvec{s}}
\newcommand{\vt}{\myvec{t}}
\newcommand{\vu}{\myvec{u}}
\newcommand{\vv}{\myvec{v}}
\newcommand{\vw}{\myvec{w}}
\newcommand{\vx}{\myvec{x}}
\newcommand{\vy}{\myvec{y}}
\newcommand{\vz}{\myvec{z}}
\newcommand{\vzero}{\myvec{0}}

%\usepackage[active,tightpage]{preview}
\setlength\PreviewBorder{7.77pt}
\usepackage{varwidth}
\AtBeginDocument{\begin{preview}\begin{varwidth}{\linewidth}}
\AtEndDocument{\end{varwidth}\end{preview}}


\author{Book: Erwin Kreyszig, Presenter: Richard Fabiano, Notes by Michael Reed}
\title{Functional Analysis}
%date{}

\begin{document}
\maketitle

%\begin{abstract}
%\end{abstract}

Chapter 1 - Metric spaces:
skip but will refer back to some examples.

\ul{Chapter 2}

\begin{definition}
	A vector space over a field $K$ is nonempty set $X$ of elements (called vectors) together with algebraic operations of vector addition and scalar multiplication which satisfy axioms (p. 50-51).
\end{definition}
\begin{note}
	In this course, $K$ is always either $\mb R$ or $\mb C$.
\end{note}
\begin{example}
	[$\mb R^2$ and $\mb R^3$] These we can \say{visualize} vectors as directed line segments and we have some \say{intuition.}
\end{example}
\begin{example}
	[$X = \mb R^n$] $x = (\xi_1,\dots,\xi_n)$, $\xi_i\in\mb R$.
\end{example}
\begin{example}
	[$X=\mb C^n$] $x = (\xi_1,\dots,\xi_n)$, $\xi_i\in\mb C$.
\end{example}
\begin{example}
	[$X=\ell^\infty$] , vectors are sequences $x=(\xi_1,\xi_2,\dots)$ satisfying $\sup\setc{|\xi_i|}{i=1,2,\dots}<\infty$.
\end{example}
\begin{example}
	$X = C[a,b]$ vectors $x=x(t)$ are continuous functions on interval $[a,b]$.
\end{example}

\begin{definition}
	A \ul{subspace} of a vector space $X$ is a nonempty subset $Y$ such that $\forall x,y\in Y$ and $\forall \alpha,\beta\in K$, then $\alpha x+\beta y\in Y$.
\end{definition}
\begin{note}
	A subspace $Y$ is itself a vector space.
\end{note}
\begin{definition}
	A \ul{linear combination} of vectors $x_1,\dots,x_n$ in $X$ is a vector of the form $\alpha_1x_1+\dots+\alpha_nx_n$, with $\alpha_n\in K$.
\end{definition}
\begin{definition}
	If $M\subset X$ is a subset of $X$, the set of all linear combinations of vectors of vectors in $M$ is called span of $M$, denoted $\spn M$.
\end{definition}
\begin{note}
	$\spn M$ is a subspace.
\end{note}
\begin{definition}
	Consider a finite set $M = \set{x_1,\dots,x_n}$ and the equation $\alpha_1x_1 + \dots + \alpha_nx_n = 0$ $(*)$.
	If $(*)$ holds only for $\alpha_1=\alpha_2=\dots=\alpha_n = 0$, then $M$ is linearly independent. otherwise dependent.
\end{definition}

An infinite set $M$ is linearly independent if every finite subset is linearly independent.

\begin{definition}
	If $X$ is a vector space and $\mc B$ is a linear independent subset such that $\spn \mc B = X$, then $\mc B$ is a basis for $X$ (Hamel basis).
\end{definition}

\begin{definition}
	A vector space $X$ is \ul{finite dimensional} if there is a natural number $n$ such that $X$ contains a set of $n$ linearly independent vectors, whereas any set of $n+1$ vectors is linearly dependent.
	In this case, $n$ is the \ul{dimension} of $X$.
	If $X$ is not finite dimensional, it is \ul{infinite dimensional}.
\end{definition}
\begin{note}
	$X = \set0$ has dimension 0.
\end{note}
\begin{corollary}
	Every finite dimensional vector space has a basis.
\end{corollary}

\begin{theorem}
	Every vector space has a basis (requires axiom of choice).
\end{theorem}

\ul{HW} 2.1: 3,4,6,7,10.

Last time: vector spaces and their (algebraic) properties.

To motivate definition of a norm on a vector space, consider $\mb R^2$.

\begin{definition}
	A normed vector space is a vector space $X$ with a norm $||\cdot||$ defined on it.
	A \ul{norm} $||\cdot||$ on a vector space $X$ is a real-valued function on $X$, with values denoted by $||x||$, which satisfies:
	\begin{itemize}
		\item[N1)] $||x||\geq 0$
		\item[N2)] $||x||=0$ if and only if $x=0$
		\item[N3)] $||\alpha x|| = |\alpha|\,||x||$ for all $x\in X$, for all $\alpha\in K$
		\item[N4)] $||x+y||\leq ||x||+||y||$ for all $x,y\in X$ (Triangle inequality).
	\end{itemize}
\end{definition}
\begin{note}
	The norm defines a metric on $X$ by $d(x,y) = ||x-y||$.
\end{note}
\begin{example}
	$\mb R^3$ with Euclidean norm: for $x=(\xi_1,\xi_2,\xi_3)$, $||x||=\sqrt{\xi_1^2+\xi_2^2+\xi_3^2}$.
\end{example}
\begin{example}
	$\mb R^n$ for $x=(\xi_1,\dots,\xi_n)$, define $||x||_2 = \sqrt{\sum_{i=1}^n \xi_i^2}$.
\end{example}
\begin{example}
	$X = C[a,b]$ for $x=x(t)$, define $\displaystyle||x|| = \max_{a\leq t\leq b}|x(t)|$.
	Notation: this is also denoted $||\cdot||_\infty$.
\end{example}
\begin{example}
	$X = \ell^\infty$ for $x = (\xi_i)_{i=1}^\infty$, define norm $||x|| = \sup\setc{|\xi_i|}{i=1,\dots}$.
\end{example}

\begin{definition}
	A sequence of vectors $(x_n)_{n=1}^\infty$ in a normed vector space $X$ is \ul{convergent} if there exists $x\in\mb X$ such that $\lim_{n\ra\infty}||x_n-x||=0$.
	Notation: we write $x_n\ra x$.
\end{definition}

\begin{recall}
	$\displaystyle\lim_{n\ra\infty} ||x_n-x||=0$ means: for every $\epsilon>0$, there exists $N>0$ such that if $n\geq N$, then $||x_n-x||<\epsilon$.
\end{recall}

\begin{definition}
	A sequence of vectors $(x_n)_{n=1}^\infty$ is \ul{Cauchy} if for every $\epsilon>0$, there exists $N>0$ such that if $m,n>N$, then $||x_m-x_n||<\epsilon$.
\end{definition}
\begin{definition}
	A normed vector space is \ul{complete} if every Cauchy sequence in $X$ is convergent in $X$.
\end{definition}
\begin{definition}
	A complete normed vector space is called a \ul{Banach space}.
\end{definition}

\begin{example}
	$X = \mb R^n$, with Euclidean norm $||\cdot ||_2$. For $x=(\xi_1,\dots,\xi_n)$, $||x||_2 = \sqrt{\sum_{i=1}^n \xi_i^2}$ is a norm. N1-N3 easy.
	For $x,y=(\eta_1,\dots,\eta_n)$.
	To show $||x+y|| \leq ||x||+||y||$, show
	\begin{align*}
		||x+y||^2 &\leq (||x||+||y||)^2 = ||x||^2 + ||y||^2 + 2 ||x||\,||y||. \\
		||x+y||^2 &= \sum_{i=1}^n (\xi_i+\eta_i)^2 
				  = \sum_{i=1}^n (\xi_i^2 + \eta_i^2 + 2\xi_i\eta_i) \\
				  &= \sum_{i=1}^n \xi_i + \sum_{i=1}^n \eta_i + 2\sum_{i=1}^n \xi_i\eta_i
				  = ||x||^2 + ||y||^2 + 2\sum_{i=1}^n \xi_i\eta_i \\
				  &\leq ||x||^2 + ||y||^2 + 2\sum_{i=1}^n |\xi_i\eta_i| \qquad \text{Cauchy-Schwarz ineqaulity}\\
				  &\leq ||x||^2 + ||y||2 + 2\sqrt{\sum_{i=1}^n|\xi_i|^2} \sqrt{\sum_{i=1}^n|\eta_i|^2} \\
				  &= ||x||^2 + ||y||^2 + 2||x||\cdot||y|| = (||x||+||y||)^2
	\end{align*}
	To show $\mb R^n$ is complete with this norm:
	Let $(x_m)_{m=1}^\infty$ be a Cauchy sequence in $\mb R^n$.
	Notation: $x_m = (\xi_1^m,\dots,\xi_n^m)$.
	Let $\epsilon>0$. There $\exists N>0$ such that if $m,r>N$, then $||x_m-r_r||<\epsilon$.
	So $||x_m-x_r||^2 < \epsilon^2$.
	So $\sum_{i=1}^n (\xi_i^m-\xi_i^r)^2 < \epsilon^2$ $(*)$.
	For each $i$, $(\xi_i^m-\xi_i^r)^2<\epsilon^2$, so $|\xi_i^m-\xi_i^r|<\epsilon$.
	So $(\xi_i^m)_{m=1}^\infty$ is a Cauchy sequence of real numbers, here convergent since $\mb R$ is complete.
	Thus $\lim_{m\ra\infty} \xi_i^m = \xi_i$ for each $i$.
	Define $x = (\xi_1,\dots,\xi_n)$.
	Let $r\ra\infty$ in $(*)$ to get $\sum_{i=1}^n(\xi_i^m-\xi_i)^2\leq\epsilon^2$ and $||x^m-x||^2 \leq \epsilon^2$ and $||x^m-x||\leq \epsilon$.
	Thus $x_m\ra x$.
\end{example}

\ul{HW} 2.2: 6,10

Last time: normed vector spaces, $\mb R^n$ is complete.

Define vector space $s = $ set of all sequences (bounded or unbounded).
$x\in s$, $x=(\xi_1,\xi_2,\dots)$ or $x=(\xi_i)_{i=1}^\infty$.
Can define a metric by $$d(x,y) = \sum_{i=1}^\infty \frac1{2^i} \frac{|\xi_i-\eta_i|}{1+|\xi_i-\eta_i|},$$ where $x = (\xi_i)_{i=1}^\infty,y=(\eta_i)_{i=1}^\infty$.
Is there a norm $||\cdot||$ on $S$ such that $d(x,y) = ||x-y||$ for all $x,y\in S$.

\begin{theorem}
	[Translation Invariance] On vector space $X$, a metric $d$ induced by a norm is translation invariant: $d(x+a,y+a)=d(x,y)$ and $d(\alpha x,\alpha y) = |\alpha|d(x,y)$ for all $x,y,a\in X$, and all scalar.
\end{theorem}
\begin{proof}
	Let $x,y,a\in X$ and $\alpha$ scalar.
	$d(x+a,y+a) = ||(x+a)-(y+a)|| = ||x-y|| = d(x,y)$ and
	$d(\alpha x -\alpha y) = ||\alpha x-\alpha y|| = ||\alpha(x-y)|| = |\alpha|\ ||x-y|| = |\alpha| d(x,y)$.
\end{proof}

HW: Show metric $d$ on $s$ is not induced by any norm.

\begin{recall}
	$\displaystyle X = \ell^\infty = \setc{x=(\xi_i)_{i=1}^\infty}{\sup_{1\leq i\leq\infty}|\xi_i|<\infty}$.
	Can define norm by $\displaystyle||x||_\infty = \sup_i|\xi_i|$.
	$\ell^\infty$ with this norm is complete, hence a Banach space.
\end{recall}
\begin{example}
	$X = C[a,b]$.
	Define a norm on $X$ by $\displaystyle||x(t)||_\infty = \max_{a\leq t\leq b}|x(t)|$.
	Can check this is a norm. $C[a,b]$ with $||\ ||_\infty$ is complete (HW grad).
\end{example}
\begin{example}
	$X = C[a,b]$. Define a norm on $X$ by $||x||_1 = \int_a^b|x(t)|\,dt$.
	Can check this is a norm.
	Claim: $C[a,b]$ with $||\ ||_1$ is not complete.
	To show this, we must demonstrate a sequence of functions $(x_n)_{n=1}^\infty$ in $C[a,b]$ which is Cauchy in $||\cdot ||_1$, but $(x_n)_{n=1}^\infty$ does not converge to any $x\in X$.
Take $X = C[0,1]$. Define $x_n(t) \begin{cases} 0 & 0\leq t\leq 1/2 \\ nt-\frac n2 & \frac12 \leq t \leq \frac12+\frac1n \\ 1 & \frac12+\frac1n\leq t\leq 1 \end{cases}$.
		Consider $\displaystyle||x_n-x_m||_1 = \int_0^1|x_n(t)-x_m(t)|\,dt \leq \frac12\cdot\frac1N$ for $n,m\geq N$.
		Given $\epsilon>0$, set $N>\frac1{2\epsilon}$.
		If $n,m\geq N$, then $||x_n-x_m||\leq\frac1{2N}<\epsilon$.
		Clearly $(x_n)_{n=1}^\infty$ is Cauchy in $||\,||_1$.
		Can we have $x_n\ra x$ for some $x(t)\in C[a,b]$?
		If so, then $||x_n-x||_1\ra 0$ but $$||x_n(t)-x(t)||_1 = \int_0^\frac12 |x(t)|\, dt + \int_\frac12^{a_n} |x_n(t)-x(t)|\,dt + \int_{a_n}^1 |1-x(t)|\,dt$$ must have $\lra 0$ as $n\ra\infty$.
		This implies $\int_0^\frac12|x(t)|\,dt=0$, so $x(t)=0$ for $0\leq t<\frac12$ and $\int_\frac12^1 |1-x(t)|\, dt = 0$, so $x(t) = 1$ for $\frac12<t\leq 1$.
		Can't have $x(t)\in C[0,1]$.
\end{example}
\begin{example}
	Similar for $X=C[a,b]$ with $||\cdot||_2,||\cdot||_p,p\geq 1$.
	$\displaystyle ||x||_p = \plr{\int_a^b|x(t)|^p\,dt}^\frac1p$.
\end{example}

\begin{definition}
	A subspace $Y$ of a normed vector space $X$ is \ul{closed} whenever $(x_n)_{n=1}^\infty$ is a sequence in $Y$ with $x_n\ra x$, then $x\in Y$.
\end{definition}
\begin{theorem}
	Let $X$ be a Banach space.
	A subspace $Y$ of $X$ is complete if and only if $Y$ is closed.
\end{theorem}
\begin{proof}
	\ul{$\implies$}$|$ Suppose $Y$ is complete subspace of $X$. Let $(x_n)_{n=1}^\infty$ be a sequence in $Y$, and suppose $x_n\ra x\in X$. Since $x_n\ra x$ in $X$, then $(x_n)_{n=1}^\infty$ is Cauchy in $X$. Thus $(x_n)_{n=1}^\infty$ is Cauchy in $Y$.
	Because $Y$ is complete, $(x_n)$ converges in $Y$. Thus $x\leq y$, so $Y$ is closed.

	\ul{$\Lla$}$|$ Suppose $Y$ is closed. Let $(x_n)_{n=1}^\infty$ be a Cauchy sequence in $Y$. Then $(x_n)_{n=1}^\infty$ is Cauchy in $X$, hence converges in $X$, say $x_n\ra x\in X$. Since $Y$ is closed, we ahve $x\in Y$.
	Thus $Y$ is complete.
\end{proof}

\begin{recall}
	For $x = (x_1,x_2)\in\mb R^2$, $||x||_2 = \sqrt{x_1^2+x_2^2}$ and $||x||_\infty = \max\set{|x_1|,|x_2|}$.
	So $$S_2 = \setc{x\in\mb R^2}{||x||_2 = 1} = \setc{(x_1,x_2)}{x_1^2+x_2^2=1}$$ 
	and $$S_\infty = \setc{x\in\mb R^2}{||x||_\infty=1} = \setc{(x_1,x_2)}{|x_1|=1\text{ or }|x_2|=1}.$$
\end{recall}

\begin{definition}
	Let $X$ and $Y$ be normed spaces.
	A mapping $T:X\ra Y$ is an \ul{isometry} if it preserves length: that is, if $||Tx||_Y = ||x||_X$ for all $x\in X$.
	If $T$ is also an isomorphism, then we say $X$ and $Y$ are \ul{isomorphically isometric} i.e., they are the \say{same.}
\end{definition}

\begin{theorem}
	Let $X$ be a normed space.
	Then there exists a completion of $X$.
\end{theorem}

\begin{lemma}
	Let $\set{x_1,x_2,\dots,x_n}$ be linearly independent in a normed vector space $X$.
	Then there exists $c>0$ such that for every set of scalars $\set{\alpha_1,\dots,\alpha_n}$ we have $(*)$ $||\alpha_1x_1+\alpha_2x_2+\dots+\alpha_nx_n|| \geq c(|\alpha_1|+\dots+|\alpha_n|)$.
\end{lemma}
\begin{proof}
	Set $s = |\alpha_1|+\dots+|\alpha_n|$. If $s=0$, then $(*)$ always holds for any $c>0$.
	So let $s>0$. Instead of $(*)$ consider $(**)$ $||\beta_1x_1+\beta_2x_2+\dots+\beta_nx_n||\geq c$ for all $\sum_{j=1}^n|\beta_j| = 1$, where $\beta_j = \frac{\alpha_j}s$.

	BWOC, suppose $(**)$ not true for all scalars such that $\sum_{j=1}^n|\beta_j| = 1$.
	Thus there is exists a sequence $(y_n)_{m=1}^\infty$, $y_m = \beta_1^m x_1 + \beta_2^m x_2 + \dots + \beta_n^m x_n$, with $\sum_{j=1}^n|\beta_j^m| = 1$, and $||y_m||\ra0$ as $m\ra 0$.
	We have
	\begin{align*}
		y_1 &: \beta_1^1, \beta_2^1,\dots,\beta_n^1 \\
		y_2 &: \beta_1^2, \beta_2^2,\dots,\beta_n^2 \\
			&\vdots \\
		y_m &: \beta_1^m,\beta_2^m,\dots,\beta_n^m \\
			&\vdots
	\end{align*}
	Notice $|\beta_j^m|\leq 1$ for all $m,j$. The sequence $(\beta_1^m)_{m=1}^\infty$ is bounded, hence has a convergent subsequence, $(\beta_1^{m,1})$, and $\beta_1^{m,1}\ra\beta_1$.
	There is a subsequence $(y_{1,m})$.
	Next consider $(\beta_2^{m,1})$.
	Similarly it has convergent subsequence $\beta_2^{m,2}$ and $\beta_2^{m,2}\ra\beta_2$.
	Also $\beta_1^{m,2}\ra\beta_1$. (Bolzano-Weierstrass theorem) Repeat for each index, getting subsequence of the previous subsequence.
	Get a subsequence $(y_{n,m})_{m=1}^\infty$ of $(y_m)$ such that $y_{n,m} = \gamma_1^mx_1 + \gamma_2^mx_2 + \dots + \gamma_n^mx_n$ and $\sum_{j=1}^n|\gamma_j^m|=1$ and $\lim_{m\ra\infty}\gamma_j^m = \beta_j$.
	Define $y = \beta_1x_1+\beta_2x_2+\dots+\beta_nx_n$.
	Thus $y_{n,m}\ra y$, where $\sum_{j=1}^n |\beta_j| = 1$.
	But $||y_{n,m}||\ra0$, and $||y||\neq0$ contradiction.
\end{proof}

\begin{theorem}
	Every finite dimensional subspace $Y$ of normed space $X$ is complete and closed.
\end{theorem}
\begin{corollary}
	Every finite dimensional normed space is a Banach space.
\end{corollary}

There are things that happen in infinite dimensional spaces that don't happen in finite dimensional spaces. For example, every finite dimensional vector space is complete; but not all infinite dimensional ones.

\ul{HW 2.3}: 1,3,10, G2;
\ul{HW 2.4}: 1,2,8, G6.

Last time: Every finite dimensional normed space is complete.
Use Lemma:
If $\set{x_1,\dots,x_n}$ is linearly independent in $X$, then $\exists c>0$ such that $||\alpha_1x_1+\dots+\alpha_nx_n|| \geq c(|\alpha_1|+\dots+|\alpha_n|)$.

\begin{definition}
	Let $X$ be a vector space, and let $||\cdot||$ and $||\cdot||_0$ be two norms on $X$.
	The norms are said to be \ul{equivalent} if there are positive numbers $a,b$ such that $\forall x\in X$ we have $a||x||_0 \leq ||x|| \leq b||x||_0$.
\end{definition}
\begin{remark}
	Note that $a||x||_0 \leq ||x||\leq b||x||_0 \implies \frac1b||x||\leq ||x||_0 \leq \frac1a||x||$ and $||x_n-x||\ra0$ implies $||x_n-x||_0\ra 0$ + vice versa.
\end{remark}
\begin{theorem}
	Let $X$ be a finite dimensional vector space.
	If $||\cdot||,||\cdot||_0$ are norms on $X$, they are equivalent.
\end{theorem}
\begin{proof}
	Let $X$ be finite dimensional and let $||\cdot||,||\cdot||_0$ be norms on $X$.
	Sufficient to show $\exists$ constants $c_1,c_2>0$ such that \ul{$||x||_0\leq c_1||x||$} and $||x||\leq c_2||x||_0$ for all $x\in X$.
	Let $\set{x_1,\dots,x_n}$ be a basis for $X$.
	Set $k = \max\setc{||x_j||_0}{j=1,\dots,n}>0$ and let $c>0$ be given by the Lemma in $||\cdot||$.
	Let $x\in X$. Then $x = \alpha_1x_1+\dots+\alpha_nx_n$.
	Then 
	\begin{align*}
		||x||_0 &= ||\alpha_1x_1+\dots+\alpha_nx_n||_0 \leq |\alpha_1|\,||x_1||_0 + \dots + |\alpha_n|\,||x_n||_0 \leq k(|\alpha_1|+\dots+|\alpha_n|) \\
				&\leq \frac kc||\alpha_1x_1+\dots+\alpha_nx_n|| \leq \frac kc||x|| = c_1||x||
	\end{align*}
	Similar argument shows that $||x||\leq c_2||x||_0$.
\end{proof}
\begin{example}
	An example of normed space with two norms that are not equivalent.
	We know it must be infinite dimensional by last result.
	$X = C[a,b]$ with norms $||\cdot||_\infty$ and $||\cdot||_1$.
	Not equivalent, $X$ with $||\cdot||_\infty$ is complete (Banach space) but $X$ with $||\cdot||_1$ is not complete.
\end{example}

\newpage

\begin{definition}
	A subset $M$ of a normed space (or metric space) $X$ is \ul{compact} if every sequence in $M$ has a subsequence that converges in $M$.
\end{definition}
\begin{lemma}
	A compact set $M$ is closed and bounded.
	Converse is not true in general, but it is true for compact subsets of $\mb R$ by Bolzano-Weierstrass theorem.
\end{lemma}
\begin{theorem}
	Let $X$ be a finite dimensional normed space.
	A subset $M$ is compact if and only if $M$ is closed and bounded.
\end{theorem}
\begin{example}
	A subset $M$ of a normed space which is closed and bounded, but not compact.
	Let $X = \ell^2$ with $||\cdot||_2$.
	Set 
	\begin{align*}
		e_1 &= (1,0,0,\dots) = (\delta_{1j})_{j=1}^\infty \\
		e_2 &= (0,1,0,\dots) = (\delta_{2j})_{j=1}^\infty \\
		e_n &= (\delta_{nj})_{j=1}^\infty
	\end{align*}
	$M = \set{e_1,e_2,\dots}$ is bounded because $||e_n||=1$ for $n=1,\dots$.
	Also $||e_m-e_n|| = \sqrt2$ for all $m\neq n$.
	So $M$ is closed, but $M$ is not compact.
	That is, $(e_n)_{n=1}^\infty$ is a sequence in $M$ with no convergent subsequence.
\end{example}

\begin{theorem}
	[Riesz Lemma]
	Let $Y,Z$ be subspaces of normed space $X$, and suppose $Y$ is a closed proper subspace of $Z$.
	Then for every $\theta\in(0,1)$ there exists $z\in Z$ such that $||z||=1$ and $||z-y||\geq\theta$ for all $y\in Y$.
\end{theorem}
\begin{proof}
	next time.
\end{proof}
\ul{HW} 2.5: 7

\end{document}
