\documentclass[]{article}
\usepackage[latin1]{inputenc}
\usepackage{graphicx}
\usepackage[left=1.00in, right=1.00in, top=1.10in, bottom=1.00in]{geometry}

\usepackage{dirtytalk}
\usepackage[normalem]{ulem}
\usepackage{tikz-cd}
\usepackage{units}
\usepackage{algorithm}
\usepackage{algpseudocode}
\usepackage{alltt}
\usepackage{mathrsfs}
\usepackage{amssymb}
\usepackage{amsmath}
\DeclareMathOperator\cis{cis}

% (font shortcuts)
\usepackage{amsfonts}
\newcommand{\mb}[1]{\mathbb{#1}}
\newcommand{\mc}[1]{\mathcal{#1}}
\newcommand{\ms}[1]{\mathscr{#1}}
\newcommand{\mf}[1]{\frak{#1}}

% (arrow shortcuts)
\newcommand{\ra}{\rightarrow}
\newcommand{\lra}{\longrightarrow}
\newcommand{\la}{\leftarrow}
\newcommand{\lla}{\longleftarrow}
\newcommand{\Ra}{\Rightarrow}
\newcommand{\Lra}{\Longrightarrow}
\newcommand{\La}{\Leftarrow}
\newcommand{\Lla}{\Longleftarrow}
\newcommand{\lr}{\leftrightarrow}
\newcommand{\llr}{\longleftrightarrow}
\newcommand{\Lr}{\Leftrightarrow}
\newcommand{\Llr}{\Longleftrightarrow}

% (match parenthesis)
\newcommand{\mlr}[1]{\left|#1\right|}
\newcommand{\plr}[1]{\left(#1\right)}
\newcommand{\blr}[1]{\left[#1\right]}

% (exponent shortcuts)
\newcommand{\inv}{^{-1}}
\newcommand{\nrt}[2]{\sqrt[\leftroot{-2}\uproot{2}#1]{#2}}

% (annotation shortcuts)
\newcommand{\conj}[1]{\overline{#1}}
\newcommand{\ol}[1]{\overline{#1}}
\newcommand{\ul}[1]{\underline{#1}}
\newcommand{\os}[2]{\overset{#1}{#2}}
\newcommand{\us}[2]{\underset{#1}{#2}}
\newcommand{\ob}[2]{\overbrace{#2}^{#1}}
\newcommand{\ub}[2]{\underbrace{#2}_{#1}}
\newcommand{\bs}{\backslash}
\newcommand{\ds}{\displaystyle}

% (set builder)
\newcommand{\set}[1]{\left\{ #1 \right\}}
\newcommand{\setc}[2]{\left\{ #1 : #2 \right\}}
\newcommand{\setm}[2]{\left\{ #1 \, \middle| \, #2 \right\}}

% (group generator)
\newcommand{\gen}[1]{\langle #1 \rangle}

% (functions)
\newcommand{\im}[1]{\text{im}(#1)}
\newcommand{\range}[1]{\text{range}(#1)}
\newcommand{\domain}[1]{\text{domain}(#1)}
\newcommand{\dist}[1]{(#1)}
\newcommand{\sgn}{\text{sgn}}

% (Linear Algebra)
\newcommand{\mat}[1]{\begin{bmatrix}#1\end{bmatrix}}
\newcommand{\pmat}[1]{\begin{pmatrix}#1\end{pmatrix}}
%\newcommand{\dim}[1]{\text{dim}(#1)}
\newcommand{\rnk}[1]{\text{rank}(#1)}
\newcommand{\nul}[1]{\text{nul}(#1)}
\newcommand{\spn}[1]{\text{span}\,#1}
\newcommand{\col}[1]{\text{col}(#1)}
%\newcommand{\ker}[1]{\text{ker}(#1)}
\newcommand{\row}[1]{\text{row}(#1)}
\newcommand{\area}[1]{\text{area}(#1)}
\newcommand{\nullity}[1]{\text{nullity}(#1)}
\newcommand{\proj}[2]{\text{proj}_{#1}\left(#2\right)}
\newcommand{\diam}[1]{\text{diam}\,#1}

% (Vectors common)
\newcommand{\myvec}[1]{\vec{#1}}
\newcommand{\va}{\myvec{a}}
\newcommand{\vb}{\myvec{b}}
\newcommand{\vc}{\myvec{c}}
\newcommand{\vd}{\myvec{d}}
\newcommand{\ve}{\myvec{e}}
\newcommand{\vf}{\myvec{f}}
\newcommand{\vg}{\myvec{g}}
\newcommand{\vh}{\myvec{h}}
\newcommand{\vi}{\myvec{i}}
\newcommand{\vj}{\myvec{j}}
\newcommand{\vk}{\myvec{k}}
\newcommand{\vl}{\myvec{l}}
\newcommand{\vm}{\myvec{m}}
\newcommand{\vn}{\myvec{n}}
\newcommand{\vo}{\myvec{o}}
\newcommand{\vp}{\myvec{p}}
\newcommand{\vq}{\myvec{q}}
\newcommand{\vr}{\myvec{r}}
\newcommand{\vs}{\myvec{s}}
\newcommand{\vt}{\myvec{t}}
\newcommand{\vu}{\myvec{u}}
\newcommand{\vv}{\myvec{v}}
\newcommand{\vw}{\myvec{w}}
\newcommand{\vx}{\myvec{x}}
\newcommand{\vy}{\myvec{y}}
\newcommand{\vz}{\myvec{z}}
\newcommand{\vzero}{\myvec{0}}


\author{Book: Walter Rudin 3rd, Presenter: Maya Chhetri, Notes by Michael Reed}
\title{Real Analysis I}
%date{}

\begin{document}
\maketitle

%\begin{abstract}
%\end{abstract}

\section{Chapter 1}

\ul{Notation}: $\mb{N} = \{0,1,2,3,\dots\}$ - set of natural numbers.
$\mb Z = \{0,\pm 1, \pm 2,\dots\}$ - set of integers.
$\mb Q = \{\frac{m}{n}:m,n\in\mb Z,\, n\neq 0\}$ - set of rationals $\ra$ we know $\mb Q$.
$\mb R$ - set of all real numbers.
$\mb C = \{a+ib: a,b\in\mb R \text{ and } i^2 = -1\}$.

\ul{Sets} - collection of objects. Empty set - $\emptyset$ - a set which contains no objects.
$A\subset B$ if $a\in A \implies a\in B$ for all $a\in A$.
\begin{example}
	$A\varsubsetneq B$ - a proper subset if $A\subset B$ and $\exists x_0 \in B$ such that $x_0 \notin A$.
\end{example}
\begin{note}
	$A = B$ iff $A\subset B$ and $B\subset A$.
\end{note}

\ul{Order on a set}: Let $S$ be a set. An order, denoted by '$<$' is a \ul{relation} on $S$ satisfying:
\begin{enumerate}
	\item[(i)] If $x,y\in S$ then one of the following must hold:
	$x<y$, $x = y$, $y<x$.
	\item[(ii)] If $x,y,z\in S$ with $x<y$ and $y<z$ then $x<z$ (transitive property).
\end{enumerate}
\begin{example}
	$S = \mb N$ with \say{$<$} ($x<y$ if $y-x>0$) is an ordered set.
\end{example}
\begin{definition}
	[Upper bound] Let $(S,<)$ be an ordered set. Let $E\subset S$. We say that $\beta$ is an \ul{upper bound of $E$} if $s\leq\beta$ for all $s\in E$.
\end{definition}
\begin{definition}
	[Lower bound] $\alpha$ is a \ul{lower bound of $E$} if $\alpha \leq s$ for all $s\in E$.
\end{definition}

\begin{example}
	$(S = \mb Q,<)$ and
	\begin{enumerate}
		\item $E = \{x \in \mb Q: 0\leq x < 1\}$. $\beta = 1$ is an upper bound of $E$ and $1\notin E$. $\beta \geq 1$ are upper bounds of $E$. $\alpha = 0$ is a lower bound of $E$ and $0\in E$. $\alpha \leq 0$ are lower bounds of $E$.
	\end{enumerate}
\end{example}
\begin{definition}
	[Least upper bound] Let $E\subset (S,<)$. We say that $\alpha \in S$ is the least upper bound or supremum of $E$ if:
	\begin{itemize}
		\item[(i)] $\alpha$ is an upper bound of $E$, and
		\item[(ii)] if $\gamma$ is an upper bound of $E$, then $\alpha \leq \gamma$ (or $\gamma $ is not an upper bound for any $\gamma < \alpha$).
	\end{itemize}
\end{definition}
\ul{We write}: $\alpha = \text{lub} E$ or $\alpha = \sup E$.
\begin{example}
	[continued] $\sup E = 1 \notin E$, $\inf = 0 \in E$. $\max E$ is a the $\sup E$ and it belongs to $E$.
\end{example}

Rational numbers $\mb Q$ has holes.
\begin{lemma}
	$\sqrt{2} \notin \mb Q$. That is $x^2 = 2$ has no solution in $\mb Q$.
\end{lemma}
\begin{proof}
	We will prove by contradiction (BWOC). Suppose $\sqrt{2} \in \mb Q$. Then $\sqrt{2} = \frac{m}{n}$, with $m,n\in \mb Z$, $n\neq 0$ and $\frac{m}{n}$ in least terms or $\gcd (m,n) = 1$. Then $2 = \frac{m^2}{n^2} \implies m^2 = 2n^2$. Then $m^2$ is even. \ul{So $m$ is even}.
	\begin{proof}
		Suppose $m$ is odd. Then $m = 2k + 1$ for some $k \in \mb Z$. $\implies m^2 = (2k+1)^2 = 4k^2 + 4k + 1 = 2(2k^2 + 2k) + 1 \implies m^2$ is odd, a contradiction. So $m$ must be even.
	\end{proof}
	Then $m = 2l$ for some $l\in \mb Z$. Then $(2l)^2 = 2n^2 \implies 4l^2 = 2n^2 \implies 2l^2 = n^2$. So $n^2$ is even and hence $n$ is even. This is a contradiction to $\frac{m}{n}$ being in least terms. So $\sqrt{2} \notin \mb Q$.
\end{proof}

Prove that $\sup E = 1$ if $E = \{x \in \mb Q : 0 \leq x < 1\}$.

\begin{proof}
	\begin{enumerate}
		\item[(i)] Clearly $x<1$ for all $x\in E$. So $1$ is an upper bound of $E$.
		\item[(ii)] If $\gamma < 1$ then $\gamma$ is not an upper bound of $E$. If $\gamma \leq 0$, then clearly $\gamma$ is not an upper bound of $E$. If $0<\gamma < 1$, then $x = \frac{\gamma + 1}{2}$. But $\gamma < x\in E$ by construction, so $\gamma$ is not an upper bound. Therefore \ul{$ \sup E = 1$}.
	\end{enumerate}
\end{proof}

\begin{theorem}
	[Uniqueness of supremum] Let $(S,<)$ be an ordered set. If $E\subset S$ has supremum (or infimum) in $S$, then it is unique.
	\begin{proof}
		Let $\alpha_1 = \sup E$ and $\alpha_2 = \sup E$. Then by definition $\alpha_1$ and $\alpha_2$ are upper bounds of $E$. But $\alpha_1 = \sup E$ and $\alpha_2$ is an upper bound of $E$, so $\alpha_1 \leq \alpha_2$. Switching roles of $\alpha_1$ and $\alpha_2$, we get $\alpha_2 \leq \alpha_1$. Combining these two, $\alpha_1 = \alpha_2$. So $\sup E$ is unique.
	\end{proof}
\end{theorem}
\begin{example}
	$A := \{ x\in\mb Q : x>0 \text{ and } x^2 \leq 2\}$. $B:=\{ x\in\mb Q: x>0 \text{ and } x^2\geq 2\}$. $A\cap B = \{ x \in\mb Q : x>0 \text{ and } x^2 = 2\} = \emptyset$ - showed.
	\begin{enumerate}
		\item $A$ is bounded above in $\mb Q$ ($B$ is bouned below in $\mb Q$).
		If $a\in A$ and $b\in B$ then $a\leq b$.
		\item There is no upper bound of $A$ in $A$.
		Let $p>0\in \mb Q$ be fixed and arbitrary. Set $q:= p - \frac{p^2-2}{p+2} = \frac{2(p+2)}{p+2}$. $q^2-2 = \frac{4p^2 + 8p +4}{(p+2)} - 2 = \frac{4p^2 + 8p + 4 - 2p^2 - 8p - 8}{(p+2)^2} = \frac{2(p^2-2)}{(p+2)^2}$. So, $q^2 < 2$ if and only if $p^2 < 2$. Therefore $q\in A$ if and only if $p\in A$.
		
		Suppose $p\in A$ is an upper bound of $A$. Then $q = \frac{2p+2}{p+2} > p$ and $q\in A$. Therefore $p$ cannot be an upper bound of $A$.
		\item $A$ has no least upper bound or supremum in $\mb Q$.
		$B$ is the set of upper bounds of $A$. Repeating the argument of 2. for $B$, there is no lower bound of $B$ in $B$.
		That means $A$ has no least upper bound in~$\mb Q$.
	\end{enumerate}
\end{example}

\begin{definition}
	[Least upper bound property] Let $(S,<)$ be an ordered set. Then $S$ is said to have the least upper bound property if every nonempty subset of $S$ that is bounded above has supremum in $S$ i.e., $\forall E\subset S$ with $E\neq \emptyset$ and bounded above, $\sup E \in S$.
\end{definition}
\begin{example}
	$(Q,<)$ (usual ordering) does not have least upper bound property.
	
	\begin{proof}
		[Reason] $A = \{x\in\mb Q: x>0 \text{ and } x^2\leq 2\}\subset \mb Q$. 
		\begin{itemize}
			\item $A\neq\emptyset$ since $1\in \mb Q$.
			\item $A$ is bounded above (by $B$).
		\end{itemize}
		But $\sup A$ does not exist in $\mb Q$
	\end{proof}
\end{example}

\begin{theorem}
	Let $(S,<)$ with the least upper bound property. Let $B\subset S$ with $B\neq \emptyset$ and $B$ is bounded below.
	Set $L = $ set of all lower bounds of $B$. Then $\alpha = \sup L$ exists in $S$, and $\alpha = \inf B$.
	\begin{proof}
		First, we want to show:
		\begin{enumerate}
			\item[(i)] $L$ is nonempty, and
			\item[(ii)] $L$ is bounded above.
		\end{enumerate}
		$L$ is nonempty since $B$ is bounded below and hence $\exists l\in S$ such that $l\leq b$ for all $b\in B$. By definition of $L$, $l \leq b$ for all $l\in L$ and $\forall b\in B$ and $B\neq\emptyset$. So $L$ is bounded above.
		By the least upper bound property of $S$, $\sup L\in S$. Define $\alpha = \sup L$. \ul{We claim}:
		\begin{enumerate}
			\item[(a)] $\alpha$ is a lower bound of $B$, and
			\begin{proof}
				$\alpha = \sup L$ and if $b\in B$ then $\alpha \geq l$ for all $l\in L$. But $l\leq b$ for $b\in B$, so $b$ is an upper bound of $L$. Therefore, $\alpha \leq b$ since $\alpha = \sup L$.
			\end{proof}
			\item[(b)] $\gamma>\alpha$ is not a lower bound of $B$.
			\begin{proof}
				If $\gamma > \alpha$ then $\gamma \notin L$ and so $\alpha = \inf B$.
			\end{proof}
		\end{enumerate}
		(a), (b) $\implies \alpha = \inf B$.
		
	\end{proof}
\end{theorem}

\begin{definition}
	[Field] A field $F$ is a set with two operations
	\begin{itemize}
		\item addition '+'
		\item multiplication '$\cdot$'
	\end{itemize}
	Satisfying the following axioms:
	\begin{enumerate}
		\item[(A1)] $x+y \in F$, $\dots$ (A5).
		\item[(M1)] $x\cdot y \in F$, $\dots$ (M5).
		\item[(D)] $x\cdot(y+z) = x\cdot y + x\cdot z$ for all $x,y,z\in F$.
	\end{enumerate}
\end{definition}

\begin{proposition}
	\label{prop1.14}
	Let $(F,+\cdot)$ be a field. Then for $x,y,z\in F$,
	\begin{enumerate}
		\item[(a)] If $x+y = x+z$, then $y = z$. (Cancellation law)
		\begin{proof}
			$y \os{A4}{=} 0+y \os{A5}{=} (-x+x) + y \os{A3}{=} -x+(x+y) \os{A3}{=} -x+(x+z)\equiv (-x+x)+z \os{A5}{=} 0+z \os{A4}{=} z$
		\end{proof}
		\item[(b)] If $x+y = x$, then $y = 0$.
		\begin{proof}
			Taking $z=0$ in (a), we get $x+y = x+0 \implies y=0$.
		\end{proof}
		\item[(c)] If $x+y = 0$, then $y = -x$.
		\item[(d)] $-(-x)=x$.
		\begin{proof}
			Let $x\in F$. Then $-x\in F$ such that $x+(-x)=0$. Since $-x\in F$, $\exists -(-x)\in F$ such that $-(-x)+(-x)=0$. So $x$ and $-(-x)$ are additive inverse of $-x$, by (c) $x=-(-x)$.
		\end{proof}
	\end{enumerate}
\end{proposition}

\begin{proposition}
	Let $(F,+,\cdot)$ a field with $x,y,z\in F$. Then
	\begin{enumerate}
		\item[(a)] If $x\neq 0$ and $xy=xz$ then $y=z$.
		\begin{proof}
			Let $x\neq 0$ and $xy = xz$. $y \os{M4}{=} 1\cdot y \os{M5}{=} (\frac{1}{x}\cdot x)\cdot y \os{M3}{=} \frac{1}{x}(xy) \equiv \frac{1}{x}(xz) \os{M3}{=} (\frac{1}{x}\cdot x)\cdot z \os{M5}{=} 1\cdot z \os{M4}{=} z$
		\end{proof}
		\item[(b)] If $x\neq 0$ and $xy=x$ then $y=1$.
		\item[(c)] If $x\neq 0$ and $xy = 1$ then $y=\frac{1}{x}$.
		\item[(d)] If $x\neq 0$, then $\frac{1}{1/x} = x$.
	\end{enumerate}
\end{proposition}

\begin{proposition}
	\label{prop1.16}
	Let $x,y,z\in (F,+,\cdot)$ field. Then
	\begin{enumerate}
		\item[(a)] $0\cdot x = 0$
		\begin{proof}
			$0\cdot x = (0+0)\cdot x \os{D}{=} 0\cdot x + 0\cdot x$. Then $0\cdot x = 0$ (Proposition \ref{prop1.14}(b)).
		\end{proof}
		\item[(b)] If $x\neq 0$ and $y\neq 0$ then $xy\neq 0$.
		\begin{proof}
			We'll prove by contradiction. Let $x\neq 0$, $y\neq 0$ but $xy=0$. Then $1 \os{M4}{=} 1\cdot 1 = (x\cdot \frac{1}{x})(y\cdot \frac{1}{y})$ because $\exists \frac{1}{x},\frac{1}{y}\in F$ such that $x\cdot \frac{1}{x} = 1$ and $y\cdot \frac{1}{y} = 1$. Then $(x\cdot \frac{1}{x})(y\cdot \frac{1}{y}) \os{M2-M3}{=} (xy)(\frac{1}{x}\cdot\frac{1}{y}) \equiv 0(\frac{1}{x}\cdot\frac{1}{y}) \os{(a)}{=} 0 \implies 1 =0$, a contradiction to the assumption that $1\neq 0$ (M4). So $xy\neq 0$.
		\end{proof}
		\item[(c)] $(-x)y = -(xy) = x(-y)$.
		\begin{proof}
			Need to show $(-x)y$ is an additive inverse of $xy$. $(-x)y + xy \os{D}{=} (-x+x)y \os{A5}{=} 0\cdot y \os{(a)}{=} 0 \implies (-x)y = -(xy)$ by Proposition \ref{prop1.14}(c). Similarly, $-(xy) = x(-y)$.
		\end{proof}
		\item[(d)] $(-x)(-y) = xy$.
		\begin{proof}
			$(-x)(-y) \os{(c)}{=} -(x(-y)) \os{(c)}{=} -[-(xy)] \os{\text{Prop \ref{prop1.14}(d)}}{=} xy$
		\end{proof}
	\end{enumerate}
\end{proposition}

Field, Ordered set $\}\implies$ Ordered field.

\begin{definition}
	An ordered field $(F,+,\cdot,<)$ is a field which is also an ordered set and satisfies:
	\begin{enumerate}
		\item[(i)] If $y<z$ then $x+y<x+z$ for all $x\in F$.
		\item[(ii)] If $x>0$ and $y>0$ then $xy>0$.
	\end{enumerate}
\end{definition}

\begin{example}
	$(\mb Q,+,\cdot,<)$ is an ordered field with:
	\begin{enumerate}
		\item[+:] $\frac{a}{b} + \frac{c}{d} = \frac{ad+bc}{bd}$ for $\frac{a}{b},\frac{c}{d}\in\mb Q$.
		\item[$\cdot$:] $\frac{a}{b}\cdot\frac{c}{d} = \frac{ac}{bd}$ for $\frac{a}{b},\frac{c}{d}\in\mb Q$.
		\item[$<$:] $\frac{a}{b}<\frac{c}{d}$ if $ad<bc$.
	\end{enumerate}
\end{example}

\begin{proposition}
	Let $(F,+,\cdot,<)$ an ordered field. Then
	\begin{enumerate}
		\item[(a)] If $x>0$ then $-x<0$. (If $x<0$ then $-x>0$).
		\begin{proof}
			Let $x>0$. Then $0 \os{A5}{=} -x + x \os{\text{(i)OF}}{>} -x+0 \os{A4}= -x \implies 0>-x$. Similarly, $\dots$
		\end{proof}
		\item[(b)] If $x>0$ and $y<z$ then $xy<xz$.
		\begin{proof}
			Let $x>0$ and $y<z$. Since $y<z\implies z-y >y-y = 0$. Then $x(z-y) \os{(a)}{>} 0$. Then $xz = xz + 0 = xz - xy + xy = x(z-y) + xy \os{(i)}{>} 0 + xy = xy \implies xy <xz$.
		\end{proof}
		\item[(c)] If $x<0$ and $y<z$ then $xy>xz$.
		\begin{proof}
			Try it.
		\end{proof}
		\item[(d)] If $x\neq 0$ then $x^2 = x\cdot x >0$. In particular $1>0$.
		\begin{proof}
			Let $x\neq 0$. Then $x>0$ or $x<0$ (ordered set). If $x>0$ then $x^2 = x\cdot x \os{\text{OF}(ii)}{>}0$. Let $x<0$. Then $-x>0 \implies 0<(-x)(-x) \os{\text{Prop \ref{prop1.16}(d)}}{=} x\cdot x = x^2$. In particular, $1\neq 0$, so $1^2 = 1 >0$.
		\end{proof}
		\item[(e)] If $x> 0$ then $\frac{1}{x}>0$.
		\begin{proof}
			Let $x>0$. Suppose that $\frac{1}{x}\geq 0$. Then $-\frac{1}{x}\geq 0$. Then $0 = 0\cdot x \leq (-\frac{1}{x})\cdot x = -1 \implies 0\leq -1$ ($1\leq 0$), a contradiction to $0<1$.
		\end{proof}
		\item[(f)] If $0<x<y$ then $\frac{1}{x}>\frac{1}{y}$.
	\end{enumerate}
\end{proposition}

\begin{theorem}
	[Real field] There exists an ordered field, say $(\mb R,+,\cdot,<)$, which has the least upper bound property. Moreover, $\mb R$ contains $\mb Q$ as a subfield.
\end{theorem}
\begin{note}
	$\mb R$ is called real field and elements of $\mb R$ are called real numbers.
\end{note}

\begin{theorem}
	\begin{enumerate}
		\item[(a)] Let $x,y\in\mb R$ with $x>0$. Then there exists $n\in\mb N\backslash\{0\}$ such that $n\cdot x>y$. (Archimedean property of $\mb R$).
		\item[(b)] Let $x,y\in\mb R$ with $x<y$. Then $\exists q\in\mb Q$ such that $x<q<y$. (Denseness of $\mb Q$ in $\mb R$).
	\end{enumerate}
	\ul{Special case}: For each $y\in\mb R$, $\exists n\in\mb N\backslash \{0\}$ such that $n>y$. (Taking $x=1>0$).
\end{theorem}
\begin{proof}
	\begin{enumerate}
		\item[(a)] Let $x,y\in\mb R$ with $x>0$. Suppose by contradiction that $n\cdot x \geq y$ for all $n\in\mb N\backslash\{0\}$. Set $A:=\{nx:n\in\mb N\backslash\{0\}\}$. $A$ is nonempty since $x = 1\cdot x\in A$. $A$ is bounded above by $y$ since $nx\leq y$ for all $n\in\mb N$. Since $\mb R$ has the least upper bound property, $\sup A\in\mb R$. $\alpha:= \sup A\in\mb R$. Now, $x>0$, so $-x<0$. $\alpha-x < x+0 = \alpha = \sup A \implies \alpha - x$ is not an upper bound of $A$ $\implies \exists m\in\mb N$ such that $m\cdot x\in A$ and $\alpha-x<mx<\alpha$. $\implies \alpha= \alpha - x + x < mx + x = (m+1)x\in A$, a contradiction that $\alpha = \sup A$. Therefore $\exists n\in\mb N\backslash \{0\}$ such that $nx>y$.
		\item[(b)] Want to construct $q = \frac{m}{n}$, $m,n\in\mb Z$ and $n\neq 0$ such that $x<\frac{m}{n}<y$. Since $x<y$, $y-x>0$. By part (a), $\exists n\in\mb N\backslash \{0\}$ such that $n(y-x)>1$. Next, we find $m$. Since $1>0$ by (a), $\exists k_1,k_2\in\mb N\backslash \{0\}$ such that $k_1>ny$ and $k_2>-nx$ ($\implies -k_2<nx$). Combining $-k_2<nx<ny<k_1$. We're done if $\exists m\in\mb Z$ such that $nx<m<ny$. Define $S:=\{j\in\mb Z:-k_2\leq j\leq k_1 \text{ and } j>nx\}$. Then $S$ is finite, $S\neq \emptyset$ since $k_1\in S$, $S$ is bounded below by $-k_2$. So $\inf S = \min S$ exists, say $m = \min S$. $\implies m>nx$ by construction. Since $m=\min S$, $m-1 \leq nx$. Then $m = (m-1)+1 \leq nx+1 < nx+n(y-x) = ny$. So $nx<m<ny$ and hence $x<\frac{m}{n}<y$, where $q = \frac{m}{n}\in\mb Q$.
	\end{enumerate}
\end{proof}

\begin{theorem}
	\label{thm1.21}
	\begin{enumerate}
		\item For every positive real number $x$ and every $n\in\mb N\backslash \{0\}$, there exists a unique $y\in\mb R$, $y>0$ so that $y^n = x$ ($y$ is called $n^\text{th}$ root of $x$). $y = ^n\sqrt{x}$ or $y = x^{1/n}$.
		\item For every $n\in\mb N\backslash \{0\}$, there exists a unique $y\in\mb R$, namely \ul{$y=0$}, such that $0^n = 0$ (or $0^{1/n} = 0$).
		\item If $0<a<b$ then $0<a^{1/n} < b^{1/n}$ (monotonicity of $n^\text{th}$ root).
	\end{enumerate}
\end{theorem}
\begin{proof}
	Uniqueness of $n^\text{th}$ root.
	
	\ul{Case}: When $x=0$. Suppose $y^n = 0$ given fixed $n$ but $y\neq 0$. Then $y^2 \neq 0$, and by induction $y^k \neq 0$ for all $k\in \mb N\backslash\{0\}$, a contradiction to the fact that $y^n = 0$ for some fixed $n\in\mb N$. So $y=0$ is the only $n^\text{th}$ root for $x=0$.
	
	\ul{Case}: When $x>0$. Suppose $\exists y_1,y_2\in\mb R$ with $y_1,y_2>0$ and $y_1\neq y_2$ such that $y_1^n = x = y_2^n$. WLOG assume $y_1<y_2$. If $0<a<b$ then $0<a^k<b^k$ for any $k\in\mb N\backslash \{0\}$ and $a,b\in\mb R$. By induction: Base case: $a<b$. Suppose $a^{k-1} < b^{k-1}$. Then $a^k = a\cdot a^{k-1} < a\cdot b^{k-1} \underset{a<b}{<} b\cdot b^{k-1} = b^k$. By induction $0<a^k<b^k$ for all $k\in\mb N\backslash \{0\}$. Then $y_1^n<y)2^n$, a contradiction to $y_1^n = y_2^n$.
	
	\ul{Existence of $n^\text{th}$ root}: If $x=0$ then $0^n = 0$ and therefore it follows from the uniqueness that $y=0$. Let $x>0$. If $n=1$, then $y^n = y^1 = x$. Assume $n\geq 2$. Define $S = \{t\in\mb R: t^n < x\}$. $S\neq \emptyset$ because $0\in S$. $S$ is bounded above: find an upper bound of $S$. Let $\alpha := 1+x$. Want to show that $\alpha$ is an upper bound of $S$, i.e. $\alpha \geq t$ for all $t\in S$. Equivalently, we show that $\alpha < t$ then $t\notin S \implies t^n > (1+x)^n = (1+x)(1+x)^{n-1} > (1+x)\cdot 1 \underset{1>0}{>} x \implies t\notin S \implies \alpha$ is an upper bound of $S$ or $S$ is bounded above $\implies \sup S\in \mb R$.
\end{proof}

\newpage

\begin{recall}
	\begin{enumerate}
		\item[(i)] If $x\in \mb R$, $x>0$, then $\exists! y\in\mb R$, $y>0$ such that $y^n = x$.
		\item[(ii)] If $x=0$, then $y=0$ is the only real number such that $0^n = 0$.
		\item[(iii)] $0<a<b\implies 0<a^{1/n}<b^{1/n}$.
	\end{enumerate}
\end{recall}
\begin{proof}
	In last class, $S := \{t\in \mb R: t^n < x\}$ and $y = \sup S\in\mb R$. We need to show $y^n = x$. We will show $y^n \nless x$ and $y^n \ngtr x$. \ul{So $y^n = x$}.
	Useful inequality: Let $0<a<b$. Then $$b^n - a^n = (b-a)(b^{n-1}+b^{n-2}a + b^{n-3}a^2 + \dots + ba^{n-2} + a^{n-1}) < (b-a)\cdot nb^{n-1}.$$
	\ul{Case I}: Suppose $y^n < x$. (Find $y_1>y$, $y_1\in S$ such that $y_1^n > x$). Let $0<h<1$, $h\in\mb R$ such that $h< \frac{x-y^n}{n(y+1)^{n-1}}$. Taking $a = y$ and $b= y+h$ in the inequality, we get $$(y+h)^n - y^n < h\cdot n\cdot (y+h)^{n-1} < h\cdot n(y+1)^{n-1} < x-y^n$$ implies $(y+h)^n < x$ implies $y+h\in S$. But $y+h>y$ and $y=\sup S$, a contradiction.
	\ul{Case II}: Assume $y^n > n$. (Find $y_2 < y$ so that $y_2$ is an upper bound of $S$). Define $k = \frac{y^n - x^{>0}}{ny^{n-1}} < \frac{y^n}{n\cdot y^{n-1}} = \frac{y}{n} \underset{n\geq 2}{<} y$. \ul{Claim}: $y - k > 0$ is an upper bound of $S$. $y^n - (y-k)^n < k\cdot n y^{n-1} \underset{\text{choice of }k}{=} y^n - x$ implies $-(y-k)^n < -x$ implies $(y-k)^n > x$ implies $y-k\notin S$ implies $y-k$ is an upper bound of $S$. But $y-k<y = \sup S$, a contradiction. Therefore, $y^n = x$.
	
	(iii) Suppose $0<a<b$ but $a^{1/n}\geq b^{1/n}$. Then $b = (b^{1/n})^n \us{\text{power are monotone}}{\leq} (a^{1/n})^n = a$, a contraction to $a<b$. This completes the proof of Theorem \ref{thm1.21}.
\end{proof}
\begin{corollary}
	Let $a,b\in\mb R$ and $a\geq 0,b\geq 0$. Then $(ab)^{1/n} = a^{1/n}\cdot b^{1/n}$.
\end{corollary}
\begin{proof}
	By the uniqueness of the $n^\text{th}$ root, it is enough to show $(a^{1/n}\cdot b^{1/n})^n = ab$. Now, $ab = (a^{1/n})^n (b^{1/n})^n = a^{1/n}\cdots a^{1/n}\cdot b^{1/n}\cdots b^{1/n} = (a^{1/n}b^{1/n})$ respectively $n$ times and using commutativity and associativity.
\end{proof}

\subsection*{Decimal expansion of real numbers}

Let $x\in\mb R$ and $x>0$. By the decimal expansion of $x$, we mean a number of the form $n_0. n_2 n_2 n_3\cdots$ where $n_i$ are defined inductively as follows:
Let $n_0$ be the largest integer such that $n_0\leq x$, i.e., $n_0 = \max\{k\in\mb N: k\geq x\}$. Let $n_1$ be the largest integer such that $n_0 + \frac{n_1}{10} \leq x$, i.e., $n_1 = \max\{k\in\mb N : n_0 + \frac{k}{10} \leq x\}$. Once $n_0,\dots,n_{k-1}$ are chosen, $n_k$ is chosen as follows: $n_k = \max\{j\in\mb N: n_0 + \frac{n_1}{10} + \frac{n_2}{10^2} + \cdots + \frac{n_{k-1}}{10^{k-1}} + \frac{j}{10^k} \leq x\}$.
\begin{note}
	By construction, $n_k\in\{0,1,2\dots,8,9\}$.
\end{note}
\begin{remark}
	Let $x\in\mb R$ and $x>0$. Let $n_0.n_1n_2\dots n_k\dots$ be a decimal expansion of some real number. Define $E:=\{n_0 + \frac{n_1}{10} + \frac{n_2}{10^2} + \dots + \frac{n_k}{10^k} : k\in\mb N\}$. Then $x\sup E$ if and only if $x = n_0.n_1n_2n_3\dots$.
\end{remark}

\subsection*{Extended real number system}

$\mb R^\# := \mb R\cup\{-\infty,\infty\}$. Order on the set $\mb R^\#$: $-\infty < x<\infty$ for all $x\in\mb R$. $\mb R^\#$ is not a field (Not closed under \say{usual} addition, i.e. $-\infty + \infty \os{?}{\in}\mb R^\#$. Use these conventions: $x+\infty = \infty$, $x-\infty = -\infty$, $\frac{1}{\infty} = 0$, $\frac{1}{-\infty} = 0$. For $x\in\mb R^\#$, $x>0$: $x\cdot(+\infty) = \infty$, $x\cdot (-\infty) = -\infty$. For $x\in\mb R^\#$, $x<0$: $x\cdot(+\infty) = -\infty$, $x\cdot(-\infty) = \infty$.

For any $A\subset \mb R^\#$, $\sup A\in\mb R^\#$, $\inf A\in\mb R^\#$. In particular, if $A$ is not bounded above, then $\sup A = \infty$. If $A$ is not bounded below, then $\inf A = -\infty$. Also, $\sup \emptyset = -\infty$ and $\inf \emptyset = \infty$ because every element real number is an upper bound for empty set and every real number is a lower bound for empty set.

\subsection*{Complex Field}

Dictionary order: $(a,b) < (c,d)$ if $\begin{cases} a<c , \\ a=c, & b<d\end{cases}$.

\begin{definition}
	Set of all complex numbers $\mb C :=\{(a,b):a,b\in\mb R\}$, where $(a,b)$ is an ordered pair, i.e. $(a,b)\neq (b,a)$ if $a\neq b$.
	If $z = (a,b)\in\mb C$, $a = \Re(z)$, $b=\Im(z)$.
\end{definition}

Addition \say{$+_\mb C$}: $z_1= (a,b)\in\mb C$ and $z_2 = (c,d)\in\mb C$. Then $z_1+z_2 = (a +_\mb R c, b+_\mb R d)$.

Multiplication \say{$\cdot_\mb C$}: $z_1\cdot z_2 = (ac-bd, bc+ad)$.

Additive identity: $0_\mb C = (0,0)$.

Multiplicative identity: $1_\mb C = (1,0)$.

Then $(\mb C,+_\mb C,\cdot_\mb C)$ is a field.

\begin{recall}
	$(\mb C,+_\mb C,\cdot_\mb C)$ is a field.
\end{recall}
\begin{note}
	$\mb R$ is a subfield of $\mb C$, let $f:\mb R\ra \mb C$ by defining $f(a) = (a,0)$. It allows us to identify $a\in\mb R$ as $a\in\mb C$.
\end{note}
\begin{definition}
	[\say{$i$}] We define the imaginary unit $i$ as $(0,1)\in\mb C$.
\end{definition}
\begin{theorem}
	$i^2 = -1$.
\end{theorem}
\begin{proof}
	$i^2 = i\cdot i = (0,1)\cdot_\mb C(0,1) = (0-1,0) = (-1,0)\in\mb C = -1\in\mb R$.
\end{proof}
Given $z=(a,b)\in\mb C$, we can write $z = a+bi$. Indeed, $a+bi = (a,0) +_\mb C (b,0)(0,1) = (a,0)+_\mb C(0,b) = (a,b)$.
\begin{definition}
	[Conjugate] For $z = a+bi = (a,b)\in\mb C$, the conjugate of $z$ is $\conj z\in\mb C$, where $\conj z = a-bi = (a,-b)$.
\end{definition}
\begin{theorem}
	[Conjugate] Let $z,w\in\mb C$, $z = a+ib$, $w = c+id$; where $a,b,c,d\in\mb R$. Then
	\begin{enumerate}
		\item[(a)] $\conj{z+w} = \conj z+\conj w$
		\begin{proof}
			$\conj{z+w} = \conj{(a+ib)+(c+id)} = \conj{(a+c)+i(b+d)} = (a+c)-i(b+d) = a+c-ib-id = (a-ib)+(c-id) = \conj{a+ib} + \conj{c+id} = \conj z + \conj w$.
		\end{proof}
		\item[(b)] $\conj{z\cdot w} = \conj z\cdot \conj w$
		\begin{proof}
			skip!
		\end{proof}
		\item[(c)] $z+\conj z = 2\Re(z)$ and $z-\conj z = 2i\Im(z)$
		\begin{proof}
			skip!
		\end{proof}
		\item[(d)] $z\cdot \conj z\in\mb R$, $z\cdot \conj z \geq 0$ and $z\cdot\conj z = 0 \Lr z = 0$
		\begin{proof}
			$z\cdot \conj z = (a+ib)(a-ib) = a^2+b^2\in\mb R$. If $a\in\mb R$ and $b\in\mb R$ then $a^2\geq 0$ and $b^2\geq 0 \implies z\cdot \conj z \geq 0$. Now suppose $z = 0$. Then $z\cdot \conj z = 0\cdot \conj z = 0$. Suppose $z\cdot \conj z = 0$. Then $a^2+b^2 = 0$, which implies $a=0$ and $b=0$. So $z = 0+0i = 0$.
		\end{proof}
		\item[(e)] $\conj z = z$
		\begin{proof}
			skip!
		\end{proof}
		\item[(f)] $z\in\mb R \implies \conj z = z$.
		\begin{proof}
			skip!
		\end{proof}
	\end{enumerate}
\end{theorem}
\begin{remark}
	$f:\mb C\ra \mb C$ by $f(z) = \conj z$. By Theorem, $f(z+w) = f(z)+f(w)$ for all $z,w\in\mb C$ and $f(z\cdot w) = f(z)\cdot f(w)$ for all $z,w\in\mb C$ and $f(0) = 0$ and $f(1) = 1$. This implies conjugate preserves field properties of $\mb C$. However, $f$ is not an identity mapping since $f(i) = -i$.
\end{remark}
\begin{definition}
	[Absoulute value, modulus] Given $z\in\mb C$, modulus or absolute value of $z$ is: $|z|:=(z\cdot \conj z)^\frac{1}{2}$ - well defined by $n^\text{th}$ root theorem since $0\leq z\cdot \conj z\in \mb R$.
\end{definition}
\begin{note}
	If $x\in\mb R$, then $|x| = x\cdot x^{1/2} = (x^2)^\frac{1}{2} \us{x\in\mb R}{=} ((-x)^2)^\frac{1}{2} \implies |x| = \begin{cases} x & \text{if } x\geq 0\\ -x & \text{if } x<0\end{cases}$.
\end{note}
\begin{observe}
	$|x|^2 = x^2 = (-x)^2$ for all $x\in\mb R$. But the identity is not true in $\mb C$. Take $z = i$, so $|z|^2 = |i|^2 = i\cdot\conj i = i(-i) = 1$. But $z^2 = i\cdot i = i^2 = -1$.
\end{observe}
\begin{theorem}
	Let $z,w\in\mb C$. Then
	\begin{enumerate}
		\item[(a)] $|z|\geq 0$ and $|z|= 0 \Lr z= 0$.
		\begin{proof}
			Since $z\cdot \conj z \geq 0$ and $|z| = (z\cdot\conj z)^{1/2}$, $|z|\geq 0$ follows by $n^\text{th}$ root theorem. By previous theorem, $z\cdot \conj z = 0\Lr z = 0$. So $|z|=0\Lr z = 0$.
		\end{proof}
		\item[(b)] $|z\cdot w| = |z|\cdot |w|$
		\begin{proof}
			$|z| = (\conj z\cdot \conj{\conj z})^\frac{1}{2} = (\conj z\cdot z)^\frac{1}{2} = |z|$.
		\end{proof}
		\item[(c)] $|z\cdot w| = |z|\cdot|w|$
		\begin{proof}
			$|z\cdot w|^2 = (z\cdot w)\conj{(z\cdot w)} = (z\cdot w)(\conj z\cdot \conj w) = (z\cdot \conj z)\cdot(w\cdot\conj w) = |z|^2\cdot |w|^2 = (|z|\cdot|w|)^2$. By uniqueness of the square root, $|z\cdot w| = |z|\cdot |w|$. 
		\end{proof}
		\item[(d)] $|\Re (z)| \leq |z|$ and $|\Re(z)|=|z|\Lr z\in\mb R$.
		\begin{proof}
			$|\Re(z)|^2 = \Re(z)\cdot \conj{(\Re(z))} = (\Re(z))^2 \leq (\Re(z))^2 + (\Im(z))^2 = |z|^2$. By monotonicity of square root, $|\Re(z)|\leq |z|$.
		\end{proof}
		\item[(e)] $|z+w|\leq |z|+|w|$ and equality holds $\Lr z = \alpha w$ or $w = \alpha z$ for some $\alpha\in\mb R$.
		\begin{proof}
			$|z+w|^2 = (z+w)\conj{(z+w)} = (z+w)(\conj z+\conj w) = z\cdot \conj z + w\cdot \conj z + z\cdot \conj w + w\cdot\conj w = |z|^2 + w\cdot\conj z + \conj{\conj z \cdot w} + |w|^2 = |z|^2 + 2\Re(w\cdot \conj z) + |w|^2 \leq |z|^2 + 2|\Re(w\cdot \conj z)| + |w|^2$ since $x\leq |x| \os{(d)}{\leq} |z|^2 + 2|z\cdot\conj w| + |w|^2 \os{(b)+(c)}{=} |z|^2 + 2|z|\cdot|w| + |w|^2 = (|z|+|w|)^2$ for all $x\in\mb R$. By monotonicity of roots, $|z+w| \leq |z| + |w|$.
		\end{proof}
	\end{enumerate}
\end{theorem}

\begin{theorem}
	[Schwartz inequality] Let $z_1,z_2,\dots,z_n,w_1,\dots,w_n\in\mb C$. Then $$ \left| \sum_{i=1}^n z_i\cdot \conj{w_i} \right|^2 \leq \left(\sum_{i=1}^n |z_i|^2 \right) \left( \sum_{i=1}^n |w_i|^2\right).$$
\end{theorem}
\begin{proof}
	Let $Z:= \sum_{i=1}^n |z_i|^2 \in\mb R$ and $W: = \sum_{i=1}^n |w_i|^2\in\mb R$ and $P:=\sum_{i=1}^n z_i\cdot \conj{w_i}\in\mb C$. \ul{Claim}: $zw \geq |p|^2$. Note that $W \geq 0$. Then $W = 0 \Lr |w_i| = 0 \Lr w_i = 0$. Then $P = \sum_{i=1}^n z_i\cdot 0 = 0$, we're done.
	
	\ul{Take $W>0$}: Then 
	\begin{align*}
		0\leq \sum_{i=1}^n |Wz_i - Pw_i|^2 &\os{?}{\geq} W(WZ-|P|^2) \\
		&= \sum_{i=1}^n(Wz_i-Pw_i)\conj{(Wz_i-Pw_i)} \\
		&= \sum_{i=1}^n(Wz_i-Pw_i)(W\conj{z_i}-\conj P\conj{w_i}) \\
		& = \sum_{i=1}^n \left[ W^2 z_i\cdot \conj{z_i} - W\conj P z_i \conj{w_i} - PW w_i \conj{z_i} + P\conj P w_i \cdot \conj{w_i} \right] \\
		&= \sum_{i=1}^n \left[ W^2|z_i|^2 - W\conj P z_i \conj{w_i} - PW w_i \conj{z_i} + |P|^2 \cdot |w_i|^2 \right] \\
		&= W^2 \sum_{i=1}^n |z_i|^2 - W\conj P\sum_{i=1}^n z_i \cdot \conj{w_i} - PW \sum_{i=1}^n w_i \conj{z_i} + |P|^2 \sum_{i=1}^n |w_i|^2 \\
		&= W^2 Z  - W\conj P P - P W \conj P + |P|^2 W = W^2Z - W|P|^2 \\
		&= W(WZ-|P|^2) \geq 0 \os{W>0}{\implies} WZ-|P|^2 \geq 0
	\end{align*}
\end{proof}

\ul{Test 1, Sep 19, 2017}: Problems to ignore \#7, \# 11, \#16, \#20, Expect:
\begin{itemize}
	\item Definitions (always give example)
	\item Statement of named theorems
	\item HW like problems, exercises,
	\item Proofs of theorems/propositions
\end{itemize}

\subsection{Euclidean spaces}
\begin{definition}
	[Vector space over a field] Let $(F,+_F,\cdot_F)$ be a field. A vector space $V$ over a field $F$ is a nonempty set $V$ with two operations -- vector addition + and scalar multiplication $\cdot$ with following:
	\begin{enumerate}
		\item $V$ satisfies (A1)-(A5) of field axioms with `+'.
		\item \begin{itemize}
			\item $a\vv\in V\ \forall a\in F,\ \forall \vv\in V$.
			\item $a(\vu + \vv) = a\vu + a\vv\ \forall a\in F,\ \forall \vu,\vv\in V$.
			\item $(a+_Fb)\vu = a\vu + b\vu\ \forall a,b\in F,\ \forall \vu \in V$.
			\item $(a\cdot_Fb)\vu = a(b\vu)\ \forall a,b\in F,\ \forall \vu\in V$.
			\item $1_F\vu = \vu\ \forall \vu\in V$.
		\end{itemize}
	\end{enumerate}
\end{definition}

Define $\mb R^k = \us{k\text{ times}}{\mb R\times \mb R\times \dots \times \mb R}$ for $k\geq 1$.

For $\vx = (x_1,\dots,x_k), \vy = (y_1,\dots,y_k)\in\mb R$, define addition + and scalar multiplication $\cdot$ as follows:
$$ \vx + \vy = (x_1+y_1,x_2+y_2,\dots,x_k+y_k)$$
$$ a\vx = (ax_1,ax_2,\dots,ax_k) \quad \text{for }a\in\mb R.$$
$$ \vzero = (0,0,\dots,0).$$

\begin{theorem}
	$(\mb R^k,+,\cdot)$ is a vector space over the field of $\mb R$.
\end{theorem}
\begin{proof}
	Skip.
\end{proof}
\begin{definition}
	For $\vx,\vy\in\mb R^k$, inner product or scalar product is defined as:
	$$ \vx\cdot\vy = x_1y_1+x_2y_2+\dots+x_ky_k = \sum_{i=1}^k x_iy_i\ul{\ul{\in\mb R}}$$
\end{definition}
\begin{definition}
	[Norm or modulus or absolute value]
	$$ |\vx| = \left( \sum_{i=1}^k x_i^2 \right)^\frac{1}{2}$$
	Other notation $||\vx|| = \left( \sum_{i=1}^k x_i^2 \right)^{1/2}$.
\end{definition}
\begin{theorem}
	Let $\vx,\vy,\vz\in\mb R^k$ and $a\in\mb R$. Then
	\begin{enumerate}
		\item[(a)] $|\vx|\geq 0$ and $|\vx| = 0 \Lr \vx = \vzero$.
		\begin{proof}
			$|\vx|^2 = \sum_{i=1}^k x_i^2 \geq 0$, since $x_i^2 \geq 0$ for $x_i\in\mb R$. $\implies |\vx| \geq 0$ using the monotonicity of root. Suppose that $|\vx| = 0$ and suppose by contradiction that $x_i\neq 0$ for some $i\in\{1,\dots,k\}$. $\implies x_i^2 >0$ and $\sum_{j=1}^k x_j^2 > 0$, a contradiction. If $\vx = \vzero$, then $x_i = 0$ for all $i = 1,\dots,k$. So $|\vx| = \left( \sum_{i=1}^k 0^2 \right)^{1/2} = 0$.
		\end{proof}
		\item[(b)] $|a\vx| = |a|\cdot|\vx|$
		\begin{proof}
			$|a\vx|^2 = \sum_{i=1}^k (a x_i)^2 = a^2 \sum_{i=1}^k x_i^2 = a^2(|\vx|)^2 \implies |a\vx| = \sqrt{a^2}\sqrt{|\vx|^2} = |a||\vx|$.
		\end{proof}
		\item[*(c)] $|\vx\cdot\vy| \leq |\vx|\cdot |\vy|$
		\begin{proof}
			WTS: $|\vx|^2|\vy|^2$. Recall, Schwartz inequality $$\left| \sum_{i=1}^n z_i \cdot \conj{w_i} \right|^2 \leq \sum_{i=1}^n |z_i|^2 \sum_{i=1}^n |w_i|^2.$$
			\begin{align*}
				|\vx\cdot \vy|^2 &= \left| \sum_{i=1}^k x_iy_i \right|^2
				\os{\text{Schwartz}}{\leq} \left(\sum_{i=1}^n |x_i|^2 \right) \left( \sum_{i=1}^n |y_i|^2 \right) \os{x_i,y_i\in\mb R}{=} \left(\sum_{i=1}^k x_i^2 \right) \left( \sum_{i=1}^k y_i^2 \right) = |\vx|^2 |\vy|^2.
			\end{align*}
			By monotonicity of root $|\vx\cdot \vy| \leq |\vx||\vy|$.
		\end{proof}
		\item[(d)] $|\vx+\vy| \leq |\vx|+|\vy|$
		\begin{proof}
			Want to show: $|\vx + \vy|^2 \leq (|\vx| + |\vy|)^2 = |\vx|^2 + 2|\vx|\cdot|\vy| + |\vy|^2$.
			\begin{align*}
				|\vx + \vy|^2 &= \sum_{i=1}^k (x_i+y_i)^2 = \sum_{i=1}^k (x_i^2 + 2x_i y_i + y_i)^2
				= \sum_{i=1}^k x_i^2 + 2 \sum_{i=1}^k x_i y_i + \sum_{i=1}^k y_i^2 \\
				&\leq |\vx|^2 + 2|\vx \cdot \vy| + |\vy|^2 \os{(c)}{\leq} |\vx|^2 + 2|\vx|\cdot |\vy| + |\vy|^2 = (|\vx| + |\vy|)^2
			\end{align*}
			since $\vx\cdot\vy\in\mb R$ and $a\leq |a|$ for all $a\in\mb R$. This implies $|\vx + \vy| \leq |\vx| + |\vy|$ by monotonicity of root.
		\end{proof}
		\item[(e)] $|\vx - \vz| \leq |\vx - \vy| + |\vy-\vz|$
		\begin{proof}
			$|\vx - \vz| = |\vx + \vzero - \vz| = |\vx - \vy + \vy - \vz| \os{(d)}{\leq} |\vx-\vy| + |\vy-\vz|$.
		\end{proof}
	\end{enumerate}
\end{theorem}
\begin{observe}
	Observe that $|\vx\cdot \vy| = |\vx|\cdot |\vy|$ if $x,y\in\mb R$ \ul{\ul{or}} if $x,y\in\mb C$. \ul{Not true in $\mb R^2$}:
	
	Compare the result of (c) in $\mb C$ versus in $\mb R^2$.
	Take $z = a + ib,w = c+id\in\mb C$. Then
	\begin{align*}
		|z\cdot w|^2 &= |(a+ib)\cdot(c+id)|^2 = |(ac-bd) + i(ad+bc)|^2 = (ac-bd)^2 + (ad+bc)^2 \\
		&= (ac)^2 + (bd)^2 + (ad)^2 + (bc)^2 = (a^2 + b^2)(c^2+d^2) = |z|^2 \cdot |w|^2 \implies |z\cdot w| = |z|\cdot |w|.
	\end{align*}
	Take $\vx = (a,b)$, $\vy = (c,d)$ in $\mb R^2$.
	\begin{align*}
		|\vx\cdot \vy|^2 &= (ac+bd)^2 = (ac)^2 + 2abcd + (bd)^2 \\
		&\cdots\ ?\ \cdots\\
		&\leq |(a^2+b^2) \cdot (c^2+d^2) = |\vx|\cdot |\vy|
	\end{align*}
\end{observe}

\end{document}
