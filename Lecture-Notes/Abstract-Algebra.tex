\documentclass[]{article}
\usepackage[latin1]{inputenc}
\usepackage{graphicx}
\usepackage[left=1.00in, right=1.00in, top=1.10in, bottom=1.00in]{geometry}

\usepackage{dirtytalk}
\usepackage[normalem]{ulem}
\usepackage{tikz-cd}
\usepackage{units}
\usepackage{algorithm}
\usepackage{algpseudocode}
\usepackage{alltt}
\usepackage{mathrsfs}
\usepackage{amssymb}
\usepackage{amsmath}
\DeclareMathOperator\cis{cis}

% (font shortcuts)
\usepackage{amsfonts}
\newcommand{\mb}[1]{\mathbb{#1}}
\newcommand{\mc}[1]{\mathcal{#1}}
\newcommand{\ms}[1]{\mathscr{#1}}
\newcommand{\mf}[1]{\frak{#1}}

% (arrow shortcuts)
\newcommand{\ra}{\rightarrow}
\newcommand{\lra}{\longrightarrow}
\newcommand{\la}{\leftarrow}
\newcommand{\lla}{\longleftarrow}
\newcommand{\Ra}{\Rightarrow}
\newcommand{\Lra}{\Longrightarrow}
\newcommand{\La}{\Leftarrow}
\newcommand{\Lla}{\Longleftarrow}
\newcommand{\lr}{\leftrightarrow}
\newcommand{\llr}{\longleftrightarrow}
\newcommand{\Lr}{\Leftrightarrow}
\newcommand{\Llr}{\Longleftrightarrow}

% (match parenthesis)
\newcommand{\mlr}[1]{\left|#1\right|}
\newcommand{\plr}[1]{\left(#1\right)}
\newcommand{\blr}[1]{\left[#1\right]}

% (exponent shortcuts)
\newcommand{\inv}{^{-1}}
\newcommand{\nrt}[2]{\sqrt[\leftroot{-2}\uproot{2}#1]{#2}}

% (annotation shortcuts)
\newcommand{\conj}[1]{\overline{#1}}
\newcommand{\ol}[1]{\overline{#1}}
\newcommand{\ul}[1]{\underline{#1}}
\newcommand{\os}[2]{\overset{#1}{#2}}
\newcommand{\us}[2]{\underset{#1}{#2}}
\newcommand{\ob}[2]{\overbrace{#2}^{#1}}
\newcommand{\ub}[2]{\underbrace{#2}_{#1}}
\newcommand{\bs}{\backslash}
\newcommand{\ds}{\displaystyle}

% (set builder)
\newcommand{\set}[1]{\left\{ #1 \right\}}
\newcommand{\setc}[2]{\left\{ #1 : #2 \right\}}
\newcommand{\setm}[2]{\left\{ #1 \, \middle| \, #2 \right\}}

% (group generator)
\newcommand{\gen}[1]{\langle #1 \rangle}

% (functions)
\newcommand{\im}[1]{\text{im}(#1)}
\newcommand{\range}[1]{\text{range}(#1)}
\newcommand{\domain}[1]{\text{domain}(#1)}
\newcommand{\dist}[1]{(#1)}
\newcommand{\sgn}{\text{sgn}}

% (Linear Algebra)
\newcommand{\mat}[1]{\begin{bmatrix}#1\end{bmatrix}}
\newcommand{\pmat}[1]{\begin{pmatrix}#1\end{pmatrix}}
%\newcommand{\dim}[1]{\text{dim}(#1)}
\newcommand{\rnk}[1]{\text{rank}(#1)}
\newcommand{\nul}[1]{\text{nul}(#1)}
\newcommand{\spn}[1]{\text{span}\,#1}
\newcommand{\col}[1]{\text{col}(#1)}
%\newcommand{\ker}[1]{\text{ker}(#1)}
\newcommand{\row}[1]{\text{row}(#1)}
\newcommand{\area}[1]{\text{area}(#1)}
\newcommand{\nullity}[1]{\text{nullity}(#1)}
\newcommand{\proj}[2]{\text{proj}_{#1}\left(#2\right)}
\newcommand{\diam}[1]{\text{diam}\,#1}

% (Vectors common)
\newcommand{\myvec}[1]{\vec{#1}}
\newcommand{\va}{\myvec{a}}
\newcommand{\vb}{\myvec{b}}
\newcommand{\vc}{\myvec{c}}
\newcommand{\vd}{\myvec{d}}
\newcommand{\ve}{\myvec{e}}
\newcommand{\vf}{\myvec{f}}
\newcommand{\vg}{\myvec{g}}
\newcommand{\vh}{\myvec{h}}
\newcommand{\vi}{\myvec{i}}
\newcommand{\vj}{\myvec{j}}
\newcommand{\vk}{\myvec{k}}
\newcommand{\vl}{\myvec{l}}
\newcommand{\vm}{\myvec{m}}
\newcommand{\vn}{\myvec{n}}
\newcommand{\vo}{\myvec{o}}
\newcommand{\vp}{\myvec{p}}
\newcommand{\vq}{\myvec{q}}
\newcommand{\vr}{\myvec{r}}
\newcommand{\vs}{\myvec{s}}
\newcommand{\vt}{\myvec{t}}
\newcommand{\vu}{\myvec{u}}
\newcommand{\vv}{\myvec{v}}
\newcommand{\vw}{\myvec{w}}
\newcommand{\vx}{\myvec{x}}
\newcommand{\vy}{\myvec{y}}
\newcommand{\vz}{\myvec{z}}
\newcommand{\vzero}{\myvec{0}}


\author{Book: Thomas Judson 2017, Presenter: Igor Erovenko, Notes by Michael Reed}
\title{Abstract Algebra}
%date{}

\begin{document}
\maketitle

%\begin{abstract}
%\end{abstract}

\section{Proofs}

Let $p$ and $q$ be statements. \say{If $p$ then $q$.} $p$ is the hypothesis and $q$ is the conclusion. \ul{Implication} is $p\implies q$, read \say{$p$ implies $q$.}

\begin{tabular}{|c|c|c|}
	\hline 
	$p$ & $q$ & $p\implies q$ \\ 
	\hline 
	T & T & T \\ 
	\hline 
	T & F & F \\ 
	\hline 
	F & T & T \\ 
	\hline 
	F & F & T \\ 
	\hline 
\end{tabular}

\subsection*{Methods of proof}

\begin{enumerate}
	\item Direct proof. Assume $p$ is true, show $q$ is true. $p\implies \dots \implies \dots \implies\dots \implies q$.
	\begin{example}
		Show that if $n$ is an odd integer, then $n^2$ is odd. $p = $ \say{$n$ is odd} and $q = $ \say{$n^2$ is odd.}
		\begin{proof}
			Suppose $n$ is an odd integer. Then $n = 2k+1$ for some integer $k$. In this case, $n^2 = (2k+1)^2 = 4k^2 + 4k + 1 = 2(2k^2 + 2k) + 1$. Notice that $2k^2+2k$ is an integer. So, $n^2$ is odd.
		\end{proof}
	\end{example}
	\item Reduction to cases. If $p$ is true, then one of $p_1,p_2,\dots,p_n$ must be true. Then show each $p_i \implies q$.
	\begin{example}
		Show that if $n$ is an integer, then $n^2-n$ is even. $p = $ \say{$n$ is an integer} and $q = $ \say{$n^2-n$ is even.}
		\begin{proof}
			Suppose $n$ is an integer. Then $n$ is even or odd.
			\ul{Case 1}: $n$ is even. In this case, $n=2k$ for some integer $k$ and $n^2-n = (2k)^2 - 2k = 4k^2 - 2k = 2(2k^2-k)$ is even. \ul{Case 2}: $n$ is odd. In this case $n = 2k+1$ for some integer $k$ and $n^2-n = (2k+1)^2-(2k+1) = 4k^2 + 4k + 1 - 2k -1 = 4k^2 +2k = 2(2k+k)$ is even.
		\end{proof}
	\end{example}
	\item Proof by contradiction. If $p$ is true, then either $q$ is true or it is not true. Show that $\neg q$ is impossible. Assume: $p$ is true and $q$ is false (not $q$ is true). Show: this leads to a contradiction.
	\begin{example}
		Show that $\sqrt{2}$ is irrational. If $r$ is rational, then $r^2 \neq 2$.
		\begin{proof}
			Assume $r$ is rational and $r^2 = 2$. Then $r = \frac{m}{n}$ for some integers $m$ and $n$, in lowest terms. Then $(\frac{m}{n})^2 = 2$. So, $m^2 = 2n^2$. It follows that $m^2$ is even, and hence $m$ is even. Then $m = 2k$ and we get $(2k)^2 = 4k^2 = 2n^2$ or $2k^2 = n^2$. Therefore $n^2$ is even, and consequently $n$ is even. Both $m$ and $n$ are even, which contradicts our assumption that $\frac{m}{n}$ is in lowest terms. Therefore $r^2 \neq 2$ and $\sqrt{2}$ is irrational.
		\end{proof}
	\end{example}
\end{enumerate}

To disprove a general statement: find one counterexample.

\begin{example}
	Every integer is odd. False, $4$ is not odd.
\end{example}

Equivalences: $p$ if and only if $q$. $p\Longleftrightarrow q$, needs proofs for both $p\implies q$ and $q\implies p$.

\section{Sets}

A \ul{set} is a well-defined collection of objects. Requires a criterion for determining set membership. $a\in A$ means $a$ is a \ul{member} or an \ul{element} of a set $A$.

$$X = \{ x_1,x_2,\dots,x_n\}, \, X = \{ x: x \text{ satisfies } P\} = \{x \mid x \text{ satisfies } P\}.$$

$\mb N$ natural numbers $\{1,2,3,4,\dots\}$.
$\mb Z$ integers $\{\dots,-3,-2,-1,0,1,2,3,\dots\}$.
$\mb Q$ rational numbers $\{\frac{m}{n} \mid m,n\in \mb Z, n\neq 0\}/\sim$.
$\mb R$ real numbers.
$\mb C$ complex numbers $\{a+bi \mid a,b\in\mb R\}$.
$\mb H$ Hamiltonian quaternions $\{a+bi+cj+dk \mid a,b,c,d\in\mb R\}$.

\begin{definition}
	[Subset] $A\subset B$ or $A\subseteq B$, $a\in A \implies a\in B$.
\end{definition}
\begin{definition}
	[Equality of sets]
	$A = B \Longleftrightarrow A \subseteq B$ and $B\subseteq A$.
\end{definition}

\begin{definition}
	[Proper subset] $A\subsetneq B \Longleftrightarrow A\subseteq B$ and $A\neq B$.
\end{definition}
\begin{definition}
	[Empty set] $\emptyset \subseteq A$ for any set $A$.
\end{definition}

\begin{definition}
	[Union] $A\cup B = \{ x\mid x\in A \text{ or } x\in B\}$.
\end{definition}
\begin{definition}
	[Intersection] $A\cap B = \{x \mid A \text{ and } x\in B\}$.
\end{definition}
\begin{definition}
	[Difference] $A-B = A\backslash B = \{x \mid x\in A \text{ and } x\notin B\}$.
\end{definition}

\begin{proposition}
	Let $A,B,$ and $C$ be sets.
	\begin{enumerate}
		\item $A\cup A = A$, $A\cap A = A$, $A\backslash A = \emptyset $.
		\item $A\cup \emptyset = A$, $A\cap \emptyset = \emptyset$.
		\item $A\cup (B\cup C) = (A\cup B)\cup C$.
		$A\cap (B\cap C) = (A\cap B) \cap C$.
		\item $A\cup B = B\cup A$. $A\cap B = B\cap A$.
		\item $A\cup (B\cap C) = (A\cup B) \cap (A\cup C)$.
		\begin{proof}
			Let $x\in A\cup (B\cap C)$. Then $x\in A$ or $x\in B\cap C$. \ul{Case 1}: $x\in A$. In this case, $x\in A\cup B$ and $x\in A\cup C$, so $x\in (A\cup B)\cap (A\cup C)$. \ul{Case 2}: $x\in B\cap C$. In this case, $x \in B$ and $x\in C$. So, $x\in A\cup B$ and $x\in A\cup C$. Hence $x \in (A\cup B)\cap (A\cup C)$. Therefore $A\cup (B\cap C) \subseteq (A\cup B)\cap (A\cup C)$.
		\end{proof}
		\item $A\cap (B\cup C) = (A\cap B) \cup (A\cap C)$.
	\end{enumerate}
\end{proposition}

\subsection*{Cartesian products and mappings}

Cartesian product of $A$ and $B$ is $A\times B = \{(a,b)\mid a\in A \text{ and } b\in B\}$.
In general, $A_1\times A_2\times\dots\times A_n = \{(a_1,a_2,\dots,a_n)\mid a_i\in A_i \text{ for } 1\leq i \leq n\}$.
\ul{Notation}: $A\times A\times \dots\times A = A^n$.

\begin{example}
	$\mb R^2, \mb R^3$.
\end{example}

A \ul{relation} from $A$ to $B$ is a subset of $A\times B$.
A \ul{mapping} (or a \ul{function}) $f$ from $A$ to $B$ is a relation from $A$ to $B$ such that for every $a\in A$ there exists a unique $b\in B$ for which $(a,b)\in f$. \ul{Notation}: $(a,b) \in f \ra f(a) = b$ or $f:a\mapsto b$ and on the level of sets $f:A\ra B$. $A$ is the domain of $f$, $B$ is the codomain, $f(A) = \{f(a)\mid a\in A\}\subseteq B$ is the range of $f$.

\begin{definition}
	A mapping $f:A\ra B$ is
	\begin{enumerate}
		\item \ul{injective} (or one-to-one) if $f(a_1) = f(a_2) \implies a_1 = a_2$.
		\item \ul{surjective} (or onto) if $f(A) = B$, or for every $b\in B$ there exists $a\in A$ such that $f(a) = b$.
		\item \ul{bijective} (or one-to-one correspondence) if it's injective and surjective.
	\end{enumerate}
\end{definition}
\begin{example}
	$f:\mb R\ra \mb R$, $f(x) = e^x$. $e^{x_1} = e^{x_2} \overset{?}{\implies} x_1 = x_2$ then $\ln(e^{x_1}) = \ln(e^{x_2})$, so injective. $f(\mb R) = \mb R^+ = (0,\infty) \neq \mb R$, not surjective and not bijective.
	$f:\mb R \ra (0,\infty)$ is bijective.
\end{example}
\begin{example}
	$f:\mb Z\ra\mb Z$, $f(n) = n^2$. $f(-1) = 1 = f(1)$, not injective. $2\notin f(\mb Z)$, not surjective and not bijective.
\end{example}
\begin{example}
	$f:\mb{R}\ra \mb R$, $f(x) = 2x+1$. $2x_1 + 1 = 2x_2 + 1 \implies x_1 = x_2$, injective. Given any $y\in \mb R$ find $x\in \mb R$ such that $2x+1 = y$: $x = (y-1)/2$, surjective and bijective.
\end{example}

A \ul{composition} of mappings $f:A\ra B$ and $g:B\ra C$ is a mapping $g\circ f: A\ra C$ defined by $(g\circ f)(a) = g(f(a))$. $A \overset{f}{\lra} B \overset{g}{\lra} C$ or $a \rightsquigarrow f(a) \rightsquigarrow g(f(a))$.
\begin{theorem}
	Let $f:A\ra B$, $g:B\ra C$, and $h:C\ra D$ be mappings.
	\begin{enumerate}
		\item The composition of mappings is associative: $(h\circ g) \circ f = h\circ(g\circ f)$.
		\begin{proof}
			Let $a\in A$. Then $((h\circ g)\circ f)(a) = (h\circ g)(f(a)) = h(g(f(a))) \equiv h((g\circ f)(a)) = (h\circ(g\circ f))(a)$.
		\end{proof}
		\item If $f$ and $g$ are injective, then so is $g\circ f$.
		\begin{proof}
			$(g\circ f)(a_1) = (g\circ f)(a_2) \implies g(f(a_1)) = g(f(a_2)) \implies f(a_1) = f(a_2) \implies a_1 = a_2$.
		\end{proof}
		\item If $f$ and $g$ are surjective, then so is $g\circ f$.
		\begin{proof}
			$A\os{f}{\lra} B \os{g}{\lra} C$ or $A\os{g\circ f}{\lra} C$. $g(b) = c$, $f(a) = b$.
		\end{proof}
		\item If $f$ and $g$ are bijective, then so is $g\circ f$.
	\end{enumerate}
\end{theorem}

\begin{definition}
	The \ul{identity} mapping on a set $A$ is $\text{id}_A: A\ra A$, $\text{id}_A(a) = a$.
	An \ul{inverse} mapping of a mapping $f:A\ra B$ is a mapping $g:B\ra A$ such that $g\circ f = \text{id}_A \ra g(f(a)) = a$, $f\circ g = \text{id}_B \ra f(g(b)) = b$.
	A mapping is \ul{invertible} if it has an inverse mapping. \ul{Notation}: $f\inv$ inverse of $f$.k
\end{definition}

\begin{theorem}
	A mapping is invertible $\Leftrightarrow$ it is bijective.
\end{theorem}

\subsection*{Equivalence relations}

\begin{definition}
	An \ul{equivalence relation} on a set $X$ is a relation $R\subset X\times X$ which is
	\begin{enumerate}
		\item \ul{Reflexive}: $(x,x)\in R$ for all $x\in X$
		\item \ul{Symmetric}: $(x,y)\in R \implies (y,x)\in R$
		\item \ul{Transitive}: $(x,y)\in R,(y,z)\in R\implies (x,z)\in R$
	\end{enumerate}
	\ul{Notation}: $x\sim y$ for $(x,y)\in R$.
\end{definition}
\begin{example}
	$X=\mb R$, $a\sim b$ if $a<b$. Reflexive: $0\nless 0$. Symmetric: $0<1$ but $1\nless 0$. Transitive: $a<b$ and $b<c\implies a<c$. However, for $a\neq b$ either $a<b$ or $b<a$ (antisymmetric).
\end{example}
\begin{example}
	$X = \mb R$, $a\sim b$ if $ab\geq 0$. \ul{Reflexive}: $a\sim a \Leftrightarrow a^2 \geq0$. \ul{Symmetric}: $a\sim b \implies b\sim a$ or $ab\geq 0\implies ba\geq 0$. \ul{Transitive}: $a\sim b,b\sim c\implies a\sim c$ or $ab\geq 0,bc\geq 0 \implies ac\geq 0$. Counterexample: $a=-1,b=0,c=1$.
\end{example}
\begin{example}
	$X$ any set, $x\sim y$ if $x=y$. \ul{Reflexive}: $x\sim x \Leftrightarrow x=x$. \ul{Symmetric}: $x\sim y\implies y\sim x$ or $x=y\implies y=x$. \ul{Transitive}: $x\sim y,y\sim z\implies x\sim z$ or $x=y,y=z\implies x=z$.
\end{example}
\begin{example}
	$X = $ all differentiable functions $\mb R\ra\mb R$. $f\sim g$ if $f' = g'$. \ul{Reflexive}: $f\sim f \Leftrightarrow f' = f'$. \ul{Symmetric}: $f\sim g \implies g\sim f$ or $f' = g' \implies g' = f'$. \ul{Transitive}: $f\sim g,g\sim h\implies f\sim h$ or $f' = g',g'=h'\implies f'=h'$.
\end{example}
\begin{example}
	$X = \mb R^2$, $(x_1,y_1)\sim (x_2,y_2)$ if $x_1^2 + y_1^2 = x_2^2 + y_2^2$.
\end{example}
\begin{example}
	$X = M_n(\mb R)$ all $n\times n$ matrices. $A\sim B$ if $\exists$ a nonsingular P such that $B = PAP\inv$. \ul{Reflexive}: $A = IAI\inv$ let $P = I$. \ul{Symmetric}: $A\sim B \implies B\sim A \lra A = *B*\inv$, then $B = PAP\inv \implies P\inv B = AP\inv \implies P\inv BP = A$, so let $* = P\inv$. \ul{Transitive}: $A\sim B,B\sim C\implies A\sim C$. $B = PAP\inv, C = SBS\inv \implies C = *A*\inv$. $C = SBS\inv = S(PAP\inv)S\inv = (SP)A(P\inv S\inv) = (SP)A(SP)\inv$, so $*= SP$.
\end{example}

\newpage

\begin{definition}
	Let $\sim$ be an equivalence relation on a set $X$, and let $x\in X$. Then the set $$[x] = \{y\in x \mid y\sim x\}$$ is the \ul{equivalence class} of $x$.
\end{definition}
\begin{example}
	$X$ is any set. $x\sim y$ if $x=y$. Then $[x] = \{y\in X\mid y=x\} = \{x\}$.
\end{example}
\begin{example}
	$X$ is the set of all differentiable functions from $\mb R\ra\mb R$. $f\sim g$ if $f' = g'$. $[f] = \{g\in X\mid g' = f'\} = \{f + c\mid c\in\mb R\}$.
\end{example}
\begin{example}
	$X = \mb R^2$. $(x_1,y_1)\sim (x_2,y_2)$ if $x_1^2 + y_1^2 = x_2^2 + y_2^2$. The equivalence class is a circle.
\end{example}

\begin{definition}
	A \ul{partition} of a set $X$ is a collection of disjoint nonempty subsets of $X$ which cover $X$.
	$$\mc P = \{X_\alpha\}_{\alpha\in \Lambda}, \quad X\alpha\subseteq X\, \forall \alpha,$$
	$X_\alpha \cap X_\beta = \emptyset \text{ if } \alpha \neq \beta$, disjoint, and $X = \bigcup_{\alpha\in\Lambda} X_\alpha$ cover $X$.
\end{definition}
\begin{theorem}
	Let $\sim $ be an equivalence relation on a set $X$. Then the equivalence classes of $\sim$ partition $X$. Conversely, given a partition $\mc P = \{X_\alpha\}_{\alpha\in\Lambda}$ of $X$, there exists an equivalence relation $\sim$ on $X$ such that $\{X_\alpha\}$ are the equivalence classes of $\sim$.
\end{theorem}
\begin{proof}
	Let $\sim$ be an equivalence relation on $X$. Then for every $x\in X$, $x\sim x$, so $x\in [x]$. Hence, $[x]\neq \emptyset$ for all $x\in X$. Also $\bigcup_{x\in X}[x] = X$. By definition $\bigcup_{x\in X}[x] \subseteq X$ and  every $x$ is in one of equivalence classes so $\bigcup_{x\in X}[x] \supseteq X$. It's left to prove that if $x,y\in X$ then $[x]\cap [y] = \emptyset$ or $[x]=[y]$.
	\ul{Case 1}: $[x]\cap[y] = \emptyset$ done. \ul{Case 2}: $[x]\cap[y]\neq \emptyset$ (WTS: $[x]=[y]$). There is $z\in[x]\cap[y]$. Then $z\in[x]$ and $z\in[y]$, so $z\sim x$ and $z\sim y$. By symmetric property $x\sim z$ and by transitive property $x\sim y$. Take any $a\in [x]$. Then $a\sim x$ and also $x\sim y$, so $a\sim y$ and thus $a\in [y]$. So $[x]\subseteq [y]$. If $x\sim y$ then $y\sim x \implies [y]\subseteq [x]$. Thus $[x]=[y]$.
	
	Conversely, let $\mc P = \{X_\alpha\}_{\alpha\in\Lambda}$ be a partition of $X$. Define a relation $\sim$ on $X$ by $x\sim y$ if $x$ and $y$ belong to the same $X_\alpha$ for some $\alpha\in\Lambda$. Show:
	\begin{enumerate}
		\item $\sim$ is an equivalence relation on $X$,
		\item the equivalence classes of $\sim$ are $X_\alpha$'s
	\end{enumerate}
\end{proof}

\section*{Mathematical induction}

$$P(1),P(2),P(3),\dots,P(n),P(n+1),\dots$$

Mathematical induction: prove that $P(n)$ is true for all $n\geq 1$.

\begin{definition}
	[Well-Ordering Principle] Every nonempty subset of $\mb N$ contains a smallest element.
\end{definition}
\begin{theorem}
	[First principle of mathematical induction] Suppose $P(n)$ is a statement involving a positive integer $n$, and
	\begin{enumerate}
		\item \textit{Base}: $P(n_0)$ is true for some $n_0\in\mb N$
		\item \textit{Induction hypothesis}: $P(k)$ implies $P(k+1)$ for all $k\geq n_0$.
	\end{enumerate}
	$P(n)$ is true for all $n\geq n_0$.
\end{theorem}
\begin{remark}
	This is nothing but a \say{domino} effect. $P(k) = $ \say{domino $k$ falls}.
\end{remark}
\begin{proof}
	Assume on the contrary that not all $P(n)$ are true for $n\geq n_0$. Then $S = \{n\geq n_0\mid P(n) \text{ is false}\}\subseteq \mb N$ is a nonempty subset of $\mb N$. By the Well-Ordering Principle, the set $S$ contains a least element $m$. Then $P(m)$ is false and $m\neq n_0$ by assumption (1). This means that $m>n_0 \implies m-1\geq n_0$. So $P(m-1)$ is true since $m-1\notin S$. By assumption (2), if $P(m-1)$ is true, then $P(m)$ is true, which is a contradiction.
\end{proof}
\begin{example}
	$1+2+\dots+n = \frac{n(n+1)}{2}$ for all $n\geq 1$. $P(n) = $ \say{$1+2+\dots+n = \frac{n(n+1)}{2}$.}
	\ul{Base}: $P(1)$ is true since $1 = \frac{1\cdot(1+1)}{2}$.
	\ul{Inductive step}: Assume $P(k)$ is true for some fixed arbitrary $k \geq 1$. Want to show: $P(k+1)$ is true. Have: $1+2+\cdots+k = \frac{k(k+1)}{2}$. Want: $1+2+\dots+k + (k+1) = \frac{(k+1)(k+2)}{2}$.
	\begin{align*}
		1+2+\dots + k + (k+1) &= \frac{k(k+1)}{2} + (k+1) = \frac{k(k+1)}{2} + \frac{2(k+1)}{2} \\
		&= \frac{k(k+1) + 2(k+1)}{2} = \frac{(k+1)(k+2)}{2}
	\end{align*}
	Therefore, $P(n)$ is true for all $n\geq 1$.
\end{example}
\begin{example}
	All cars are the same color. $P(n) = $ \say{in any collection of $n$ cars, all cars are the same color.} Will prove by induction that $P(n)$ is true for all $n\geq 1$. \ul{Base}: $P(1)$ is true. \ul{Inductive step}: $P(k)$ is true for some $k\geq 1$. WTS: $P(k+1)$ is true. Cars $c_1,\dots,c_k$ are same color. However $P(1)\nrightarrow P(2)$.
\end{example}

\begin{example}
	The UNCG poker club plays with \$5 and \$8 chips. What is the largest bet that cannot be made using these num? Largest bet that cannot be made is \$27.
	
	We need to show
	\begin{enumerate}
		\item A bet of \$27 cannot be made.
		\item Any bet $\geq \$28$ can be made.
	\end{enumerate}
	A bet of $\$n$ can be made if $n=5a+8b$, where $a$ and $b$ are nonnegative integers. Take this modulo 5, then $2\equiv 3b\pmod5$. Let $b=0$, then $2\not\equiv 0\pmod5$. Let $b=1$, then $2\not\equiv 3\pmod 5$. Let $b=2$, then $2\not\equiv 6\pmod5$. Let $b = 3$, then $2\not\equiv 9\pmod5$. Let $b=4$, then $2\equiv 12\pmod 12$. So $b\geq 4$ and $27 = 5a + 8b\geq 5a+32$ implies $5a\leq -5\implies a\leq -1$. $P(n) = $ \say{A bet of $\$n$ can be made} = \say{$n = 5a+8b$ for some nonnegative integers $a$ and $b$.} WTS: $P(n)$ is true for all $n\geq 28$.
	\ul{Base}: $P(28)$ is true since $28 = 5\cdot 4 + 8\cdot 1$.
	\ul{Inductive step}: Assume $P(k)$ is true for some arbitrary fixed $k\geq 28$. WTS: $P(k+1)$ is true. Assume: $k=5a+8b$. Want: $k+1 = 5\tilde a + 8\tilde b$. If $k = 5a+8b$ then $k+1 = 5a+8b+1$. Then $1 = 5\cdot(-3) + 8\cdot(2) = 5\cdot(5) + 8\cdot(-3)$. Then $5a+8b +1 = 5\cdot(a-3) + 8\cdot(b+2) = 5\cdot(a+5) + 8\cdot(b-3)$.
	\ul{Claim}: $a \geq 3$ or $b\geq 3$. Assume on the contrary $a<3$ and $b<3$. Then $k = 5a+8b \leq 5\cdot 2 + 8\cdot 2 = 26$. This is a contradiction.
	\ul{Case 1}: Let $a\geq 3$, then $k+1 = 5\cdot(a-3) + 8\cdot(b+2)$ is a valid bet.
	\ul{Case 2}: Let $b\geq 3$, then $k+1 = 5\cdot (a+5) + 8\cdot(b-3)$ is a valid bet.
\end{example}

\begin{theorem}
	[Second form of mathematical induction] Suppose $P(n)$ is a statement involving a positive integer $n$, and
	\begin{enumerate}
		\item $P(n)$ is true for some $n_0\in\mb N$.
		\item $P(n_0),P(n_0+1),\dots,P(k)\implies P(k+1)$ for all $k\geq n_0$.
	\end{enumerate}
	Then $P(n)$ is true for all $n\geq n_0$.
\end{theorem}
\begin{proof}
	Let $Q(n) = $ \say{$P(n_0),P(n_0+1),\dots,P(n)$ are true.} Apply the first principle of mathematical induction to the statements $Q(n)$. Let $Q(n_0) = $ \say{$P(n_0)$ is true}, then $Q(k) \implies Q(k+1)$.
\end{proof}
\begin{example}
	Every natural number $\geq 2$ is a product of primes.
	
	\ul{Base}: Let $n=2$, which is a prime.
	\ul{Inductive step}: Assume that all $2,3,\dots,k$ are products of primes. Look at $k+1$. \ul{Case 1}: $k+1$ is prime, then done. \ul{Case 2}: $k+1$ is composite, then $k+1 = uv$ with $u,v\geq 2$. Then $2\leq u,v\leq k$. Let $u = p_1p_2\cdots p_s$ and $v = q_1q_2\cdots q_r$, where $p_i,q_j$ are primes. Then $k+1 = p_1p_2\cdots p_sq_1q_2\cdots q_r$. m
\end{example}


\end{document}
