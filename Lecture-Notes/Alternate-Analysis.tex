\documentclass[]{article}
\usepackage[latin1]{inputenc}
\usepackage{graphicx}
\usepackage[left=1.00in, right=1.00in, top=1.10in, bottom=1.00in]{geometry}

\usepackage{dirtytalk}
\usepackage[normalem]{ulem}
\usepackage{tikz-cd}
\usepackage{units}
\usepackage{algorithm}
\usepackage{algpseudocode}
\usepackage{alltt}
\usepackage{mathrsfs}
\usepackage{amssymb}
\usepackage{amsmath}
\DeclareMathOperator\cis{cis}

% (font shortcuts)
\usepackage{amsfonts}
\newcommand{\mb}[1]{\mathbb{#1}}
\newcommand{\mc}[1]{\mathcal{#1}}
\newcommand{\ms}[1]{\mathscr{#1}}
\newcommand{\mf}[1]{\frak{#1}}

% (arrow shortcuts)
\newcommand{\ra}{\rightarrow}
\newcommand{\lra}{\longrightarrow}
\newcommand{\la}{\leftarrow}
\newcommand{\lla}{\longleftarrow}
\newcommand{\Ra}{\Rightarrow}
\newcommand{\Lra}{\Longrightarrow}
\newcommand{\La}{\Leftarrow}
\newcommand{\Lla}{\Longleftarrow}
\newcommand{\lr}{\leftrightarrow}
\newcommand{\llr}{\longleftrightarrow}
\newcommand{\Lr}{\Leftrightarrow}
\newcommand{\Llr}{\Longleftrightarrow}

% (match parenthesis)
\newcommand{\mlr}[1]{\left|#1\right|}
\newcommand{\plr}[1]{\left(#1\right)}
\newcommand{\blr}[1]{\left[#1\right]}

% (exponent shortcuts)
\newcommand{\inv}{^{-1}}
\newcommand{\nrt}[2]{\sqrt[\leftroot{-2}\uproot{2}#1]{#2}}

% (annotation shortcuts)
\newcommand{\conj}[1]{\overline{#1}}
\newcommand{\ol}[1]{\overline{#1}}
\newcommand{\ul}[1]{\underline{#1}}
\newcommand{\os}[2]{\overset{#1}{#2}}
\newcommand{\us}[2]{\underset{#1}{#2}}
\newcommand{\ob}[2]{\overbrace{#2}^{#1}}
\newcommand{\ub}[2]{\underbrace{#2}_{#1}}
\newcommand{\bs}{\backslash}
\newcommand{\ds}{\displaystyle}

% (set builder)
\newcommand{\set}[1]{\left\{ #1 \right\}}
\newcommand{\setc}[2]{\left\{ #1 : #2 \right\}}
\newcommand{\setm}[2]{\left\{ #1 \, \middle| \, #2 \right\}}

% (group generator)
\newcommand{\gen}[1]{\langle #1 \rangle}

% (functions)
\newcommand{\im}[1]{\text{im}(#1)}
\newcommand{\range}[1]{\text{range}(#1)}
\newcommand{\domain}[1]{\text{domain}(#1)}
\newcommand{\dist}[1]{(#1)}
\newcommand{\sgn}{\text{sgn}}

% (Linear Algebra)
\newcommand{\mat}[1]{\begin{bmatrix}#1\end{bmatrix}}
\newcommand{\pmat}[1]{\begin{pmatrix}#1\end{pmatrix}}
%\newcommand{\dim}[1]{\text{dim}(#1)}
\newcommand{\rnk}[1]{\text{rank}(#1)}
\newcommand{\nul}[1]{\text{nul}(#1)}
\newcommand{\spn}[1]{\text{span}\,#1}
\newcommand{\col}[1]{\text{col}(#1)}
%\newcommand{\ker}[1]{\text{ker}(#1)}
\newcommand{\row}[1]{\text{row}(#1)}
\newcommand{\area}[1]{\text{area}(#1)}
\newcommand{\nullity}[1]{\text{nullity}(#1)}
\newcommand{\proj}[2]{\text{proj}_{#1}\left(#2\right)}
\newcommand{\diam}[1]{\text{diam}\,#1}

% (Vectors common)
\newcommand{\myvec}[1]{\vec{#1}}
\newcommand{\va}{\myvec{a}}
\newcommand{\vb}{\myvec{b}}
\newcommand{\vc}{\myvec{c}}
\newcommand{\vd}{\myvec{d}}
\newcommand{\ve}{\myvec{e}}
\newcommand{\vf}{\myvec{f}}
\newcommand{\vg}{\myvec{g}}
\newcommand{\vh}{\myvec{h}}
\newcommand{\vi}{\myvec{i}}
\newcommand{\vj}{\myvec{j}}
\newcommand{\vk}{\myvec{k}}
\newcommand{\vl}{\myvec{l}}
\newcommand{\vm}{\myvec{m}}
\newcommand{\vn}{\myvec{n}}
\newcommand{\vo}{\myvec{o}}
\newcommand{\vp}{\myvec{p}}
\newcommand{\vq}{\myvec{q}}
\newcommand{\vr}{\myvec{r}}
\newcommand{\vs}{\myvec{s}}
\newcommand{\vt}{\myvec{t}}
\newcommand{\vu}{\myvec{u}}
\newcommand{\vv}{\myvec{v}}
\newcommand{\vw}{\myvec{w}}
\newcommand{\vx}{\myvec{x}}
\newcommand{\vy}{\myvec{y}}
\newcommand{\vz}{\myvec{z}}
\newcommand{\vzero}{\myvec{0}}

% Theorems and Propositions
\usepackage{amsthm}
\newtheorem{theorem}{Theorem}
\newtheorem{proposition}{Proposition}

\theoremstyle{definition}
\newtheorem{definition}{Definition}

\theoremstyle{remark}
\newtheorem*{remark}{Remark}
\newtheorem{example}{Example}
\newtheorem*{recall}{Recall}
\newtheorem*{note}{Note}
\newtheorem*{observe}{Observe}
\newtheorem*{question}{\underline{Question}}
\newtheorem*{fact}{Fact}
\newtheorem{corollary}{Corollary}
\newtheorem*{lemma}{Lemma}
\newtheorem{xca}{Exercise}

%\usepackage[active,tightpage]{preview}
\setlength\PreviewBorder{7.77pt}
\usepackage{varwidth}
\AtBeginDocument{\begin{preview}\begin{varwidth}{\linewidth}}
\AtEndDocument{\end{varwidth}\end{preview}}


\author{Presenter: Richard Fabiano, Notes by Michael Reed}
\title{Mathematical Analysis (Alternate)}
%date{}

\begin{document}
\maketitle

%\begin{abstract}
%\end{abstract}

We assume basic definitions of sets:
$A\subset B$ means if $x\in A$ then $x\in B$.
Text does not use $\subseteq$.
$A = B$ means $A\subset B$ and $B\subset A$.
Assume familiarity with rational numbers, denoted by $\mb Q$.
Rational numbers are not adequate for calculus and analysis.

\begin{example}
	There is no solution to $p^2-2=0$ for $p\in\mb Q$.
\end{example}
\begin{proof}
	BWOC, suppose $p^2=2$ for some $p\in\mb Q$. Then $p = \frac mn$, where $m,n$ are integers, $n\neq0$, $m$ and $n$ are not both even.
	Then $\plr{\frac mn}^2=2$, so $m^2$ is even. Thus $m$ is even. (If $r$ is prime and $r\mid a\cdot b$, where $a$ and $b$ are integers, then $r$ divides either $a$ or $b$).
	Thus $m = 2k$ for some integer $k$. So $(2k)^2=2n^2$, so $2k^2=n^2$.
	Thus $n^2$ is even, which implies $n$ is even.
	So $m$ and $n$ are both even, a contradiction.
	Thus there is no $p\in\mb Q$ such that $p^2-2=0$.
\end{proof}

So one way $\mb Q$ is not adequate is there are not always solutions to such equations in $\mb Q$.
We want to characterize another more important way in which $\mb Q$ is not adequate for analysis.
Let $ A = \setc{p\in\mb Q}{p^2<2} $ and $B=\setc{p\in\mb Q}{p^2>2}$.
Claim: $A$ has no largest number $B$ has no smallest number.

\begin{proof}
	For any $p>0$ define $q = p-\frac{p^2-2}{p+2} = \frac{2p+2}{p+2}$ $(*)$ so $q>0$.
	Also
	\begin{align*}
		q^2-2 &= \frac{(2p+2)^2}{(p+2)^2} - 2 = \frac{(2p+2)^2-2(p+2)^2}{(p+2)^2} \\
			  &= \frac{4p^2+8p+4-2p^2-8p-8}{(p+2)^2} \\
		q^2-2 &= \frac{2p^2-4}{(p+2)^2} = \frac{2(p^2-2)}{(p+2)^2} \qquad (**)
	\end{align*}
	If $p\in A$, $p>0$, then $p^2-2<0$, so $q>p$ by $(*)$.
	Also, $q^2-2<0$ by $(**)$, so $q\in A$.
	So $A$ has no largest element. If $p\in B$, $p>0$, then $p^2-2>0$, so $q<p$ by $(*)$.
	Also, if $q^2-2>0$ by $(**)$, so $q\in B$.
\end{proof}

\begin{definition}
	Let $S$ be a set. An \ul{order} on $S$ is a relation, denoted $<$, with the properties
	\begin{enumerate}
		\item[(i)] If $x,y\in S$, then exactly one of the following is true. $x<y$ or $y<x$ or $x=y$.
		\item[(ii)] If $x,y,z\in S$ and $x<y$ and $y<z$, then $x<z$.
	\end{enumerate}
	Also $>,\leq,\geq$.
\end{definition}

\begin{definition}
	An \ul{ordered set} is a set on which an order is defined.
\end{definition}

\begin{definition}
	Let $S$ be ordered set and $E\subset S$. If $\exists\beta\in S$ such that $x\leq \beta$ for every $x\in E$, then we say $\beta$ is an \ul{upper bound of $E$}, and \ul{$E$ is bounded above}. Similar for lower bounds.
\end{definition}

\begin{definition}
	Let $S$ be ordered set, and suppose $E\subset S$ and $E$ is bounded above.
	If $\exists\alpha\in S$ such that
	\begin{enumerate}
		\item[(i)] $\alpha$ is an upper bound of $E$
		\item[(ii)] If $\gamma<\alpha$, then $\gamma$ is not an upper bound of $E$

			or

		\item[(ii)'] If $\gamma$ is an upper bound of $E$, then $\alpha\leq\gamma$.
	\end{enumerate}
	Then $\alpha$ is the least upper bound of $E$.
	Also called the \ul{supremum} of $E$, denoted $\alpha=\sup E$.
	Similar definition for infimum, greatest lower bound, $\inf E$.
\end{definition}

\begin{example}
	Consider $A$ and $B$ from previous (with only positive elements). $A$ is bounded above.
	The set of upper bounds of $A$ is $B$.
	Since $B$ has no least element, then $A$ has no least upper bound $\sup A$ does not exist in $\mb Q$.
\end{example}

HW Ch1: 1,2,4,5

\begin{definition}
	An ordered set $S$ has the least upper bound property if every nonempty subset $E\subset S$ which is bounded above has a supremum in $S$.
\end{definition}
\begin{note}
	by previous example, $\mb Q$ does not have l.u.b. property.
\end{note}

\begin{theorem}
	Suppose $S$ is an ordered set with l.u.b property, $B\subset S$, $B$ is nonempty, and bounded below. Then $\inf B$ exists in $S$.
	In particular, $\inf B=\alpha$ where $\alpha = \sup L$ and $L$ is the set of all lower bounds of $B$.
\end{theorem}
\begin{proof}
	Let $B$ and $L$ be as described. $L$ is nonempty since $B$ is bounded below. $L$ is bounded above because $B$ is nonempty and every element of $B$ is an upper bound of $L$.
	Since $S$ has l.u.b prperty, $\alpha=\sup L$ exists in $S$. We claim $\alpha=\inf B$.
	If $\gamma<\alpha$, then $\gamma$ is no an upper bound for $L$ (because $\alpha=\sup L$), so $\gamma\notin B$.
	Thus $\alpha$ is a lower bound for $B$.

	If $\alpha < \beta$, then $\beta\notin L$ since $\alpha=\sup L$.
	So $\beta$ is not a lower bound of $B$.
	Thus $\alpha=\inf B$.
\end{proof}

Fields and axioms: 1.12-1.16

\begin{example}
	$\mb Q$ is a field.
\end{example}

\begin{definition}
	An \ul{ordered field} is a field $F$ which is also an ordered set, satisfying
	\begin{enumerate}
		\item[(i)] $x+y<x+z$ if $x,y,z\in F$ and $y<z$.
		\item[(ii)] If $x,y\in F$ and $x>0$ and $y>0$, then $xy>0$.
	\end{enumerate}
\end{definition}
\begin{note}
	$\mb Q$ is an ordered field.
\end{note}

\begin{theorem}
	The following are true in any ordered field
	\begin{itemize}
		\item[a.] If $x>0$, then $-x<0$ and vice-versa.
		\item[b.] If $x>0$, and $y<z$, then $xy<xz$.
		\item[c.] If $x<0$, and $y<z$, then $xy>xz$.
		\item[d.] If $x\neq0$, then $x^2>0$. Thus $1>0$.
		\item[e.] If $0<x<y$, then $0<\frac1y<\frac1x$.
	\end{itemize}
\end{theorem}
\begin{proof}:
	\begin{itemize}
		\item[a.] If $x>0$, then $0=-x+x>-x+0=-x$. So $-x<0$.
		\item[b.] If $z>y$, then $z-y>0$, so $x(z-y)>0$. Thus $xz=x(z-y)+xy>0+xy=xy$.
		\item[d.] If $x>0$, then $x^2=x\cdot x>0$. If $x<0$, then $-x>0$, so $(-x)^2>0$.
			But $(-x)^2=x^2$ so $x^2>0$. Also, $1>0$ because $1=1^2$.
	\end{itemize}
\end{proof}

HW Chap1: 8, Grad: Let $S$ be a nonempty subset of $R$, bounded above and below, and let $b<0$. Show that $\inf(bS) = b\sup S$ and $\sup(bS) = b\inf S$.

\begin{note}
	$bS = \setc{bs}{s\in S}$.
\end{note}

Last time: least-upper-bound property, ordered fields

\begin{theorem}
	There exists an ordered field with the l.u.b. property, called the real numbers, denoted $\mb R$, and $\mb R$ contains $\mb Q$ as a subfield.
\end{theorem}
\begin{note}
	$\mb R$ contains $\mb Z$ and $\mb N$.
\end{note}
\begin{note}
	l.u.b. property is equivalent to completeness (in $\mb R^1$).
\end{note}
\begin{note}
	Proof using \say{cuts.}
\end{note}

\begin{theorem}:
	\begin{enumerate}
		\item[a.] (Archimedean Property) If $x,y\in\mb R$ and $x>0$, then there exists positive integer $n$ such that $nx>y$.
		\item[b.] (denseness) If $x,y\in\mb R$ and $x<y$, then $\exists p\in\mb Q$ such that $x<p<y$.
	\end{enumerate}
\end{theorem}
\begin{proof}
	[Proof of (a)]
	Let $x,y\in\mb R$ and $x>0$.
	Define set $A=\setc{nx}{n\in\mb N}$. BWOC, suppose (a) is false. Then $y$ is an upper bound of $A$. Since $A$ is nonempty, $A$ has a l.u.b. in $\mb R$, say $\alpha = \sup A$. Since $x>0$, $\alpha-x<\alpha$ so $\alpha-x$ is not an upper bound of $A$.
	Thus $\alpha-x\leq a$ for some $a\in A$. That is, $\alpha-x\leq mx$ for some $m\in\mb N$.
	This implies $\alpha\leq\ul{(m+1)x}$, which contradicts that $\alpha$ is an upper bound of $A$. Thus (a) is true.
\end{proof}
\begin{note}
	(a) is equivalent to: If $x\in\mb R$ and $x>0$, then $\exists n\in\mb N$ such that $\frac1n<x$.
\end{note}
\begin{note}
	(a) is equivalent to: $\mb N$ is unbounded above.
\end{note}

\begin{theorem}
	For every real $x>0$ and positive integer $n>0$, $\exists!$ positive real $y$ such that $y^n=x$.
\end{theorem}
\begin{proof}
	Uniqueness is clear since $0<y_1<y_2$ implies $y_1^n<y_2^n$.
	To show existence, define $E=\setc{t\in\mb R}{t^n<x}$.
	For $t=\frac x{x+1}$, we have $0<t<1$. Thus $t^n<t<x$, so $t\in E$, $E$ is nonempty.
	To show $E$ is bounded above, suppose $t\in\mb R$ and $t>x+1$.
	Then $t^n>t>x$, so $t\notin E$. So $x+1$ is an upper bound of $E$.
	Thus $E$ has a l.u.b., say $y=\sup E$. To show $y^n=x$, we show $y^n\nless x$ and $y^n\ngtr x$.
	We use $b^n-a^n<(b-a)nb^{n-1}$ $(*)$ when $0<a<b$. This follows from $b^n-a^n = (b-a)(b^{n-1}+b^{n-2}a+\dots+ba^{n-2}+a^{n-1})$.
	
	Suppose $y^n<x$. Pick $h$ such that $0<h<1$ and $h<\frac{x-y^n}{n(y+1)^{n-2}}$.
	With $a=y$, $b=y+h$ in $(*)$ we get
	\begin{align*}
		(y+h)^n - y^n &< hn(y+h)^{n-1} < hn(y+1)^{n-1} < x-y^n.
	\end{align*}
	Thus $(y+h)^n<x$, so $y+h\in E$. But $y+h>y$, contradicting $y=\sup E$.
	Thus $y^n\nless x$.

	Suppose $y^n$. Set $k = \frac{y^n-x}{ny^{n-1}}$, clearly $0<k$.
	Also, $ky^{n-1} \leq kny^{n-1} = y^n-x < y^n$. So $k<y$, $0<k<y$.
	To see that $y-k$ is an upper bound of $E$, suppose $t\geq y-k$. Then
	\begin{align*}
		y^n-t^n &\leq y^n-(y-k)^n \\
				&< kny^{n-1} \\
				&= y^n-x
	\end{align*}
	Thus $t^n>x$, so $t\notin E$. Thus $y-k$ is an upper bound of $E$.
	But $y-k<y$, contradicting $y=\sup E$. Thus $y^n\ngtr x$, so $y^n=x$.
\end{proof}

HW: Chap 1, 12-14,17, grad 9,10.

Last time: properties of $\mb R$

\begin{note}:
	\begin{itemize}
		\item $\mb R$ can be represented via infinite decimal expansion
		\item Extended real numbers consist of $\mb R$ with $\pm\infty$ and some algebra, such as $x+\infty=+\infty$, $\frac x{+\infty}=0$, $x\cdot(\infty) = \pm\infty$ depending on sign of $x$.
	\end{itemize}
\end{note}

One deficiency of $\mb R$ is not all polynomials have zeros in $\mb R$.
This can be \say{fixed} by considering the complex field.

\begin{definition}
	A complex number is an ordered pair $(a,b)$, where $a,b\in\mb R$.
	The set of complex numbers is a field with $+,\cdot$ defined by $(a,b)+(c,d)=(a+c,b+d)$ and $(a,b)\cdot(c,d)=(ac-bd,ad+bc)$.
\end{definition}
\begin{definition}
	$i=(0,1)$
\end{definition}
\begin{theorem}
	$i^2=-1=(-1,0)$
\end{theorem}
\begin{theorem}
	$(a,b)=a+bi$
\end{theorem}

\begin{definition}
	If $a,b\in\mb R$ and $z=a+bi$, then the \ul{complex conjugate} of $z$ is $\conj z=a-bi$. The \ul{absolute value} or \ul{modulus} of $z$ is $|z|=\sqrt{z\conj z}=\sqrt{a^2+b^2}$.
\end{definition}
\begin{note}
	$z\conj z \geq0$ and $z\conj z=0$ only when $z=0$.
\end{note}
\begin{proposition}
	Properties:
	\begin{enumerate}
		\item[(i)] $|\Re z|\leq|z|$
		\item[(ii)] $|z+w|\leq|z|+|w|$
	\end{enumerate}
\end{proposition}
\begin{proof}
	Let $z=a+bi,w=c+di$, and $a,b,c,d,\in\mb R$. Then
	\begin{enumerate}
		\item[(i)] $|\Re z|=|a|=\sqrt{a^2}\leq\sqrt{a^2+b^2}=|z|$
	\item[(ii)] Using $w\conj z = \conj{z\conj w}$,
		\begin{align*}
			|z+w|^2 &= (z+w)(\conj z+\conj w)=z\conj z+z\conj w+w\conj z+w\conj w = |z|^2+2\Re z\conj w+|w|^2 \\
					&\leq |z|^2+2|w\conj z|+|w|^2 = |z|^2+2|w||z|+|w|^2 = (|z|+|w|)^2
		\end{align*}
	\end{enumerate}
	So $|z+w|\leq|z|+|w|$ $\checkmark$
\end{proof}

\ul{Schwarz Inequality}: If $a_1,\dots,a_n$ and $b_1,\dots,b_n$ are complex, then
$$ \mlr{\sum_{j=1}^n a_j\conj{b_j}}^2 \leq \sum_{j=1}^n|a_j|^2\cdot\sum_{j=1}^n|b_j|^2.$$

\begin{example}
	[HW] Suppose $S\subset R$ bounded above and $a>0$. Show $\sup(aS)=a\sup S$.
	\begin{recall}
		$aS = \setc{as}{s\in S}$
	\end{recall}
	Let $\alpha = \sup S$. Want to show $\sup(aS)=a\alpha$. To do this, show $a\alpha$ is an upper bound of $aS$, and if $\gamma$ is an upper bound of $aS$, then $a\alpha\leq\gamma$.
	Let $x\in aS$. Then $x=a\cdot s$ for some $s\in S$. Thus $s\leq\alpha$, so $as\leq a\alpha$, $x\leq a\alpha$.
	So $a\alpha$ is an upper bound of $aS$. Suppose $\gamma$ is an upper bound of $aS$.
	Then $\gamma\geq as$ for all $s\in S$, so $\frac\gamma a\geq s$ for all $s\in S$.
	Thus $\frac\gamma a$ is an upper bound of $S$, so $\frac\gamma a\geq\alpha$ since $\alpha=\sup S$. So $\gamma\geq a\alpha$. Thus $a\alpha$ is the l.u.b. (or $\sup$) of $aS$.
\end{example}



\end{document}
