\documentclass[]{article}
\usepackage[latin1]{inputenc}
\usepackage{graphicx}
\usepackage[left=1.00in, right=1.00in, top=1.10in, bottom=1.00in]{geometry}

\usepackage{dirtytalk}
\usepackage[normalem]{ulem}
\usepackage{tikz-cd}
\usepackage{units}
\usepackage{algorithm}
\usepackage{algpseudocode}
\usepackage{alltt}
\usepackage{mathrsfs}
\usepackage{amssymb}
\usepackage{amsmath}
\DeclareMathOperator\cis{cis}

% (font shortcuts)
\usepackage{amsfonts}
\newcommand{\mb}[1]{\mathbb{#1}}
\newcommand{\mc}[1]{\mathcal{#1}}
\newcommand{\ms}[1]{\mathscr{#1}}
\newcommand{\mf}[1]{\frak{#1}}

% (arrow shortcuts)
\newcommand{\ra}{\rightarrow}
\newcommand{\lra}{\longrightarrow}
\newcommand{\la}{\leftarrow}
\newcommand{\lla}{\longleftarrow}
\newcommand{\Ra}{\Rightarrow}
\newcommand{\Lra}{\Longrightarrow}
\newcommand{\La}{\Leftarrow}
\newcommand{\Lla}{\Longleftarrow}
\newcommand{\lr}{\leftrightarrow}
\newcommand{\llr}{\longleftrightarrow}
\newcommand{\Lr}{\Leftrightarrow}
\newcommand{\Llr}{\Longleftrightarrow}

% (match parenthesis)
\newcommand{\mlr}[1]{\left|#1\right|}
\newcommand{\plr}[1]{\left(#1\right)}
\newcommand{\blr}[1]{\left[#1\right]}

% (exponent shortcuts)
\newcommand{\inv}{^{-1}}
\newcommand{\nrt}[2]{\sqrt[\leftroot{-2}\uproot{2}#1]{#2}}

% (annotation shortcuts)
\newcommand{\conj}[1]{\overline{#1}}
\newcommand{\ol}[1]{\overline{#1}}
\newcommand{\ul}[1]{\underline{#1}}
\newcommand{\os}[2]{\overset{#1}{#2}}
\newcommand{\us}[2]{\underset{#1}{#2}}
\newcommand{\ob}[2]{\overbrace{#2}^{#1}}
\newcommand{\ub}[2]{\underbrace{#2}_{#1}}
\newcommand{\bs}{\backslash}
\newcommand{\ds}{\displaystyle}

% (set builder)
\newcommand{\set}[1]{\left\{ #1 \right\}}
\newcommand{\setc}[2]{\left\{ #1 : #2 \right\}}
\newcommand{\setm}[2]{\left\{ #1 \, \middle| \, #2 \right\}}

% (group generator)
\newcommand{\gen}[1]{\langle #1 \rangle}

% (functions)
\newcommand{\im}[1]{\text{im}(#1)}
\newcommand{\range}[1]{\text{range}(#1)}
\newcommand{\domain}[1]{\text{domain}(#1)}
\newcommand{\dist}[1]{(#1)}
\newcommand{\sgn}{\text{sgn}}

% (Linear Algebra)
\newcommand{\mat}[1]{\begin{bmatrix}#1\end{bmatrix}}
\newcommand{\pmat}[1]{\begin{pmatrix}#1\end{pmatrix}}
%\newcommand{\dim}[1]{\text{dim}(#1)}
\newcommand{\rnk}[1]{\text{rank}(#1)}
\newcommand{\nul}[1]{\text{nul}(#1)}
\newcommand{\spn}[1]{\text{span}\,#1}
\newcommand{\col}[1]{\text{col}(#1)}
%\newcommand{\ker}[1]{\text{ker}(#1)}
\newcommand{\row}[1]{\text{row}(#1)}
\newcommand{\area}[1]{\text{area}(#1)}
\newcommand{\nullity}[1]{\text{nullity}(#1)}
\newcommand{\proj}[2]{\text{proj}_{#1}\left(#2\right)}
\newcommand{\diam}[1]{\text{diam}\,#1}

% (Vectors common)
\newcommand{\myvec}[1]{\vec{#1}}
\newcommand{\va}{\myvec{a}}
\newcommand{\vb}{\myvec{b}}
\newcommand{\vc}{\myvec{c}}
\newcommand{\vd}{\myvec{d}}
\newcommand{\ve}{\myvec{e}}
\newcommand{\vf}{\myvec{f}}
\newcommand{\vg}{\myvec{g}}
\newcommand{\vh}{\myvec{h}}
\newcommand{\vi}{\myvec{i}}
\newcommand{\vj}{\myvec{j}}
\newcommand{\vk}{\myvec{k}}
\newcommand{\vl}{\myvec{l}}
\newcommand{\vm}{\myvec{m}}
\newcommand{\vn}{\myvec{n}}
\newcommand{\vo}{\myvec{o}}
\newcommand{\vp}{\myvec{p}}
\newcommand{\vq}{\myvec{q}}
\newcommand{\vr}{\myvec{r}}
\newcommand{\vs}{\myvec{s}}
\newcommand{\vt}{\myvec{t}}
\newcommand{\vu}{\myvec{u}}
\newcommand{\vv}{\myvec{v}}
\newcommand{\vw}{\myvec{w}}
\newcommand{\vx}{\myvec{x}}
\newcommand{\vy}{\myvec{y}}
\newcommand{\vz}{\myvec{z}}
\newcommand{\vzero}{\myvec{0}}

% Theorems and Propositions
\usepackage{amsthm}
\newtheorem{theorem}{Theorem}
\newtheorem{proposition}{Proposition}

\theoremstyle{definition}
\newtheorem{definition}{Definition}

\theoremstyle{remark}
\newtheorem*{remark}{Remark}
\newtheorem{example}{Example}
\newtheorem*{recall}{Recall}
\newtheorem*{note}{Note}
\newtheorem*{observe}{Observe}
\newtheorem*{question}{\underline{Question}}
\newtheorem*{fact}{Fact}
\newtheorem{corollary}{Corollary}
\newtheorem*{lemma}{Lemma}
\newtheorem{xca}{Exercise}


\author{Presenter: Brett Tangedal, Notes by Michael Reed, Book: Russell Gordon 2nd}
\title{Introduction to Real Analysis}
\date{Fall 2017, \today}

\begin{document}
\maketitle

%\subsection{First Exam}
Calculus was developed approximately from the 1630s to 1700 by Newton (1642-1727) and Leibniz (1646-1716). The Elements of Euclid (300 BC) developed mathematics using axiomatic and rigorous development of Euclidean geometry. In 1821 the French mathematician Cauchy gave the first mainly rigorous development of Calculus.
\begin{example}
	$\mb Z =\{\dots,-3,-2,-1,0,1,2,3,\dots\}$ are the integers or Zahlen.
\end{example}

\begin{proposition}
	[M5] Given any nonzero $c\in \mb Q$, there exists an element $d\in\mb Q$ such that $c\cdot d=1$.
\end{proposition}

\begin{example}
	What is the smallest set of numbers that contains $\mb Z$ and where axiom (M5) holds? Answer: Set of rational numbers $\mb Q = \{a/b\mid a,b\in\mb Z,b\neq0\}$. $\mb Q$ is a field.
\end{example}

%On p.6 of the book:
\begin{definition}
	A field is a set of \ul{at least} 2 elements that has 2 operations defined on it.
\end{definition}

On a field $\mb F$, the additive identity (0) does not equal the multiplicative identity (1).

\begin{example}
	Consider $\mb F_2$, a field which contains 2 elements $\bar 0,\bar 1$.
\end{example}

\begin{example}
	[HW Example (9) on p.9] Let $x$ be an irrational number. Prove that there exists an irrational number $y$ such that $xy$ is a rational number.
	\begin{proof}
		\textit{I worked (alone) (with $X$)}.
		
		(M5): Given any non-zero real number $a$, there exists an element $b\in\mb R$ such that $ab=1$.
		
		We know there exists a non-zero number $y$ such that $xy=1$. Since $x$ is irrational it must be non-zero. \ul{Assume} $y$ is rational. If $y$ is rational, so is $\frac{1}{y}$ since $x=\frac{1}{y}$. So $x$ is rational, which is a contradiction since $x$ was assumed to be irrational.
	\end{proof}
\end{example}

\begin{theorem}
	If $x\in\mb R$ and $\forall\epsilon>0(|x|<\epsilon)$, then $x=0$ by (A12).
\end{theorem}
\begin{proof}
	[Proof (by Contradiction)] \ul{Assume} $x\neq 0$. By (Q2) we have $|x|\neq 0$ and so $0<|x|$. Let $\epsilon=\frac{|x|}{2}$ and note that we have $\epsilon>0$. By assumption, $|x|<\frac{|x|}{2}$ implies that multiplication by $\frac{1}{|x|}>0$. Then $1<\frac{1}{2}$, which is a contradiction. Therefore, $x=0$.
\end{proof}

\ul{Given}: $\mb R$ is an \say{ordered field} and $\mb Q$ is an \say{ordered field}. However, we still need 1 more axiom to completely distinguish $\mb R$ from $\mb Q$ and do calculus.

\begin{example}
	$\sqrt{2}$ is an irrational number. $\mb R$ is the set of real numbers.
\end{example}

We're all comfortable with the rational numbers $\mb Q$ and we use the \say{metric} $|x-y|$ to measure the distance between 2 rational numbers $x$ and $y$.

Irrational numbers have always been trickier and harder to grasp. On the other hand, almost all numbers that arise in science are irrational, such as $\sqrt{2}$, $5^{1/3}$, $\pi$, $\ln{4}$, $\sin{27\deg}$, $\tan{17\deg}$, $\dots$, so the applications of math to the sciences actually \ul{forces} us to come to grips with the irrational numbers.
How do we think of $\pi$, for example, as a number? We write it as $\pi = 3.1415926\dots$. Notice that there is a (infinite) sequence of \ul{rational} numbers implied here: $$\frac{3}{1},\frac{31}{10},\frac{314}{100},\frac{3141}{1000},\cdots.$$
We can denote this sequence as $x_1,x_2,x_3,x_4,\dots$ and with each rational number in this sequence, we get \say{one step} closer to $\pi$.
We have $x_1<x_2<x_3<\dots$.

Since we understand $\mb Q$ so well, we want to use $\mb Q$ as our \say{base field.}
The only way to understand irrational numbers in terms of rational numbers is by way of limits, i.e. by use of infinite processes.
The sequence $x_1,x_2,x_3,\dots$ is a sequence of rational numbers that \ul{converges} to an irrational number.

The new axiom we still need to completely characterize the system of real numbers $\mb R$ is required to ensure that certain sequences of rational and real numbers converge \ul{within} the real number system.

\ul{$\sim$ 1870}: One can construct the set of real numbers from the set of rational numbers alone (Weierstrass, Cantor, Dedekind).

Consider the prime numbers: $2,3,5,7,\dots$
If you complete $\mb Q$ with respect to the usual metric, you get the $\mb R$. If you complete $\mb Q$ with respect to the $p$-adic metric you get $\mb Q_p$.

Completeness can be proven as a theorem; however, for our purposes we are going to assume it as an axiom.

\begin{definition}
	[Function] A function $f:A\ra B$ is a rule of correspondence that assigns to each element of the set $A$ exactly one element in the set $B$ such that $a\in A \mapsto f(a) \in B$. The \ul{domain} is $A$ and the \ul{codomain} is $B$. The \ul{range} is $\{f(a) \mid a\in A\}$.
\end{definition}

\begin{definition}
	[Infinite sequence of real numbers] A sequence of real numbers is a \ul{function} $f:\mb Z^+ \ra \mb R$ with domain $\mb Z^+$ and codomain $\mb R$. A better definition: The domain is instead $\{n\in \mb Z\mid n\geq m\}$, where $m$ is a fixed integer. If $m=1$, we recover the book definition, but sometimes we want $m=0$ or even $m=37$.
\end{definition}

$f(1)=x_1,f(2)=x_2, f(3)=x_3,\dots$ can be represented as $\{x_n\}_{n=1}^\infty = \{x_n\}$

\begin{definition}
	If $\{x_n\}$ is an infinite sequence of real numbers, we say that $x_1,x_2,x_3,\dots$ converges to $x\in\mb R$ if given any $\epsilon>0$, there exists a positive integer $N\in\mb Z$ such that we have $|x_n-x|<\epsilon$ for all $n\geq N$.
\end{definition}

\begin{definition}
	A sequence $x_1,x_2,x_3,\dots$ of real numbers is said to converge to a number $L\in\mb R$ if for each $\epsilon>0$ or $\forall\epsilon>0$ there exists $N\in\mb Z^+$ such that \begin{equation*} \label{eq1}
	\forall n\geq N(|x_n -L|<\epsilon),
	\end{equation*}
	where $\epsilon$ is an arbitrary positive real number.
\end{definition}
The game we are always going to play is: given some $\epsilon$ find $N$
\begin{note}
	$(|x_n-L|<\epsilon) \implies (L-\epsilon<x_n<L+\epsilon)$.
\end{note}
\begin{remark}
	Infinitesimals are \say{ghosts of departed quantities,} we don't need them here.
\end{remark}

\begin{definition}
	The sequence $\{x_n\}$ is convergent (or converges) if there exists a number $L\in \mb R$ such that $\{x_n\}$ converges to $L$. The sequence $\{x_n\}$ is divergent (or diverges) if $\{x_n\}$ does not converge.
\end{definition}
\begin{example}
	$x_1=1,x_2=-1,x_3=1,x_4=-1,\dots$ does not converge.
\end{example}

\begin{theorem}
	If $\{x_n\}$ is a convergent sequence, then the number $L$ appearing in equation \ref{eq1} above is \ul{uniquely} \ul{defined}. $L$ is called \ul{The} limit of the sequence $\{x_n\}$.
\end{theorem}
\begin{proof}
	Let $\{x_n\}$ be a convergent sequence and suppose that $\{x_n\}$ converges to the real numbers $a$ and $b$. Let $\epsilon>0$. By definition, there exists $N_1\in\mb Z^+$ such that $\forall n\geq N_1 (|x_n -a|< \frac{\epsilon}{2})$. Similarly, there exists $N_2\in \mb Z^+$ such that $\forall n\geq N_2 (|x_n-b|<\frac{\epsilon}{2})$. Set $M = \max{\{N_1,N_2\}}$. Then by triangle inequality we get $$|a-b|=|a-x_M + x_M -b| \leq |a-x_M|+|x_M-b| = |x_M-a|+|x_M-b|.$$
	We know that $|x_M-a|<\frac{\epsilon}{2}$ and $|x_M-b|<\frac{\epsilon}{2}$. So $|x_M-a|+|x_M-b|<\frac{\epsilon}{2} + \frac{\epsilon}{2} = \epsilon$. Therefore, $|a-b|<\epsilon$ for every $\epsilon>0$. By (A12) $a-b=0$ or $a=b$, so the limit $L$ is uniquely defined.
\end{proof}

Homework: p.58: look at 1-5, read sec. 2.1

\begin{definition}
	A sequence $x_1,x_1,\dots$ of real numbers is said to converge to a number $L\in\mb R$ if(f) $\forall\epsilon>0$ there exists $N\in\mb Z^+$ such that $|x_n-L|<\epsilon$ for all $n\geq N$.
\end{definition}
\newpage
\begin{definition}
	Let $\{x_n\} $ be a sequence of real numbers
	\begin{enumerate}
		\item[a)] The sequence $\{x_n\}$ is \ul{bounded above} if there exists $M\in\mb R$ such that $x_n\leq M$ for all $n\geq 1$. The number $M$ is an upper bound of the $\{x_n\}$. 
		\item[d)] The sequence $\{x_n\}$ is \ul{increasing} if $x_n\leq x_{n+1}$ for all $n\geq 1$. The sequence $\{x_n\}$ is strictly increasing if $x_n< x_{n+1}$ for all $n\geq 1$.
	\end{enumerate}
\end{definition}

\begin{example}
	$\frac{1}{1},\frac{1}{2},\frac{1}{3},\cdots$ is strictly decreasing and bounded below (0 is a lower bound).	
\end{example}
\begin{example}
	$1,1,1,1,\dots$ is an increasing sequence (and decreasing).
\end{example}

\begin{definition}
	[Completeness Axiom (version I)] An infinite sequence $x_1,x_2,\dots$ of real numbers that is increasing and bounded above \ul{always} converges to a unique real number $L\in\mb R$. This also works when the sequence is decreasing and bounded below.
\end{definition}
\begin{note}
	$\mb R$ is a \ul{complete} \ul{ordered} field.
\end{note}
\begin{example}
	$x_n=(1+\frac{1}{n})^n$, so $x_1=2$, $x_2=9/4$, $x_3=64/27$. Can prove that this sequence is strictly increasing and bounded above by 3. By the completeness axiom, this sequence converges to a unique real number. We'll call this number $e$.
\end{example}
For any positive real number $t>0$, it can be proved that there exists a positive real number denoted by $\sqrt{t}$ such that $(\sqrt{t})^2=t$ and there is no other positive real number whose square is $t$. You need the completeness axiom to prove this! We define: $\sqrt{0} = 0$.

\begin{enumerate}
	\item[s1.] $|t|=\sqrt{t^2}$ for all $t\in\mb R$.
	\item[s2.] If $0\leq s\leq t$, then $\sqrt{s}\leq\sqrt{t}$.
	\item[s3.] $\sqrt{s}\cdot\sqrt{t} = \sqrt{s\cdot t} $ for all $s,t\geq0\in\mb R$.
\end{enumerate}

\begin{note}
	$||(x_1,y_1)-(x_2,y_2)|| = \sqrt{(x_1-x_2)^2+(y_1-y_2)^2}$ is the definition of distance between points in $\mb R^2$.
\end{note}
\begin{remark}
	A sequence $(x_1,y_1),(x_2,y_2),\dots$ of points in $\mb R^2$ is said to converge to a point $(L,M)\in\mb R^2$ if $\forall\epsilon>0$ there exists $N\in\mb Z^+$ such that $ ||(x_n,y_n)-(L,M)|| < \epsilon $ for all $n\geq N$.
\end{remark}

HW due Friday Feb.3, Prove:

Problem 1: If $\lim_{n\ra\infty} x_n=L$ and $\lim_{n\ra\infty}y_n=M$, then this $(x_1,y_1),(x_2,y_2),\dots$ converges to $(L,M)$

Problem 2: p.58 (4); 
Problem 3: p.58 (11)a; 
Problem 4: p.59 (18)

\begin{proposition}
	\ul{Archimedean Property of $\mb R$}: Given any $x\in\mb R$ there exists $N\in\mb Z^+$ such that $x<N$.
\end{proposition}

\begin{proof}
	Assume otherwise (proof by contradiction)! Note that $x\in\mb R$ is \ul{fixed}. We are trying to prove that
	\begin{enumerate}
		\item There exists $N\in\mb Z^+$ such that $x<N$
		\item The negation of (1) reads: $n\leq x$ for all $n\in\mb Z^+$. 
	\end{enumerate}
	Using the completeness axiom, we will show that (2) leads to a contradiction.
	Note that the positive integers $1,2,3,\dots$ form  a strictly increasing (infinite) sequence of real numbers.
	By assumption (2), this sequence is bounded above by the real  number $x$. Therefore, by the completeness axiom there exists a unique real number $L\in \mb R$ such that $\lim_{n\ra\infty} n = L$. We now show this is impossible and therefore we have a contradiction! Set $\epsilon=\frac{1}{2}$. There is at most 1 positive integer in the open interval $(L-\frac{1}{2},L+\frac{1}{2})$.
	If $1,2,3,\dots$ really did converge to $L$, we would need more than 1 positive integer in $(L-\frac{1}{2},L+\frac{1}{2})$ (actually we need an infinite number).
\end{proof}
\begin{note}
	When we negate an existential quantifier, we change over to a universal quantifier. If we negate a universal quantifier we change over to an existential quantifier.
\end{note}

\begin{proposition}
	$\{\frac{7n}{4n+5}\} \underset{n\ra\infty}{\longrightarrow} \frac{7}{4}$. Have to show $\forall\epsilon>0$ there exists $N\in\mb Z^+$ such that $|x_n-L|<\epsilon$ for all $n\geq N$. Given $\epsilon$. Goal: Find $N\in\mb Z^+$ such that this all works out.
\end{proposition}
\begin{proof}
	$|\frac{7n}{4n+5}-\frac{7}{4}|=|\frac{28n-7(4n+5)}{4(4n+5)}|=|\frac{-35}{4(4n+5)}|=\frac{35}{4(4n+5)} < \frac{35}{4(4n)}<\frac{3}{n}\leq \frac{3}{N}<\epsilon$ because $N\leq n\implies \frac{1}{n}\leq\frac{1}{N}$. Choose $\epsilon>0$. Let $x=\frac{3}{\epsilon}$. By the Archimedean principle of $\mb R$, there exists $N\in\mb Z^+$ such that $x<N$ or $\frac{3}{\epsilon}<N$ or $\frac{3}{N}<\epsilon$.
\end{proof}

p.54) In order for a sequence \ul{not} to converge to $L$, there must be at least 1 positive real number $\epsilon(>0)$ such that for every positive integer $N$ there exists an integer $n\geq N$ such that $|x_n-L|\geq \epsilon$.


p.54) $x_1,x_2,x_3,\dots$ converges to $L\in\mb R$ if $\forall \epsilon >0$ there exists $N\in\mb Z^+$ such that $|x_n-L|<\epsilon$ for all $n\geq N$.
%$$\lim_{n\ra} x_n=L$$
\begin{theorem}
	If $\{x_n\}$ is convergent (say $\lim_{n\ra\infty} x_n =L$), then $\{x_n\}$ is bounded (there exists $M$, a positive real number, such that $|x_n|\leq M$ for all $n\geq 1$). 
	\label{thm2-5}
\end{theorem}
\begin{proof}
	Set $\epsilon=1$. By assumption there exists $N\in\mb Z^+$ such that $|x_n-L|<1$ for all $n>N$.
	\ul{Claim}: The positive real number $M= \max{\{|x_1|,|x_2|,\dots,|x_N|,|L|+1\}}$ is a bound for our sequence.
	\begin{note}
		$|x_j|\leq M$ for $j=1,\dots,N$.
	\end{note}
	Assume $n>N$. Then $|x_n| = |x_n-L+L|\leq |x_n-L| + |L| < 1+|L| \leq M$.
\end{proof}
If $A$ then $B$. Contrapositive is: If not $B$, then not $A$.
\ul{Contrapositive of Theorem \ref{thm2-5}}: If $\{x_n\}$ is not bounded, then $\{x_n\}$ does not converge.
Converse to \ul{If $A$, then $B$}: If $B$, then $A$.
\ul{Converse of Theorem \ref{thm2-5}}: If $\{x_n\}$ is bounded, then $\{x_n\}$ converges. $\la$ This is \ul{not} true. Counter example: $\{(-1)^n\} = -1,1,-1,1,\dots$.

\begin{theorem}
	Suppose that $\lim_{n\ra\infty} a_n = L$ and $\lim_{n\ra\infty} b_n = M$. Then 
	\begin{enumerate}
		\item[a)] $\{ca_n\} = ca_1,ca_2,ca_3,\dots$ converges to $c\cdot L$ for any given fixed constant $c\in\mb R$.
		\begin{proof}
			True if $c=0$, so assume $c\neq 0$. Since $c$ can be factored, $|ca_n-c\cdot L| = |c||a_n-L|$. By definition, there exists $N\in\mb Z^+$ such that $|a_n-L|<\frac{\epsilon}{|c|}$ for all $n\geq N$. Therefore, $\forall n\geq N$ we have $|ca_n-cL|=|c||a_n-L| < |c|\cdot \frac{\epsilon}{|c|} = \epsilon$.
		\end{proof}
		\item[b)] $\{a_n+b_n\} = a_1+b_1,a_2+b_2,a_3+b_3,\dots$ converges to $L+M$.
		\begin{proof}
			We want to show that $|(a_n+b_n) - (L+m)| = |(a_n-L)+(b_n-M)|\leq |a_n-L| + |b_n-M| <\epsilon$ by triangle inequality. Let $\epsilon>0$. By definition, there exists $N_1$ such that $|a_n-L|<\frac{\epsilon}{2}$ for all $n\geq N_1$. By assumption there exists $N_2$ such that $|b_n-M|<\frac{\epsilon}{2}$ for all $n\geq N_2$. Let $N=\max{\{N_1,N_2\}}$.
			\ul{Claim}: $|(a_n+b_n) - (L+M)|< \epsilon$ holds $\forall n\geq N$ since $|(a_n-L)+(b_n-M)|\leq |a_n-L|+|b_n-M| <\frac{\epsilon}{2} + \frac{\epsilon}{2} = \epsilon$.
		\end{proof}
		\item[c)] $\{a_n-b_n\}$ converges to $L-M$.
		\begin{proof}
			By Setting $c=-1$, $-b_1,-b_2,\dots \ra (-1) M$ by (a). Then by (b), this sequence converges to $L+(-1)M$.
		\end{proof}
		\item[d)] $\{a_n\cdot b_n\}= a_1b_1,a_2b_2,a_3b_3\dots \ra L\cdot M$.
		\begin{proof}
			\ul{See book}.
		\end{proof}
		\item[e)] Assume furthermore that $b_j\neq 0$ for all $j\geq 1$ and that $M\neq 0$. Then $\{\frac{a_n}{b_n}\}$ converges to $\frac{L}{M}$
		\begin{proof}
			First, we wish to prove that $\frac{1}{b_1},\frac{1}{b_2},\frac{1}{b_3},\dots$ converges to $\frac{1}{M}$. (once we know this, apply (d) to obtain the general case). Since $M\neq 0$, $|M|>0$. Set $\alpha = \frac{|M|}{2}>0$. Since $\{b_n\}$ converge to $M$, we know there exists $N_1\in\mb Z^+$ such that $|b_n-M|<\alpha$ for all $n\geq N_1$. So $|M-b_n|<\alpha$ and $ |M|-|b_n| \leq ||M|-|b_n||\leq |M-b_n|<\alpha$ by (a7) and (a10). or: $|M|-\alpha < |b_n|$ for all $n\geq N_1$. Since $\alpha = \frac{|M|}{2}$, this becomes: $\frac{|M|}{2} < |b_n|$ for all $n\geq N_1$ or $\frac{1}{|b_n|}<\frac{2}{|M|}$ for all $n\geq N_1$. The sequence $\{\frac{1}{b_n}\}$ is bounded as $\left| \frac{1}{b_n} \right| \leq C$ for all $n\geq 1$, if we use $C = \max{\left\{\left|\frac{1}{b_1}\right|,\dots,\left|\frac{1}{b_{N_1}}\right|,\frac{2}{\left|M\right|}\right\}}$.
		\end{proof}
	\end{enumerate}
	\label{thm2.7}
\end{theorem}

\ul{Study for exam}: Don't worry about Prob 1 on HW1 /HW2. Look at Definitions, Theorems (hand out). HW problems. (7 problems on exam).

\newpage

%\begin{theorem}
%	Suppose that the sequence $\{a_n\}$ converges to $L$ and that the sequence of $\{b_n\}$ converges to $M$.
%	\begin{enumerate}
%		\item[a)] $\{ca_n\}$ converges to $cL$
%		\item[b)] $\{a_n+b_n\}$ converges to $L+M$
%		\item[c)] $\{a_n-b_n\}$ converges to $L-M$
%		\item[d)] $\{a_n\cdot b_n\}$ converges to $L\cdot M$
%		\item[e)] Assume furthermore that $b_j\neq 0$ for all $j\geq 1$ and that $M\neq 0$. Then $\{\frac{a_n}{b_n}\}$ converges to $\frac{L}{M}$
%		\begin{proof}
%			First, we wish to prove that $\frac{1}{b_1},\frac{1}{b_2},\frac{1}{b_3},\dots$ converges to $\frac{1}{M}$. (once we know this, apply (d) to obtain the general case). Since $M\neq 0$, $|M|>0$. Set $\alpha = \frac{|M|}{2}>0$. Since $\{b_n\}$ converge to $M$, we know there exists $N_1\in\mb Z^++$ such that $|b_n-M|<\alpha$ for all $n\geq N_1$. So $|M-b_n|<\alpha$ and $ |M|-|b_n| \leq ||M|-|b_n||\leq |M-b_n|<\alpha$ by (a7) and (a10). or: $|M|-\alpha < |b_n|$ for all $n\geq N_1$. Since $\alpha = \frac{|M|}{2}$, this becomes: $\frac{|M|}{2} < |b_n|$ for all $n\geq N_1$ or $\frac{1}{|b_n|}<\frac{2}{|M|}$ for all $n\geq N_1$. The sequence $\{\frac{1}{b_n}$ is bounded as $\left| \frac{1}{b_n} \right| \leq C$ for all $n\geq 1$, if we use $C = \max{\{|\frac{1}{b_n},\dots,|\frac{1}{b_{N_1}}|,\frac{2}{|M|}\}}$.
%		\end{proof}
%	\end{enumerate}
%\end{theorem}

Bounded above ($\forall n\geq1(x_n\leq M)$) \& bounded below ($\forall n\geq 1(m\leq x_n)$) $\implies $ bounded ($\forall n\geq 1(|x_n|\leq B)$, where $B\in\mb R^+$). Since $-B\leq x_n \leq B$ so $B=\max{\{|m|,|M|\}}$.
\begin{example}
	[Exam 1, 8.] $0\leq x_n$ for all $n\geq 1$. $\{x_n\}$ converges to $L$. We want to prove \ul{$0\leq L$}. 
\end{example}
\begin{proof}
	Proof by contradiction: Assume $L<0$. Given any $\epsilon>0$ there exists $N\in\mb Z^+$ such that $|x_n - L|<\epsilon$ for all $n\geq N$, so $L-\epsilon<x_n<L+\epsilon$. Pick $\epsilon$ such that $L+\epsilon=0$ by choosing $\epsilon=-L>0$. For this $\epsilon$ there exists $N\in \mb Z^+$ such that $L-\epsilon<x_n<L+\epsilon=0$ for all $n\geq N$. Therefore $x_N<0$. \ul{Contradiction}.
\end{proof}
\begin{recall}
	$b_1, b_2, b_3,\dots \ra M$, where $b_j \neq 0$ for all $j\geq 1$ and $M\neq0$. Want to prove: $\frac{1}{b_1},\frac{1}{b_2},\frac{1}{b_3},\dots \ra \frac{1}{M}$. 
\end{recall}
\begin{proof}
	From Monday: Showed there exists $N_1 \in \mb Z^+$ such that $\frac{1}{|b_n|}<\frac{2}{|M|}$ for all $n\geq N_1$. Note that
	\begin{equation}
		\left|\frac{1}{b_n}-\frac{1}{M}\right| = \left| \frac{M-b_n}{b_n\cdot M}\right| = \frac{|M-b_n|}{|b_n||M|} = \frac{|b_n-M|}{|b_n||M|}.
		\label{equation1}
	\end{equation}
	Since $\frac{1}{|b_n|}<\frac{2}{|M|}$ %\leq \frac{2|b_n-M|}{|M|^2}<\epsilon$ 
	for all $n\geq N_1$, we have %$\implies$ 
	$\frac{1}{|b_n||M|} < \frac{2}{|M|^2}$ for all $n\geq N_1$.
	Let an arbitrary $\epsilon>0$ be given. Since $\{b_n\}$ converges to $M$ by assumption, there exists $N_2\in \mb Z^+$ such that $|b_n-M|<\frac{|M|^2\epsilon}{2}$ for all $n\geq N_2$. 
	Set $N=\max{\{N_1,N_2\}}$. Then we have $\frac{|b_n-M|}{|b_n||M|}<\frac{2}{|M|^2} \cdot \frac{|M|^2}{2}\cdot \epsilon = \epsilon$ for all $n\geq N$. Going back to equation~(\ref{equation1}), we have
	%So $\left| \frac{1}{b_n}-\frac{1}{M} \right| \leq \frac{2|b_n-M|}{|M|^2}$ for all $n\geq N_1$, so $|b_n-M| < \frac{|M|^2\epsilon}{2}$. Then $\frac{2|b_n-M|}{|M|^2} < \epsilon$ for all $n\geq N_2$. Let $N=\max{\{N_1,N_2\}}$. Put together: 
	$\left| \frac{1}{b_n}-\frac{1}{M} \right| < \epsilon$ for all $n\geq N$.
\end{proof}

\begin{example}
	[Exam 1, 7.] $a_n\leq x_n \leq b_n$ for all $n\geq 1$.
\end{example}
\begin{proof}
	We are assuming that $\lim_{n\ra\infty} a_n = L$ and $\lim_{n\ra\infty} b_n = L$. Given $\epsilon>0$, there exists $N_1$ such that $|a_n-L|<\epsilon$ for all $n\geq N_1$ and there exists $N_2\in\mb Z^+$ such that $|b_n-L|<\epsilon$ for all $n\geq N_2$.
	Want to prove that $\lim_{n\ra\infty} x_n = L$. We have $L-\epsilon<a_n<L+\epsilon$ for all $n\geq N_1$ and $L-\epsilon<b_n<L+\epsilon$ for all $n\geq N_2$.
	Let $N=\max{\{N_1,N_2\}}$ Then $L-\epsilon<a_n\leq x_n\leq b_n<L+\epsilon$ for all $n\geq N$, which proves that $\lim_{n\ra\infty} x_n=L$.
\end{proof}

\begin{theorem}
	Let $a,b\in\mb R$ with $a<b$. Assume $\{x_n\}$ is a sequence with $a\leq x_n \leq b$ for all $n\in \mb Z^+$ and that $\{x_n\}$ converges to $L$. Then $a\leq L\leq b$.
	\label{thm5}
\end{theorem}
\begin{proof}
	(sketch) Look at the sequence $x_1-a,x_2-a,x_3-a,\dots \ra L-a$, where $0\leq x_n -a$ for all $n\geq 1$. Then (\ul{number 8 on Exam 1}) $0\leq L-a$. To get other inequality look at $b-x_1,b-x_2,\dots \ra b-L$.
\end{proof}

\begin{recall}
	[\ul{Archimedean Property of $\mb R$}] Given any $x\in \mb R$, there exists $N \in \mb Z^+$ such that $x<N$.
\end{recall}

\begin{corollary}
	Given any $x\in\mb R$, there exists $N\in\mb Z^+$ such that $-N<x$.
	\label{cor1}
\end{corollary}
\begin{proof}
	If $x\geq 0$, just pick $N=1$. If $x<0$ then we use the Archimedean property to find $N\in\mb Z^+$ such that $-x<N \implies x>-N$.
\end{proof}
\begin{corollary}
	If $t>0$, there exists $N\in\mb Z^+$ such that $0<\frac{1}{N}<t$.
	\label{cor2}
\end{corollary}
\begin{proof}
	By the Archimedean property, there exist $N\in\mb Z^+$ such that $\frac{1}{t}<N \implies 1<t\cdot N$ by (i3) $\implies 0<\frac{1}{N}<t$ by (i6).
\end{proof}
\begin{definition}
	$\mb Z^+ = \{1,2,3,\dots\}$.
	A nonempty subset $A\subseteq \mb Z$ contains a least element if there exists $q\in A$ such that $q\leq a$ for all $a\in A$.
\end{definition}
\begin{proposition}
	[\ul{Well ordering property}] Every nonempty subset $S$ of the positive integers $\mb Z^+$ contains a least element, i.e., there exists $q\in S$ such that $q\leq a$ for all $a\in S$. (equivalent to induction).
	Read Appendix~C.
\end{proposition}
\begin{corollary}
	If $y>0$, there exists $n_y\in\mb Z^+$ such that $n_y-1\leq y<n_y$.
	\label{cor3}
\end{corollary}
\begin{proof}
	Consider the set $E_y = \{m\in\mb Z^+ \mid y< m\}$. $E_y$ is  nonempty by the Archimedean property. By Well-ordering, there exists an element $n_y\in E_y$ such that $y<n_y$ and $n_y \leq a$ for all $a\in E_y$. Note in particular $n_y-1\notin E_y$ since $n_y-1<n_y$, i.e. $n_y-1\leq y$.
\end{proof}
\begin{theorem}
	[Density Theorem] If $x$ and $y$ are real numbers with $x<y$, then there exists a rational number $r\in \mb Q$ such that $x<r<y$.
\end{theorem}
\begin{proof}
	First assume that $x>0$. Since $0<y-x$, it follows from Corollary \ref{cor2} that there exists $n\in\mb Z^+$ such that $0<\frac{1}{n}<y-x$. Then $1<n(y-x) = ny-nx$. Therefore, we have: $nx+1<ny$. If we apply Corollary~\ref{cor3} to $nx>0$ there exists $m\in\mb Z^+$ such that $m-1\leq nx<m$. Then $m\leq nx+1 < ny$. So $nx<m<ny$ and $x<\frac{m}{n}<y$.
	Assume $x$ and $y$ are arbitrary elements in $\mb R$ with $x<y$. By Corollary \ref{cor1}, there exists $N\in \mb Z^+$ such that $-N<x<y \implies 0< x+N < y+N$. By what was already proven, there exists $r_1\in\mb Q$ such that $x+N<r_1<y+N$. Then $x<r_1-N<y$. We are done since $r_1-N$ is a rational number.
\end{proof}

\begin{recall}
	[Density Theorem] $x<y$, there exists $r\in\mb Q$ such that $x<r<y$.
\end{recall}
\begin{proposition}
	Let $L$ be a fixed real number. Then there exists an infinite sequence of rational numbers $r_1,r_2,r_3,\dots$ that converges to $L$.
	\label{prop5}
\end{proposition}
\begin{proof}
	Use Density Theorem and Squeeze Theorem.
	Let $a_n = L$ for $n=1,2,3,\dots$ and $b_n = L+ \frac{1}{n}$ for $n\geq 1$. So $\lim_{n\ra\infty} b_n = L$ and $a_n<b_n$ for all $n\geq 1$. For each $n\in\mb Z^+$, use the Density Theorem to pick $r_n\in\mb Q$ such that $a_n<r_n<b_n$. By the Squeeze Theorem $\lim_{n\ra\infty} r_n = L$.
\end{proof}

\begin{example}
	[\ul{Square roots}] Let $a\in\mb R^+$. We will show that there exists a unique positive real number $s$ such that $s^2 = a$. This unique real number $s$ is denoted by $\sqrt{a}$.
\end{example}
\begin{proof}
	Strategy: Construct an infinite sequence $s_1,s_2,s_3,\dots$ in $\mb R$ that converges to a real number $s$ with the desired properties. Let $s_1$ be \ul{any} fixed positive real number and define $s_{n+1} := \frac{1}{2} \left( s_n + \frac{a}{s_n} \right)$ for $n=1,2,3,\dots$.
	\ul{Claim}: $0<s_n$ for $n=1,2,3,\dots$.
	\ul{Claim}: $s_n^2 \geq a$ for $n\geq 2$.
	$$ 2s_{n+1} = s_n + \frac{a}{s_n} = \frac{s_n^2+a}{s_n} \implies 2 s_n s_{n+1} = s_n^2 + a$$ or: $s_n^2 - 2s_n \cdot s_{n+1} + a = 0$. $s_n$ is a real number solution to this quadratic equation. The discriminant: $(-2s_{n+1})^2 - 4\cdot 1 \cdot a \geq 0$ or $4s_{n+1}^2 \geq 4a$ or $s_{n+1}^2 \geq a$ for $n\geq 1$, which implies that $s_n^2 \geq a$ for $n\geq 2$.
	For $n\geq 2$, we have $s_n-s_{n+1} = s_n - \frac{1}{2} s_n - \frac{a}{2s_n} = \frac{1}{2} s_n - \frac{a}{2s_n} = \frac{s_n^2-a}{2s_n} \geq 0$ since $s_n>0$ for all $n\geq 1$ and %we know that 
	$s_n^2-a\geq 0$ for all $n\geq 2$. %We get the property 
	We now have the inequalities $0<\dots\leq s_5 \leq s_4 \leq s_3 \leq s_2$
	since if $s_n-s_{n+1} \geq 0$, then $s_{n+1}\leq s_n$.
	So, by the Completeness Axiom, we know the sequence $s_1,s_2,s_3,\dots$ \ul{converges}. Call the limit $s$, i.e. $\lim_{n\ra\infty} s_n = s$. %(check the sequence converges to the limit $a^2$ using definition)
\end{proof}

Exam 2 will  be on March 3rd.

\begin{recall}
	%\ul{Recap}: 
	$a\in\mb R^+$ (fixed positive real number). $s_1>0$ $\la$ chosen arbitrarily. $s_{n+1} = \frac{1}{2}\left(s_n + \frac{a}{s_n} \right)$ for $n\in\mb Z^+$.
\end{recall}
\begin{proof}
	At a point in the proof we had $s_n$, where $n\in\mb Z^+$, satisfies the quadratic equation $s_n^2 - 2s_{n+1} s_n + a=0$.
	If there exists $x\in\mb R$ such that $x^2 + bx + c = 0$, then $b^2 \geq 4c$.
	We had $x=s_n \in \mb R$ for $1\cdot x^2 - 2s_{n+1} x + a = 0 \implies 4s_{n+1}^2 \geq 4a$ or $s_{n+1}^2 \geq a$ for all $n\geq 1$. We showed that $s_2 \geq s_3 \geq s_4 \geq \dots \geq 0$. The sequence $s_1,s_2,s_3,\dots$ defined in this way converges to some real number $s \geq 0$ by the Completeness Axiom.
	
	Let's complete the square of the quadratic equation: $x^2+bx=-c$, so $x^2 + bx + \left(\frac{b}{2}\right)^2 = \frac{b^2}{4} - c$. Then $0\leq \left( x + \frac{b}{2} \right)^2 = \frac{b^2}{4} -c \implies \frac{b^2}{4} \geq c \implies b^2 \geq 4c$.
	\ul{Claim}: $s^2 = a$. We have $s_n^2 \ra s^2$ since $s_n \ra s$, and $s_{n+1} \ra s$, $a\ra a$. Then $s_n^2 - 2s_{n+1} s_n + a = 0 \implies s^2-2s^2 + a = 0$ or $s^2 = a$.
\end{proof}

\begin{example}
	[HW] $a = 8$. \ul{Choose}: $s_1 = 3$. Then $s_2 = \frac{1}{2} \left( 3 + \frac{8}{3} \right) = \frac{1}{2} \left( \frac{9}{3} + \frac{8}{3} \right) = \frac{17}{6}$. Calculate $s_3$ \ul{as rational} \ul{numbers}; $s_4$, $\cdots$. Then approximate $\sqrt{8}$ as rational decimal. $\sqrt{8} < \cdots< s_3 < s_2 < s_1$.
\end{example}

\begin{definition}
	Given $a\in\mb R^+$, a positive real number $s$ is said to be \ul{a} square root of $a$ if $s^2 = a$.
\end{definition}
\begin{proposition}
	Assume $s,t$ are both positive real numbers with $s^2 = a$ and $t^2 = a$.
	\ul{Claim}: $s=t$, i.e. positive square roots are uniquely defined.
	We set $\sqrt{a}:=s$, where $s$ is the unique positive real number such that $s^2=a$.
\end{proposition}
\begin{proof}
	WLOG $0<s<t$. (i9): $0<s^2<t^2$. Contradiction is $0<a<a$.
\end{proof}
\newpage

\begin{enumerate}
	\item[s1.] $|t| = \sqrt{t^2}$, $\forall t\in\mb R \backslash \{0\}$.
	\begin{enumerate}
		\item[i)] $t>0$: $|t|=t \stackrel{?}{=} \sqrt{t^2}$ $\la$ by definition is a positive real number such that when you square it you get $t^2$. True since $(t)^2 = t^2$.
		\item[ii)] $t<0$: $|t| = -t \stackrel{?}{=} \sqrt{t^2}$. True since $(-t)^2=t^2$ and $-t>0$.
	\end{enumerate}
	\item[s2.] If $0<s\leq t$, then $0<\sqrt{s} \leq \sqrt{t}$.
	\begin{proof}
		By contradiction: $0<\sqrt{t} < \sqrt{s} \implies $ (i9) $ 0<(\sqrt{t})^2 < (\sqrt{s})^2$ or $0<t<s$. Contradiction.
	\end{proof}
	\item[s3.] $\sqrt{s}\cdot \sqrt{t} = \sqrt{st}$, $\forall s,t\in\mb R^+$.
	\begin{proof}
		By Definition, we are looking for the positive real number such that when you square it you get~$st$. Note that  $\left( \sqrt{s} \cdot \sqrt{t} \right)^2 = \left(\sqrt{s}\right)^2\left(\sqrt{t}\right)^2 = s \cdot t$, and since $\sqrt{s}\cdot \sqrt{t} >0$, we are done.
	\end{proof}
\end{enumerate}

Set $0:=\sqrt{0}$. So $x\mapsto \sqrt{x}$ is a function defined on the interval: $[0,\infty)$.

\begin{example}
	[\ul{Cube root again}] Assume $b>0$ fixed. Let $t_1>0$ be chosen in an arbitrary way. Define recursively: $t_{n+1} = \frac{1}{3} \left[ 2t_n + \frac{b}{t_n^2} \right]$ for $n=1,2,3,\dots$.
\end{example}

\begin{example}
	[HW] $b=2$; $t_1 = 2$. $t_2 = \frac{1}{3} \left[ 2\cdot 2 + \frac{2}{2^2} \right] = \frac{1}{3} \left[ 4 + \frac{1}{2} \right] = \frac{1}{3} \left[ \frac{9}{2} \right] = \frac{3}{2}$. Find $t_3$ and $t_4$ as rational numbers.
\end{example}

HW Prove cubic formula converges similar to square root formula.

\begin{definition}
	A positive integer $n$ is a perfect square if $n=d^2$, where $d\in\mb Z^+$.
\end{definition}
\begin{example}
	$1,4,9,16,25,36,49$ are the first 7 perfect squares listed in increasing order.
\end{example}
\begin{recall}
	[Corollary \ref{cor3}] If $y\in\mb R^+$, then there exists $n_y\in\mb Z^+$ such that $n_y-1\leq y < n_y$.
\end{recall}
\begin{theorem}
	If $m\in\mb Z^+$ is \ul{not} a perfect square, then $\sqrt{m}$ is irrational.
\end{theorem}
\begin{proof}
	Let $n$ be an integer with $n<\sqrt{m} <n+1$. We know that $m$ is not a perfect square, so $\sqrt{m}$ is not an integer, which allows us to use the Corollary \ref{cor3} as a strict inequality.
	Goal: Prove that $\alpha = \sqrt{m} - n$ is irrational. Assume instead that $\alpha=\sqrt{m}-n$ is rational. By assumption, we have $0<\sqrt{m}-n <1$. $\la$ Because of this we have $\alpha = \frac{p}{q}$, where $p,q\in\mb Z^+$ and $0<p<q$. Assume this is carried out so that $q$ is as small as possible, \ul{using well-ordering}. Then $\frac{q}{p} = \frac{1}{\sqrt{m} -n} \cdot \frac{\sqrt{m}+n}{\sqrt{m}+n} = \frac{\sqrt{m}+n}{m-n^2} = \frac{\alpha+2n}{m-n^2}$, since $\alpha+2n = \sqrt{m} + n$. \ul{Solve for $\alpha$}: $$\frac{q(m-n^2)}{p} = \alpha + 2n \implies \alpha = \frac{(m-n^2)q}{p} - 2n = \frac{(m-n^2)q-2np}{p} = \frac{r}{p},$$ where $r\in\mb Z^+$ and $p\in\mb Z^+$. We have contradicted the well-ordering principle because $p<q$ and $q$ was assumed to be the smallest possible but we have another rational number for $\alpha$ with smaller denominator. Thus $\alpha$ is irrational. We know $\sqrt{m} = \alpha + n$, which is the sum of an irrational and integer (rational) number. In Homework \#1, Problem 3, it was proven that this sum is irrational, i.e. $\sqrt{m}$ is irrational.
\end{proof}
\begin{corollary}
	$\sqrt{2},\sqrt{3},\sqrt{5},\sqrt{6},\sqrt{7},\sqrt{8},\sqrt{10},\dots$ are \ul{all} \ul{irrational}.
\end{corollary}
\begin{recall}
	[Density Theorem] Between any two real numbers, there is a rational number.
\end{recall}
\begin{theorem}
	[One corollary to the Density Theorem] If $x$ and $y$ are real numbers with $x<y$, then there exists an irrational number $z$ such that $x<z<y$.
	\label{thm8}
\end{theorem}
\begin{proof}
	If we apply the Density Theorem to the real numbers $\frac{x}{\sqrt{2}} < \frac{y}{\sqrt{2}}$, we obtain a rational number $r\neq 0$ such that $\frac{x}{\sqrt{2}} < r < \frac{y}{\sqrt{2}}$. Multiply through \ul{by $\sqrt{2}$}: $x< r\cdot \sqrt{2} < y$. Claim: $z = r\sqrt{2}$ is \ul{irrational}. By Homework \#1, Problem 4, the product of a nonzero rational and irrational number is irrational. $z$ is irrational.
\end{proof}

\section*{Subsequences}

\begin{example}
	$1,\frac{1}{2},\frac{1}{3},\frac{1}{4},\frac{1}{5},\dots,\frac{1}{10},\dots$. Pick: $\frac{1}{2},\frac{1}{5},\frac{1}{10},\dots$.
\end{example}
\begin{definition}
	Let $\{x_n\}$ be a sequence and let $\{p_n\}$ be a strictly increasing sequence of positive integers. Then $\{x_{p_n}\}$ is called a subsequence of $\{x_n\}$.
	\label{def-subsequence}
\end{definition}
\begin{example}
	From previous example, if $p_1 = 3, p_2=7,p_3=8,p_4=17,\dots$, then $\frac{1}{3},\frac{1}{7},\frac{1}{8},\frac{1}{17},\dots$.
\end{example}
\begin{definition}
	A sequence is \say{increasing} if $x_n\leq x_{n+1}$ for all $n\in\mb Z^+$.
	A sequence is decreasing if $x_n\geq x_{n+1}$ for all $n\in\mb Z^+$.
	A sequence is \ul{monotone} if it is either increasing or decreasing.
\end{definition}
\begin{proposition}
	Given definition \ref{def-subsequence}, we claim that $n\leq p_n$ for $n=1,2,3,4,\dots$.
\end{proposition}
\begin{proof}
	(By induction). $Q(1):p_1\geq 1$ since $p_1\in\mb Z^+$. $Q(k): p_k\geq k$ for some $k\in\mb Z^+$. Want to show that $Q(k)\implies Q(k+1)$ for any given $k\in\mb Z^+$. $Q(k+1): p_{k+1} \geq k+1$. Assume that $p_k\geq k$. Then $p_k+1 \geq k+1$. We know that $p_{k+1} > p_k$ and  $p_{k+1} \geq p_k + 1$ because we are working with integers. Thus $p_{k+1} \geq p_k+1 \geq k+1$.
\end{proof}
\begin{theorem}
	[Monotone Subsequence Theorem] If $\{x_n\}$ is any given sequence of real numbers then there is a subsequence of $\{x_n\}$ that is monotone.
\end{theorem}
\begin{proof}
	We will say that the $p$-th term $x_p$ is a \say{peak} if $x_p \geq x_n$ for all $n>p$.
	If $x_p$ is a \say{peak,} then $x_p\geq x_{p+n}$ for all $n\geq 1$. 
	If $x_p$ is \ul{not} a peak, then there $\exists m>p$ such that $x_p<x_m$.
	\ul{Two cases}.
	
	\ul{Case 1}: $\{x_n\}$ has infinitely many peaks. In this case, we list the peaks by increasing subscripts: $x_{p_1},x_{p_2},x_{p_3},\dots$. Since each term is a peak, we have $x_{p_1} \geq x_{p_2} \geq x_{p_3} \geq x_{p_4} \geq \dots$ $\la $ this is a decreasing (i.e. monotone) subsequence of $\{x_n\}$.
	
	\ul{Case 2}: $\{x_n\}$ has only a finite number of peaks (possibly zero peaks).
	In this case, we list the peaks by increasing subscripts: $x_{m_1},x_{m_2},\dots,x_{m_k}$. Let $s_1 := m_k+1$. Since $x_{s_1}$ is \ul{not} a peak, then $\exists s_2 > s_1$ such that $x_{s_1} < x_{s_2}$. Since $x_{s_2}$ is \ul{not} a peak, then there $\exists s_3 > s_2$ such that $x_{s_2} < x_{s_3}$. Continuing in this way, we obtain a strictly increasing (i.e. monotone) sequence $x_{s_1} < x_{s_2} < x_{s_3} < x_{s_4} < \dots$ $\la$ subsequence.
\end{proof}

\begin{example}
	[p.74, \#3] (a) Sequence that has subsequences that converge to 1,2,3: $1,2,3,1,2,3,1,2,3,\dots$.
	(b)~Sequence that has subsequences that converge to $\pm\infty$: $1,-1,2,-2,3,-3,\dots$.
	(c) Sequence with subsequences that are strictly increasing, strictly decreasing, and constant: $1,-1,0,2,-2,0,\dots$.
	(d) Unbounded sequence which has a convergent subsequence: $1,1,1,2,1,3,1,4,\dots$.
	(e) Sequence that has no convergent subsequence: $1,2,3,4,5,\dots$.
\end{example}
\begin{theorem}
	[Bolzano-Weierstrass Theorem] Every \ul{bounded} \ul{sequence} has a \ul{convergent} \ul{subsequence}.
\end{theorem}
\begin{proof}
	If $\{x_n\}$ is bounded, then it is bounded below and bounded above, i.e. there exist $a,b\in\mb R$ such that $a\leq x_n$ for all $n\geq 1$ and $x_n\leq b$ for all $n\geq 1$.
	We wish to prove that $\{x_n\}$ has a subsequence $x_{p_1},x_{p_2},x_{p_3},\dots$ that converges. Once we know a subsequence that does converge to a limit $L$, then by Theorem \ref{thm5} we have $a\leq L \leq b$. From the theorem proved last time, we know that \ul{$\{x_n\}$} has a monotone (i.e. increasing or decreasing) subsequence, say $x_{p_1},x_{p_2},x_{p_3},\dots$. By the Completeness Axiom, this subsequence $x_{p_1},x_{p_2},\dots$ \ul{converges}. The Bolzano-Weierstrass Theorem is equivalent to the Completeness Axiom.
\end{proof}
\begin{recall}
	[Theorem \ref{thm5}] Let $a,b\in\mb R$ with $a<b$. Assume $\{x_n\}$ is a sequence with $a\leq x_n \leq b$ for all $n\in\mb Z^+$ and that \ul{$\{x_n\}$} converges to \ul{$L$}. Then $a\leq L \leq b$.
\end{recall}
\begin{proof}
	Go back and write it up.
\end{proof}
\begin{theorem}
	If $\{x_n\}$ is a sequence of numbers that converges to $L$, then every subsequence of this sequence also converges to $L$.
	\label{thm2-17}
\end{theorem}
\begin{proof}
	By assumption, given $\epsilon>0$, there exists $N\in\mb Z^+$ such that $|x_n-L|<\epsilon$ for all $n\geq N$. Let $x_{p_1},x_{p_2},x_{p_3},\dots$ be any given subsequence of $\{x_n\}$. We want to show that this subsequence also converges to $L$. $p_1<p_2<p_3<\dots$ is an infinite sequence of strictly increasing positive integers. We proved last time: $n\leq p_n$ for all $n\in\mb Z^+$.
	Note that $|x_{p_n} - L|<\epsilon$ for all $n\geq N$.
\end{proof}
\begin{theorem}
	Let $\{x_n\}$ be a sequence with distinct subsequences $x_{p_1},x_{p_2},x_{p_3},\dots \ra L$ and $x_{q_1},x_{q_2},x_{q_3},\dots \\\ra M$ that converge to 2 \ul{different} limits $L\neq M$. Then $\{x_n\}$ does \ul{not} converge.
\end{theorem}
\begin{proof}
	(by Contradiction): Assume $\{x_n\}$ does converge to say $K\in\mb R$. Then, by theorem \ref{thm2-17} $x_{p_1},x_{p_2},\dots$ converges to $K$ and $x_{q_1},x_{q_2},\dots$ converges to $k$. \ul{Contradiction}.
\end{proof}

\begin{definition}
	A sequence $\{x_n\}$ is a \ul{Cauchy} sequence if for each $\epsilon>0$ there exists $N\in\mb Z^+$ such that $|x_m-x_n|<\epsilon$ for all $m,n\geq N$.
\end{definition}
\begin{theorem}
	If $\{x_n\}$ is a convergent sequence of real numbers, then it is a Cauchy sequence.
\end{theorem}
\begin{proof}
	Since $\{x_n\}$ is convergent, say to $L$, then given any $\epsilon>0$ there exists $N\in\mb Z^+$ such that $|x_n - L|<\frac{\epsilon}{2}$ for all $n\geq N$. Also $|x_m-L|<\frac{\epsilon}{2}$ for all $m\geq N$. 
	Assume $m,n\geq N$. $|x_m - x_n| = |x_m-L + L - x_n| \leq |x_m-L| + |x_n-L| < \frac{\epsilon}{2} + \frac{\epsilon}{2} = \epsilon$.
\end{proof}
\begin{recall}
	An infinite sequence $\{x_n\}$ is said to be a \say{Cauchy sequence} if $\forall \epsilon>0$ there exists $N\in\mb Z^+$ such that $|x_m-x_n|<\epsilon$ for all $m,n\geq N$.
\end{recall}
\ul{From last time}: If $\{x_n\}$ converges, then it is a Cauchy sequence.
\begin{lemma}
	If $\{x_n\}$ is a Cauchy sequence, then it is bounded.
\end{lemma}
\begin{proof}
	Let $\epsilon=1$. Since $\{x_n\}$ is Cauchy, then $\exists N$ such that $|x_m-x_n|<1$ for all $m,n\geq N$.
	Note that $|x_m-x_N|<1$ for all $m\geq N$. Then $|x_m| = |x_m-x_N+x_N| \leq |x_m-x_N| + |x_N| < 1+ |x_N|$. Let $M = \max{\{|x_1|,|x_2|,\dots,|x_{N-1}|,|x_N|+1\}}$. \ul{Claim}: $|x_m|\leq M$ for all $m\geq 1$.
\end{proof}
\begin{theorem}
	A sequence $\{x_n\}$ of real numbers is convergent \ul{iff} it is a Cauchy sequence. ($\implies$) Done.
	($\Longleftarrow$) Assume $\{x_n\}$ is a Cauchy sequence. We wish to show this sequence converges to some real \# $L\in\mb R$.
\end{theorem}
\begin{note}
	\ul{$\forall \epsilon>0$} there exists $N\in\mb Z^+$ such that $|x_n-L|<\epsilon$ for all $n\geq N$. Also $p_1<p_2<p_3<\dots$.
\end{note}
\begin{proof}
	By the lemma just proved, the sequence $\{x_n\}$ is bounded since it is Cauchy and so by the Bolzano-Weierstrass Theorem, there is a subsequence $x_{p_1},x_{p_2},x_{p_3},\dots$ of $\{x_n\}$ that converges, say to $L\in\mb R$. Goal: show that the original sequence $\{x_n\}$ also converges to $L$.
	Given $\epsilon>0$, since $\{x_n\}$ is Cauchy, $\exists N \in\mb Z^+$ such that $|x_m-x_n|<\frac{\epsilon}{2}$ for all $m,n\geq N$. Since $x_{p_1},x_{p_2},\dots$ converges to $L$, there $\exists K\in\mb Z^+$ such that $K\geq N$ and $K$ is equal to one of the $p_j$'s so that $|x_K-L|<\frac{\epsilon}{2}$.
	Since \ul{$K\geq N$}, we know that $|x_m-x_K|<\frac{\epsilon}{2}$ for all $m\geq N$. We also have $|x_K-L|<\frac{\epsilon}{2}$. We want $|x_m-L| = |x_m-x_K+x_K-L| \leq |x_m-x_K|+|x_K-L| < \frac{\epsilon}{2} + \frac{\epsilon}{2} = \epsilon$ for all $m\geq N$.
\end{proof}
\begin{remark}
	Cauchy sequences are valid for complex numbers and metric spaces using a definition of distance. Monotone sequences don't make sense for complex numbers because they are not ordered, but Cauchy sequences can apply because of the definition of distance. A metric space is complete if it contains all the limits of its Cauchy sequences.
\end{remark}
\ul{Midterm \#1}: This Friday March 3rd. \ul{Midterm \#3}: Friday, April 7th. \ul{Midterm \#4}: Tuesday, May 2nd.

\begin{example}
	[p.63] Consider the following sequence $c_n = \sum_{k=1}^\infty \frac{1}{k^2}$. $c_1 = 1$, $c_2 = 1+ \frac{1}{4} = \frac{5}{4}$, $c_3 = 1 + \frac{1}{4} + \frac{1}{9} = \frac{49}{36}$, $\dots$. It is a strictly increasing sequence $c_1 < c_2 < c_3 < \dots$. Can show this is a Cauchy sequence, it converges! Converges to $\frac{\pi^2}{6}$, first proved by Euler.
\end{example}

Sources for exam: HW, class theorems, new applications, book theorems.

\ul{Theorems/Results whose proofs you should know}:
\begin{recall}
	[\ul{Corollary \ref{cor2}}] If $t>0$, there exists $N\in\mb Z^+$ such that $0<\frac{1}{N}<t$.
\end{recall}
\begin{recall}
	[\ul{Corollary \ref{cor3}}] If $y>0$, there exists $n_y\in\mb Z^+$ such that $n_y-1\leq y <n_y$.
\end{recall}

\begin{example}
	[\ul{Quick Application of Corollary \ref{cor2}}]
	$\{\frac{1}{n}\}_{n=1}^\infty$ converges to 0.
\end{example}
\begin{proof}
	Need to prove: \ul{Given $\epsilon>0$}, we need to find \ul{$N\in\mb Z^+$} (find Waldo) such that $\frac{1}{n}=|\frac{1}{n}|<\epsilon$ for all $n\geq N$. By Cor.2 there exists $N\in\mb Z^+$ such that $0<\frac{1}{N}<\epsilon$. Claim: $\frac{1}{n} \leq \frac{1}{N}$ for all $n\geq N$ since $0<N\leq n \implies n\geq N$ by (i10). Thus $\frac{1}{n}\leq \frac{1}{N} < \epsilon \implies \frac{1}{n}<\epsilon$ for all $n\geq N$.
\end{proof} 

\begin{example}
	[Slight strengthening of Bernoulli's inequality] Assume $x>-1$ and \ul{$x\neq 0$}. Prove $(1+x)^n > 1+nx$ for each positive integer \ul{$n\geq 2$}.
\end{example}
\begin{proof}
	\ul{Base case}: $(n=2)$: $(1+x)^2 \stackrel{?}{>} 1 + 2\cdot x \implies 1+2x + x^2 \stackrel{?}{>} 1+2x $. Note that $x^2>0$ if $x\neq 0$ by (i5), so $1+2x+x^2 > 1 + 2x + 0$ by VIII.
	\ul{Induction Step}: Assume that $(1+x)^k>1+kx$ for some positive integer $k\in\mb Z^{\geq 2}$. Want to prove: $(1+x)^{k+1}$. Since $x>-1$, we have that $1+x>0$ and so: $(1+x)^{k+1} \stackrel{(i3)}{>} (1+kx)(1+x)$. From here the proof is same as other proof from HW.
\end{proof}

Exam will not have proof of Density theorem, but know how to use it and what it states, same with square root theorem and monotone subsequence theorem.
Know these proofs:

\begin{recall}
	[\ul{Proposition \ref{prop5}}] Let $L\in\mb R$ be fixed. Then there exists an infinite sequence of rational numbers $r_1,r_2,\dots$ that converges to $L$. (needs squeeze theorem and density theorem).
\end{recall}

Squeeze theorem will be given on exam.

\begin{recall}
	[\ul{Theorem \ref{thm8}}] If $x,y\in\mb R$ with $x<y$, then there exists an irrational number $z$ such that $x<z<y$.
\end{recall}
\begin{recall}
	[\ul{3 proofs}] 
	If $\{x_n\}$ is convergent, then it is Cauchy.
	If $\{x_n\}$ is Cauchy, then it is bounded.
	If $\{x_n\}$ is Cauchy, then it converges.
\end{recall}

\begin{definition}
	[p.13] A set $S\subseteq \mb R$ is an \say{interval} if $S$ contains at least 2 points and for any 2 points $x,y\in S$, every real number between $x$ and $y$ belongs to $S$ as well.
\end{definition}
9 different forms for how an interval looks.
\begin{enumerate}
	\item $(a,b) = \{x\in\mb R \mid a<x<b \}\la$ \ul{bounded} \ul{open} interval. \ul{$a<b$}, $a,b\in\mb R$
	\item $[a,b) = \{ x\in\mb R \mid a\leq x <b \} \la$ bounded, half-open.
	\item $(a,b] = \dots$
	\item $[a,b] = \{ x\in\mb R \mid a\leq x \leq b\} \la$ bounded, \ul{closed}.
	\item $(-\infty,a) = \{x\in\mb R\mid x<a\} \la $ unbounded open interval.
	\item $(-\infty,a] \la $ unbounded closed interval.
	\item $(b,\infty)$
	\item $[b,\infty)$
	\item $(-\infty,\infty) = \mb R \la$ unbounded, both \ul{open} \& \ul{closed}.
\end{enumerate}
Read p.13
\begin{recall}
	[\ul{Bolzano-Weierstrass}] If $x_n\in[a,b]$ for all $n\in\mb Z^+$, then $\{x_n\}$ has a subsequence $\{x_{p_n}\}$ that converges to a limit $L$ with $a\leq L \leq b$, i.e. $L\in[a,b]$.
\end{recall}

\begin{example}
	$f(x) = \frac{1}{x}$ on $(0,1)$ is unbounded. A function that is continuous on $[a,b]$ is always bounded.
\end{example}
\ul{Read \S1.5}: \ul{\ul{Real}-\ul{valued} functions} (codomain = $\mb R$)

We'll focus just on functions whose domains are intervals. Say $f:I \ra \mb R$, where \ul{$I$ is an interval}.

\begin{example}
	Polynomials (addition, multiplication, subtraction) $\frac{1}{2}x+4$, rational functions (addition, multiplication, subtraction, division) $\frac{1}{x}$, algebraic functions (addition, multiplication, subtraction, division, roots) 	$\frac{\sqrt{x}+16}{(x^2-4)^{1/3}+14x^8}$, transcendental functions $\sin{x},e^x$
\end{example}
\begin{example}
	Domain of $\tan{(x)}$ is all reals except: $\{\dots,-\frac{3\pi}{2},-\frac{-\pi}{2},\frac{\pi}{2},\frac{3\pi}{2},\frac{5\pi}{2},\dots\}$. Consider \ul{$\tan{(x)}$} on $(-\frac{\pi}{2},\frac{\pi}{2})$.
	\ul{Domain of $\arcsin{(x)}$} is $[-1,1]$.
\end{example}
\begin{definition}
	[p.40] $I = $ interval; $f:I\ra \mb R$. Let $J$ be a subinterval of $I$. The function $f$ is \ul{increasing} on $J$ if $f(x)\leq f(y)$ for any $x,y \in J$ with $x<y$.
\end{definition}

\begin{definition}
	$I$ is an interval; $f:I\ra \mb R$, $J$ is a subinterval.
	\begin{enumerate}
		\item[a)] The function $f$ is bounded above on $J$ if \ul{$\exists M\in\mb R$ such that $f(x) \leq M$ for all $x\in J$}.
		%\item[b)] The function $f$ is bounded below on $J$ if \ul{$\exists M\in\mb R$ such that $f(x) \geq M$ for all $x\in J$}.
		\item[b)] The function $f$ is \ul{bounded} on $J$ if $\exists M\in\mb R^+$ such that $|f(x)| \leq M$ for all $x\in J$.
	\end{enumerate}
\end{definition}
\newpage
\ul{Read \S1.5}, 
\ul{Read \S3.1}:
\begin{recall}
	A sequence $\{x_n\}$ converges to $L\in \mb R$ if $\forall \epsilon>0$ there $\exists \ul{N\in\mb Z^+}$ such that $|x_n-L|<\epsilon$ \ul{for all $n\geq N$}.
\end{recall}
\begin{remark}
	Among the 9 types of intervals, there are 4 of these that are \ul{\say{open.}}
	\begin{enumerate}
		\item[i)] $(a,b)$, $a<b$,
		\item[ii)] $(a,\infty)$, $a\in\mb R$,
		\item[iii)] $(-\infty,b)$, $b\in\mb R$,
		\item[iv)] $(-\infty,\infty) = \mb R$.
	\end{enumerate}
\end{remark}
\begin{definition}
	Let $I$ be an \ul{open} interval that contains the point $c$ and suppose that $f$ is a function defined on $I$ except possibly at $c$. The function $f$ has limit $L$ at $c$ if $\forall \epsilon>0$ there exists \ul{$\delta>0$} ($\delta\in\mb R^+$) such that $|f(x)-L| <\epsilon$ \ul{for all $x\in I$} that satisfy: \ul{$0<|x-c|<\delta$}.
\end{definition}
\begin{remark}
	We'll usually pick $\delta>0$ small enough so that all $x$ satisfy $0<|x-c|<\delta$ are in $I$, i.e. $(c-\delta,c+\delta)\subseteq I$.
	Pick $\delta \leq \min{\{|b-c|,|c-a|\}}$.
\end{remark}
\begin{note}
	[Short hand notation] Limit of a sequence: $\lim_{n\ra\infty} x_n = L$. Limit of a function at $c$: $\lim_{x\ra c} f(x) = L$.
\end{note}
\begin{note}
	$f'(c) = \lim_{x\ra c} \frac{f(x)-f(c)}{x-c}$ is actually the limit of $g(x) = \frac{f(x)-f(c)}{x-c}$. Assume $f(x)$ is defined on the open interval $I$. $g(x) \la$ is defined on all of $I$ except at $x=c$.
\end{note}

\begin{example}
	[Exam \#7] \ul{Fact 3}: $1+na \leq (1+a)^n$ for all $n\in\mb Z^+$. Follows from Bernoulli's inequality since $-1<0<a$.
\end{example}

Read \say{Elements of Style} on canvas files.

\begin{definition}
	Let $I$ be an open interval that contains a point $c$ and suppose $f$ is defined on $I$ except possibly at $c$.
	The function $f$ has a limit $L$ at $c$ written $\lim_{x\ra c} f(x) = L$ if $\forall \epsilon>0$ there exists $\delta>0$ such that $\forall x\in I$ with $0<|x-c|<\delta$ we have $| f(x) - L|< \epsilon$.
\end{definition}
\begin{example}
	Let $f(x) = x$ on $I=\mb R$. Let $c\in\mb R$. We wish to prove that $\lim_{x\ra c} x = c$.
\end{example}
\begin{proof}
	\ul{Given $\epsilon>0$} we need to \ul{find some $\delta>0$} such that $|x-c|<\epsilon$ for all $x\in I$ with $0<|x-c|<\delta$.
	Just choose $\delta = \epsilon$, you're done.
\end{proof}
\begin{remark}
	Given $\epsilon$ you have to go out and find $\delta$, typically $\delta(\epsilon)$ depends on $\epsilon$.
\end{remark}
\begin{question}
	Assuming the set-up above, with some given $L\in \mb R$, what does it mean if \ul{$\lim_{x\ra c} f(x) \neq L$}?
\end{question}
It means: \ul{$\exists\epsilon>0$} such that $\forall \delta>0$ we have $|f(x)-L|\leq \epsilon$ \ul{for} \ul{some} $x\in I$ with \ul{$0<|x-c|<\delta$}.

\begin{theorem}
	Let $I$ be an open interval that contains a point $c$ and suppose $f$ is defined on $I$ except possibly at $c$.
	a) \ul{$\lim_{x\ra c}f(x) = L$} \ul{iff} for \ul{every} sequence \ul{$\{x_n\}$} in \ul{$I \backslash \{c\}$} that converges to $c$, the sequence \ul{$\{f(x_n)\}$ converges to $L$}. Visualize: $a<x_n<b$ for all $n\in\mb Z^+$; \ul{$x_n\neq c$ for all $n\in\mb Z^+$}.
\end{theorem}
\begin{example}
	$x_n = c+\frac{(b-c)}{2n}$ for $n=1,2,3,\dots$ on $a<x_n<b$.
\end{example}
\begin{proof}
	$(\implies)$ Assume $\lim_{x\ra c} f(x) = L$. Let \ul{$\{x_n\}$ be} a sequence of the type described above.
	\ul{Want to show}: Given $\epsilon>0$ there exists $N\in\mb Z^+$ such that $|f(x_n)-L|<\epsilon$ for all $n\geq N$.
	\ul{Given}: For every $\epsilon>0$ there exists $\delta>0$ such that $|f(x)-L|<\epsilon$ for all $x$ in $I$ with $0<|x-c|<\delta$.
	Also given: We have $x_1,x_2,x_3,\dots \in I\backslash \{c\}$ and given any $\epsilon_1>0$ there exists $N_1 \in \mb Z^+$ such that $|x_m-c|<\epsilon_1$ for all $m\geq N_1$.
	Set $\epsilon_1 = \delta$. Then $\exists N_1\in\mb Z^+$ such that \ul{$0<|x_m-c|<\delta$} for all $m\geq N_1$.
	\ul{Claim}: $N=N_1$ works, i.e. we claim that \ul{$|f(x_m)-L|<\epsilon$ for all $m\geq N_1=N$}.
\end{proof}
\begin{remark}
	[\ul{Chain of events}] Start with $\epsilon>0$. Then you have $\delta$. Set $\epsilon_1 = \delta$. Then you get $N_1$. Use $N=N_1$ to finish the proof.
\end{remark}

\begin{theorem}
	[p.85] $P:$\say{The function $f$ has a limit $L$ at $x=c$}, i.e. $\lim_{x\ra c} f = L$ for $f:I\backslash \{c\} \ra \mb R$.
	$Q:$\say{for \ul{each} sequence $\{x_n\}$ in $I\backslash\{c\}$ that converges to $c$, the sequence $\{f(x_n)\}$ converges to $L$.}.
	\ul{Last time}: $P\implies Q$ proved.
	Now $Q\implies P$. We will prove the contrapositive, i.e. $\neg P\implies \ul{\neg Q}$.
	\ul{Not $P$}: There exists an $\epsilon>0$ such that for \ul{all} $\delta >0$ there is a point $x\in I$ such that $0|x-c|<\delta$ and $|f(x)-L|\geq \epsilon$.
	\ul{Not Q}: There \ul{exists a sequence} $\{x_n\}$ in \ul{$I\backslash \{c\}$} that \ul{converges to $c$} where the sequence \ul{$\{f(x_n)\}$} does not converge to~$L$.
	\label{thm3.2}
\end{theorem}
\newpage
\begin{proof}
		\ul{Not $P$}: There \ul{exists an $\epsilon>0$} such that \ul{for \ul{all} $\delta >0$} there is a point $x\in I$ such that $0|x-c|<\delta$ and $|f(x)-L|\geq \epsilon$.
		For this \ul{$\epsilon>0$}, for each $\delta=\frac{1}{n}$, $n\in\mb Z^+$, there is a point \ul{$x_n\in I$} such that $0<|x_n-c|<\frac{1}{n}$ and $|f(x_n)-L|\geq \epsilon$.
		Note that $x_n \in I \backslash \{c\}$ for all $n\in\mb Z^+$ and $\lim_{n\ra\infty} x_n = c$.
\end{proof}
\begin{theorem}
	$I$ is an open interval containing some point $c$. Two functions $f,g$ defined on all of $I$ except possibly at $c$. Suppose that $\lim_{x\ra c} f(x) = S$ and $\lim_{x\ra c} g(x) = T$.
	\begin{enumerate}
		\item[a)] The function $f+g$ has a limit at $c$ and $\lim_{x\ra c} [ f(x) + g(x) ] = S+T$.
		\begin{proof}
			[Proof (Using $Q$).] Let $\{x_n\}$ be a sequence in $I\backslash\{c\}$ that converges to $c$.
			Want to prove that $\displaystyle\lim_{n\ra\infty} [f(x_n)+g(x_n)] = S+T$.
			We know that: $\lim_{n\ra\infty}f(x_n) = S$ by Theorem \ref{thm3.2} ($P\implies Q$) and $\lim_{n\ra\infty} g(x_n) = T$ by Theorem \ref{thm3.2} ($P\implies Q$).
			By Theorem \ref{thm2.7}(b), we have $\displaystyle\lim_{n\ra\infty} [f(x_n)+g(x_n)] = S+T$.
			$\displaystyle\lim_{x\ra c} [ f(x) + g(x) ] = S+T \la$ This now follows from Theorem \ref{thm3.2} ($Q\implies P$).
		\end{proof}
		\begin{proof}
			[Proof of $a)$ just using the definition \ul{of limits} of functions.] \ul{Given}: Given $\epsilon>0$ there exists $\delta_1>0$ such that for all $x\in I$ with $0<|x-c|<\delta_1$, we have $|f(x)-S|<\frac{\epsilon}{2}$. Given $\epsilon>0$ there exists $\delta_2>0$ such that for all $x\in I$ with $0<|x-c|<\delta_2$, we have $|f(x)-T|<\frac{\epsilon}{2}$.
			Set $\delta = \min{\{\delta_1,\delta_2\}}$ and note that $\delta>0$. Then for all \ul{$x\in I$} with \ul{$0<|x-c|<\delta$}, we have: $|f(x) +g(x) - (S+T)| = |(f(x)-S) + (g(x)-T)| \leq |f(x)-S| + |g(x)-T| < \frac{\epsilon}{2} + \frac{\epsilon}{2} = \epsilon$.
		\end{proof}
		\item[e)] Should assume that $g(x) \neq 0$ \ul{for all $x\in I\backslash{\{c\}}$}. $\lim_{x\ra c} g(x) = T \neq 0$. $g(x) = x$ on $\mb R$, $c=0$. $\frac{f(x_n)}{g(x_n)}$.
	\end{enumerate}
\end{theorem}
\begin{definition}
	Open interval. $\lim_{x\ra c^+} f(x)$. $\forall \epsilon>0$ there exists $\delta>0$ such that $\dots$ \ul{$|f(x)-L|<\epsilon$} for all $x$ with $x \in (c,c+\delta)$ or $0<x-c<\delta$ because $c<x<c+\delta$.
\end{definition}
HW Due Fri March 24: P.90 \#2, P.90 \#11 a), b); P.91 \#21, P91 \#25.

\begin{definition}
	Let $I$ be an interval (not necessarily open). Let $f:I\ra \mb R$ be a function and $c\in I$. We say that $f$ is \say{continuous at $c$} if $\lim_{x\ra c} f(x) = f(c)$ if $c$ is not an end point of $I$.
	$\lim_{x\ra c^+} f(x) = f(c)$ if $c$ is the left end point [Assuming that the left end point of $I$ is in $I$.]. 
	$\lim_{x\ra c^-} f(x) = f(c)$ if $c$ is the right end point [Assuming that the right end point of $I$ is in $I$.]. 
	These 3 limits are assumed to exist for the $c$ in question!
	\label{deflim}
\end{definition}
\ul{By Theorem \ref{thm3.2}}, Definition \ref{deflim} is equivalent to the following: 
$$\text{The sequence $\{f(x_n)\}$ converges to $f(c)$ for \ul{each} sequence $\{x_n\}$ in $I\backslash\{c\}$ that converges to $c$.}$$
The function $f$ is \say{continuous on $I$} if $f$ is continuous at \ul{each} point $c\in I$.
\begin{theorem}
	Let $f$ and $g$ be functions defined on interval $I$, let $c\in I$, $k$ a fixed constant. If $f$ and $g$ are continuous at $c\in I$, then $f+g$, $f-g$, $kf$, and $fg$ (i.e. $fg(x)=f(x)\cdot g(x)$) are all continuous at $c$.
	\begin{align*}
		\lim_{x\ra c} f(x) &= f(c), & \lim_{x\ra c} g(x) &= g(c), & \lim_{x\ra c} [f(x) + g(x)] &= f(c) + g(c)
	\end{align*}
	If \ul{$g(x)\neq 0$} by assumption for all $x\in I$, then $\frac{f}{g}:I\ra \mb R$ is continuous at $c$.
	Thus $\lim_{x\ra c} \frac{f}{g} = \frac{f(c)}{g(c)}$.
	\label{thm3.11}
\end{theorem}
\begin{example}
	[\ul{Polynomials}] $f(x) = a_nx^n + a_{n-1}x^{n-1} + \dots + a_1x^1 + a_0x^0$, where $n\in\mb Z^{\geq 0}$ continuous on~$I$, \ul{$f_0(x)=a_0\in \mb R$}. Prove the $f_i(x)$ are separately continuous on $I$. Is $f(x) = k$ continuous on $I = \mb R$? Yes, $\lim_{x\ra c} k = k \implies |f(x)-L|<\epsilon \implies |k-k|<\epsilon$. \ul{$f(x)$} is continuous $\lim_{x\ra c} x = c$. Apply Theorem \ref{thm3.11} and use induction to prove for all polynomials.
\end{example}
\begin{example}
	[\ul{Rational functions}] i.e. functions of the form $\frac{f(x)}{g(x)}$, where $f(x)$ and $g(x)$ are both polynomials and $g(x) \neq 0$.
\end{example}
\begin{remark}
	A function $f:[a,b]\ra \mb R$ continuous on $I = [a,b]$ has a lot of nice properties.
\end{remark}

\newpage

Printed notes start here

$\delta_1>0$, $f(x)<0$ $\forall x\in [a,a+\delta_1)$. $\delta_2>0$, $f(x)>0$ $\forall x\in (b-\delta-2,b]$

$$|x-c|<\delta \iff -\delta < x-c <\delta.$$

\begin{proof}
	Given $f(a)<0$. \ul{Given $\lim_{x\ra a^+}f(x) = f(a)$}.
	Given $\epsilon>0$ there exists $\delta>0$ such that $|f(x)-f(a)|<\epsilon$ for all $x\in [a,a+\delta)$.
	Let $\epsilon = |f(a)|$. Then $\exists \delta_1>0$ such that $|f(x)-f(a)|<|f(a)|$ \ul{for all $x\in [a,a+\delta_1)$}.
	So $-|f(a)|<f(x)-f(a) < |f(a)| \implies f(x) < f(a) + |f(a)| = 0$.
\end{proof}

\begin{theorem}
	[Spivak p.122] If $f$ \ul{is continuous on} $[a,b]$ with $a<b$ and $f(a)<0<f(b)$, then there is some $x\in(a,b)$ such that \ul{$f(x) = 0$}. In our book: If $f(a)<f(b)$ and $v\in\mb R$ such that $f(a)<v<f(b)$, then some $c\in(a,b)$ such that $f(c) = v$.
\end{theorem}
\begin{lemma}
	[p. 99] If $S$ is a nonempty subset of $\mb R$ that is \ul{bounded} \ul{above} and \ul{$B = \sup{S}$} [$B$ exists from Theorem~(2) in the Supremums and Infimums handout]. Then there is a sequence $\{x_n\}$ in $S$, i.e. $x_j\in S$ for all $j\in \mb Z^+$, such that $\lim_{n\ra\infty} x_n = B$.
\end{lemma}
\begin{proof}
	If $\epsilon = 1$, then $\exists x_1\in S$ such that $B-1<x_1 \leq B$ by definition of $B$ and according to Theorem (1) from supremems and infimums handout. If $\epsilon = \frac{1}{2}$, then $\exists x_2\in S$ such that $B-\frac{1}{2} < x_2 \leq B$. If $\epsilon = \frac{1}{n}$ for $n\in \mb Z^+$, then $\exists x_n \in S$ such that $B-\frac{1}{n} < x_n \leq B$. Use \ul{Squeeze Theorem!}
\end{proof}

\begin{example}
	Consider $f(x) = 0$ for $x\in\{c_1,c_2,c_2\}$. $S = \{x\in [a,b]: f(x)<0\} = [a,c_1)\cup (c_2,c_2)$ and $\sup S = c_3$. Also $S\subset [a,b]$.
\end{example}

\ul{p.97 \#4 HW}. \ul{p.97 \#7}: Let $f:\mb R \ra \mb R$ be continuous on $\mb R$, $\dots$ $f(x)=0$ for all rational $x$ in $\mb R$. Prove $f(x)=0$ for all $x\in \mb R$. \ul{p.107 \#3}: $f(x) = x^3+2x-1$ on $[0,1]$, $f(0) = -1$, $f(1) = 2$. For Bisection method, compute each $f(*)$ as rational \#'s. \ul{p.107 \#16(a)}: Assume that $\exists a,b\in\ I$ with $a<b$ such that $f(a)\neq f(b)$ and $f$ is continuous on $I$.

\begin{theorem}
	[\ul{Spivak's} Theorem 2] If $f$ is continuous on $[a,b]$ and $a<b$, then $f$ is bounded above on $[a,b]$, i.e. \ul{$\exists M\in\mb R$} such that $f(x)\leq M$ for all $x\in[a,b]$.
\end{theorem}
\begin{proof}
	Prove the contrapositive, which says that: \ul{If $f$ is \ul{not} bounded above on $I$}, then $\exists$ a point $z\in[a,b]$ where $f(x)$ is \ul{not} continuous.
	If $f(x)$ is bounded above on $I=[a,b]$, then $\exists M\in \mb R$ such that $f(x)\leq M$ \ul{for} \ul{all $x\in I$}. If $f(x)$ is not bounded above on $I$, then $\forall M\in\mb R$ there exists $x\in I$ such that $M<f(x)$.
	For each positive integer $n\in\mb Z^+$, there exists $x_n\in I$ such that \ul{$f(x_n)>n$}. So now we have constructed an infinite sequence \ul{$\{x_n\}$} in $\ul{I}$.
	By the Bolzano-Weierstrass Theorem, $\{x_n\}$ has a subsequence $\{x_{p_n}\}$ that converges to a limit $z$ with $a\leq z\leq b$.
	Since \ul{$f(x_{p_n}) > p_n \geq n$} for all $n\in\mb Z^+$ and so the sequence $f(x_{p_1}),f(x_{p_2}),\dots$ is unbounded.
	If $\{y_n\}$ converges, then $\{y_n\}$ is bounded. \ul{Contrapositive}: If $\{y_n\}$ is not bounded, then $\{y_n\}$ does \ul{not} converge.
	Therefore, $f(x_{p_1}),f(x_{p_2}),\dots$ does not converge and so $f$ is not continuous at $z$.
\end{proof}
\begin{definition}
	If $S$ is a non-empty subset of $\mb R$ \ul{and} \ul{$f:S\ra \mb R$} is a real-valued function whose domain is $S$, we let $f(S)$ denote the set of all values in $\mb R$ taken on by $f$ for $x\in S$, i.e. $f(S) := \{f(x):x\in S\}$.
\end{definition}
\begin{note}
	If the function $f$ is bounded above on $S$, i.e. $\exists M\in \mb R$ such that $f(x)\leq M$ for all $x\in S$ if and only if the set $f(S)$ is bounded above. Similarly, $f$ is bounded below on $S$ iff $f(S)$ is bounded below.
\end{note}
\begin{definition}
	Let $I$ be an interval, let $f:I\ra \mb R$, and let $c\in I$. The function $f$ is said to \say{obtain its absolute maximum value on $I$ at $c$} if $f(x) \leq f(c)$ for all $x\in I$.
\end{definition}
\begin{example}
	$f(x) = x^2$ on $I = [0,1)$ does not have an absolute maximum value on $I$.
\end{example}
\begin{example}
	$f(x) = \begin{cases} \frac{1}{x} & \text{for } x\in(0,1] \\ 0 & \text{for } x = 0 \end{cases}$ does not have an absolute maximum on $[0,1]$.
\end{example}
\begin{theorem}
	[\ul{Spivak} Theorem 3] If $f$ is continuous on $[a,b]$, then $\exists d\in[a,b]=I$ such that $f(x) \leq f(d)$ for all $x\in I$.
\end{theorem}

\begin{recall}
	[\ul{Spivak} Theorem 3] If $f$ is continuous on the closed and bounded interval $[a,b]$, then $.\exists.$ some number $d\in[a,b]$ such that $f(x) \leq f(d)$ for all $x\in[a,b]$, i.e. $f$ obtains its absolute maximum value on $I = [a,b]$.
\end{recall}
\begin{proof}
	By Spivak's Theorem 2 (proved in class last time), we know that the set $S=\ul{f([a,b])}$ is \ul{bounded} \ul{above} and therefore we have a uniquely defined number $\beta\in\mb R$ such that \ul{$\beta = \sup{\{f([a,b])\}}$}.
	Note that $f(x) \leq \ul{\beta}$ for all $x\in I$. We will prove $.\exists.$ a point $\ul{d}\in[a,b]$ such that \ul{$f(d) = \beta$} and then \ul{we're done}.
	For each positive integer $n\in\mb Z^+$, $\exists d_n\in[a,b]$ such that \ul{$\beta-\frac{1}{n}<f(d_n)\leq \beta$}. See Theorem 1 on p.3 of the \say{Supremums \& Infimums} handout. Note that $d_1,d_2,d_3,\dots$ is an infinite sequence of points in \ul{$[a,b]$}. By Bolzano-Weierstrass, there is a subsequence \ul{$\{d_{q_n}\}$} that you know converges to a point $d\in[a,b]$. Since $f$ is \ul{continuous} at every point in $I$, it is continuous at $d$. This implies that: $\lim_{n\ra\infty} f(d_{q_n}) = \ul{f(d)}$. Note that $\lim_{n\ra\infty} f(d_{q_n}) = \ul{\beta}$ from this and \ul{Squeeze} \ul{Theorem}. Therefore \ul{$f(d) = \beta$}.
\end{proof}
\begin{remark}
	This is a very important proof, study it carefully.
\end{remark}

\section*{\ul{\S3.4}: \ul{Uniform Continuity}}

\begin{definition}
	[Handout \ul{Definition 2}] The function $f$ defined on the interval $I$: $f:I\ra\mb R$ is continuous at $c\in I$ if for every $\epsilon>0$ there exists $s>0$ such that $|f(x) - f(c)<\epsilon$ \ul{for all $x\in I$} that satisfy $|x-c|<\delta$.
\end{definition}
\begin{remark}
	Works equally well for an endpoint $c\in I$.
\end{remark}
Suppose that a function $f$ is continuous on an interval $I$, i.e. $f$ is continuous at every point $\in I$. By definition, for each \ul{$c\in I$} and each \ul{$\epsilon>0$}, there exists $\ul{\delta(\epsilon,c)}>0$ such that $|f(x)-f(c)|<\epsilon$ for all $x\in I$ that satisfy $|x-c|<\delta(\epsilon,c)$.
\begin{example}
	Consider $f(x) = x^2$. Let $\epsilon = 0.4$. Then at $f(2) = 4$ a narrower $\delta$ is required than at $f(1) = 1$.
\end{example}
\begin{definition}
	Let $I$ be an interval. A function $f: I\ra\mb R$ is \ul{uniformly} \ul{continuous} on $I$ if for every $\epsilon>0$ there exists $\delta>0$ such that $|f(y)-f(x)|<\epsilon$ for all $x,y\in I$ that satisfy $|y-x|<\delta$.
\end{definition}
\begin{remark}
	If you have a function that is uniformly continuous on an interval than the function is continuous at every point in the interval, but the converse need not be true.
\end{remark}

\begin{definition}
	Let $I$ be an interval. A functions $f:I\ra \mb R$ is \ul{uniformly} \ul{continuous} \ul{on} \ul{$I$} if $\forall \epsilon>0$ there exists \ul{$\delta>0$} such that \ul{$|f(y) - f(x)|<\epsilon$} for all $x,y\in I$ that satisfy \ul{$|y-x|<\delta$}.
\end{definition}
\begin{theorem}
	If $f:I\ra \mb R$ is uniformly continuous on $I$, then $f$ is continuous on $I$, i.e. $f$ is continuous for each point $c\in I$.
\end{theorem}
\begin{proof}
	Set \ul{$x=c\in I$} and show $f$ is continuous at $c$. Note that \ul{$f$ is continuous at $c$} if \ul{$\forall \epsilon>0$} \ul{there exists} \ul{$\delta>0$} such that $|f(y)-f(c)|<\epsilon$ for all $y\in I$ that satisfy $|y-c|<\delta$.
\end{proof}
\begin{remark}
	The converse statement is definitely \ul{not} true!
	However, we do have the following:
\end{remark}
\begin{theorem}
	If $f:[a,b] \ra \mb R$ is \ul{continuous on the closed} \& bounded interval $[a,b]$, then $f$ is uniformly continuous on $[a,b]$.
\end{theorem}

\begin{example}
	[Recall + Negation] $f:I\ra\mb R$. $f$ is uniformly continuous on $I$ if $\forall \epsilon>0$ there exists $\delta>0$ such that $|f(y)-f(x)|<\epsilon$ for all $x,y\in I$ that satisfy \ul{$|y-x|<\delta$}.
	\ul{Negation}: $f$ is \ul{not} uniformly continuous on $I$ if \ul{$\exists \epsilon>0$} such that $\forall \delta>0$ $|f(y)-f(x)|\geq \epsilon$ \ul{ \ul{for some} $\ul{x,y}\in I$} that satisfy $|y-x|<\delta$.
\end{example}
\begin{proof}
	Assume $f$ is continuous on $I=[a,b]$ but not uniformly continuous on $I$. We seek to obtain a contradition.
	For the \ul{$\epsilon>0$} above, let $\delta = \frac{1}{n}$ (\ul{$n\in\mb Z^+$}) and we know \ul{$\exists x_n,y_n\in I$} with \ul{$0\leq\ul{|y_n-x_n|}<\frac{1}{n}$} such that $|f(y_n)-f(x_n)| \geq \epsilon$.
	In this way, we generate 2 infinite sequences \ul{$\{x_n\}$} and $\{y_n\}$ both in \ul{$I$}.
	By the Bolzano-Weierstrass Theorem, $.\exists.$ a subsequence $\{x_{p_n}\}$ that converge to some point $z\in I$.
	Note that $y_{p_n} = x_{p_n} + (y_{p_n} - x_{p_n})$ for each $n\in\mb Z^+$. By construction, $\lim_{n\ra\infty} (y_{p_n}-x_{p_n}) - x_{p_n}) = 0$.
	Therefore $\lim_{n\ra\infty} y_{p_n} = z+0 = z$. Since $f$ is continuous at $z\in I$, \ul{the sequence} \ul{$\{f(y_{p_n})-f(x_{p_n})\}$} converges to $f(z)-f(z) \ul{= 0}$ and $f|(y_{p_n}) - f(x_{p_n})|\geq \epsilon>0$. \ul{Contradiction}. $\displaystyle\lim_{n\ra\infty} |y_n - x_n| = 0 \implies \lim_{n\ra\infty}(y_n - x_n) = 0$.
\end{proof}
\begin{remark}
	By using $-\frac{1}{n} \leq y_n - x_n \leq \frac{1}{n}$, the theorem could have also been proved by the Squeeze theorem.
\end{remark}
 
\newpage

Exam: Be prepeared to present proofs of \ul{Lemma 1} and \ul{Lemma 2} from the IVT handout as well as EVT part 2: 
If $f:[a,b]\ra\mb R$ is continuous on $[a,b]$, then $\exists d\in I$ such that $f(x)\leq f(d)$ for all $x\in I$. 
Then $\exists c\in I$ such that $f(c) \leq f(x)$ for all $x\in I$. 
$\la$ know both version.
\ul{HW \#4}: All but \#1. \ul{HW \#5}: All but \#3.

\begin{definition}
	Let $I$ be an interval, let $f:I \ra \mb R$, and let $c\in I$. The function $f$ is differentiable at $c$ provided that the limit $\lim_{x\ra c} \frac{f(x) - f(c)}{x-c}$ \ul{exists}. 
	The domain of $g(x) = \frac{f(x) - f(c)}{x-c}$ is $I\backslash\{c\}$.
	If $f$ is differentiable at $c$ we say $f$ has a \say{derivative} at $c$, and this derivative at $c$ is denoted by \ul{$f'(c)$}.
\end{definition}
\begin{theorem}
	Let $I$ be an interval, $f:I\ra\mb R$, $c\in I$.  The function $f$ is differentiable at $c$ with derivative $f'(c) = L$ \ul{iff} for \ul{each} sequence $\{x_n\}$ in $I\backslash\{c\}$ that converges to $c$, the sequence $\{\frac{f(x_n)-f(c)}{x_n-c}\}$ converges to $L$.
\end{theorem}
\begin{theorem}
	Same hypotheses as above. If \ul{$f$ is differentiable at $c$}, then \ul{$f$ is continuous at $c$}. Therefore, if $f$ is differentiable on an interval $J$, then $f$ is continuous on $J$ as well.
\end{theorem}
\begin{proof}
	Notice that for all \ul{$x\in I\backslash \{c\}$}, we have $f(x) - f(c) = \ul{(x-c)}\cdot \left(\frac{f(x)-f(c)}{x-c} \right)$. Let \ul{$\{x_n\}$} be an arbitrary sequence in $I\backslash \{c\}$ that converges to $c$. Then $\lim_{n\ra\infty} (x_n - c) = 0$ and $\lim_{n\ra\infty} \left( \frac{f(x_n) - f(c)}{x_n-c} \right) = L\in \mb R$. So $\lim_{n\ra\infty} [f(x_n)- f(c)] = 0 \implies \lim_{n\ra\infty} f(x_n) = f(c)$.
\end{proof}
\begin{theorem}
	[\ul{Product Rule}, \ul{Quotient Rule}] \ul{Read proof on your own}! Also \ul{Chain Rule}.
\end{theorem}
\begin{theorem}
	If \ul{$f(x) = x^n$}, where $n$ is a fixed positive integer, then $f'(c) = n\cdot c^{n-1}$ for any given $c\in\mb R$.
\end{theorem}
\begin{proof}
	Look at $\lim_{x\ra c} \frac{x^n - c^n}{x-c}$. We will use the following identity: \begin{align*}x^n - c^n &= (x-c)(x^{n-1} + x^{n-2}\cdot c + x^{n-3}\cdot c^2 + \dots + x^2 \cdot c^{n-3} + x\cdot c^{n-2} + c^{n-1}) \\ &= \ul{x^n} + x^{n-1} c + \dots + x^2 c^{n-2} + xc^{n-1} - x^{n-1}c - \dots - x^2c^{n-2} - xc^{n-1} \ul{- c^n}\end{align*}
	For $n=1$: $x-c = x-c$, then $X^2 - c^2 = (x-c)(x+c)$, then $x^3-c^3 = (x-c)(x^2 + xc + c^2)$.
	Thus $\lim_{x\ra c} \frac{x^n - c^n}{x-c} = \lim_{x\ra c} [ x^{n-1} + x^{n-2}c + \dots + xc^{n-2} + c^{n-1}] = n\cdot c^{n-1}$.
\end{proof}

\begin{definition}
	Let $I$ be an interval, let $f:I\ra\mb R$, and let $c\in I$. 
	\begin{enumerate}
		\item[d)] The function $f$ has a \ul{relative} \ul{maximum} value  at $c$ if $\exists \delta>0$ such that $f(x)\leq f(c)$ for all $x\in I$ that satisfy \ul{$|x-c|<\delta$}, i.e. $c-\delta<x<c+\delta$.
	\end{enumerate}
\end{definition}
Let $f:[a,b] \ra\mb R$, $a<b$.
\begin{theorem}
	[\ul{Interior-Extremum Theorem}] Let $c\in I$ be such that $a<c<b$, i.e. $c$ is an \say{interior point} of $I$. Assume that $f$ has a relative $\max$/$\min$ value at $c$. If the derivative of $f$ \ul{at} \ul{$c$} \ul{exists}, then $f'(c) = 0$.
	\label{thm-IET}
\end{theorem}
\begin{lemma}
	Let $g$ be defined on $I\backslash\{c\}$ and assume that $\ul{\lim_{x\ra c} g(x) = L} >0$, i.e. given any $\epsilon>0$ there exists $\delta>0$ such that $|g(x)-L|<\epsilon$ for all $x\in I$ such that $0<|x-c|< \ul{\delta}$. Then $\exists\delta_1 >0$ such that \ul{$0<g(x)$ for all $x$ such that $0<|x-c|<\delta$}.
\end{lemma}
\begin{proof}
	[Proof (sketch)] Set $\epsilon = L$. Then $\exists \delta_1 > 0$ such that $|g(x)-L|<L$ for all $x\in I$ such that $0<|x-c|<\delta$, $-L<g(x)-L<L$ or $0<g(x)<2L$.
\end{proof}
\begin{proof}
	[Proof of Theorem \ref{thm-IET}] Let $g(x) = \frac{f(x)-f(c)}{x-c}$. Note that $g(x)$ is defined on \ul{$I\backslash\{c\}$}. Note that since the derivative of $f$ at $c$ exists, equal $ = f'(c)$, we have \ul{$\lim_{x\ra c} g(x) = f'(c)$}. Now assume that \ul{$f'(c) > 0$}. We wish to obtain a contradiction. By the Lemma, $\exists \delta_1 > 0$ such that $\ul{0 < g(x)}=\frac{f(x)-f(c)}{x-c}$ for all $x$ such that $0<|x-c|<\delta_1$ or $c-\delta_1<x<c$. % or $c<x<c+\delta_1$.
	Assume \ul{$c<x<c+\delta_1$}. Then $0<x-c$ and so $0 < \left( \frac{f(x)-f(c)}{x-c} \right) \cdot (x-c) = f(x) - f(c)$, i.e. $f(c)<f(x)$ for all $x$ with $c<x<c+\delta_1$, \ul{contradiction}.
\end{proof}
\begin{theorem}
	[Rolle's Theorem] Suppose that $f$ is \ul{continuous} on \ul{$I=[a,b]$}, that the derivative $f'$ exists at every point in the \ul{open interval $(a,b)$}, and that $f(a) = f(b) = 0$. Then there exists at least one point $c\in(a,b)$ such that $f'(c) = 0$.
\end{theorem}

%% deleted here?????

\begin{recall}
	[Rolle's theorem] i) $f$ is continuous on $I = [a,b]$, ii) $f'$ exists $\forall x \in (a,b)$. $f(a) = f(b) = 0$, iii) \ul{$f(a) = f(b) = 0$}. Then $\exists c\in (a,b)$ such that $f'(c) = 0$.
\end{recall}
\begin{proof}
	\ul{Case 1}: If $f(x) = 0$ for all $x\in I$, then $f'(x) = 0$ for all $x\in (a,b)$. 
	\ul{Case 2}: Assume \ul{$f(t) >0$} for some $t \in(a,b)$. By Extreme Value Theorem, \ul{$\exists c\in (a,b)$}, where $f(x) < f(c)$ for all $x\in I$. 
	By Interior Extremum Theorem, $f'(c) = 0$.
	\ul{Case 3}: Assume $f(t) < 0$ for some $t\in (a,b)$. 
\end{proof}
\begin{theorem}
	[Mean Value Theorem] Suppose that $f$ is continuous on $I = [a,b]$ and that $f'(x)$ exists $\forall x\in (a,b)$. Then there exists at least one point $c \in (a,b)$ such that $f(b)-f(a) = f'(c)(b-a)$.
\end{theorem}
\begin{note}
	$f'(c) = \frac{f(b)-f(a)}{b-a}$ is the slope of the secant line from $a$ to $b$.
\end{note}
\begin{proof}
	Consider the function $\phi(x)$ defined on $I = [a,b]$ by $\phi(x) = f(x) - \left[ f(a) + \frac{f(b)-f(a)}{b-a}\cdot(x-a) \right]$. Note that 
	\begin{enumerate}
		\item[i)] $\phi(x)$ is continuous on $I$. 
		\item[ii)] $\phi(x)$ is differentiable on $(a,b)$ and $\phi'(x) = f'(x) - \frac{f(b)-f(a)}{b-a}$. 
		\item[iii)] $\phi(a) = 0 \phi(b)$
	\end{enumerate}
	By Rolle's Theorem, $\exists c \in(a,b)$ such that $\phi'(c) = 0$ or $\exists c\in (a,b)$, where $f'(c) = \frac{f(b)-f(a)}{b-a}$.
\end{proof}
\begin{corollary}
	[\ul{1st Cor. to MVT}] Suppose $f$ is continuous on $I = [a,b]$, $a<b$, and that $f$ is differentiable on $(a,b)$ with $f'(x) = 0$ for all $x\in(a,b)$. Then $f$ is constant on $I$. 
\end{corollary}
\begin{proof}
	We will show $f(x) = \ul{f(a)}$ for all $x\in I$. Assume $a< x$ and apply the MVT to $f$ on $[a,x]$. By MVT, $\exists c$ with $a<c<x$ such that $0 = f'(c) (x-a) = f(x) - f(a) \implies f(x) = f(a)$. 
\end{proof}
\begin{corollary}
	[\ul{2nd Cor. to MVT}] Suppose $f$ and $g$ are continuous on $[a,b]$ and they are both differentiable on $(a,b)$. Assume $f'(x) = g'(x)$ for all $x\in(a,b)$. Then .$\exists$. a constant $c$ such that $f = g+C$ for all $x\in I$.
\end{corollary}
\begin{proof}
	Consider the function $h(x) = f(x) - g(x)$. Note $h'(x) = f'(x) - g'(x) = 0$ for all $x\in(a,b)$. By 1st Corollary to MVT, we have $h(x) = C$ for all  $x\in I$.
\end{proof}

\section*{Riemann Integral}

Assume $[a,b]$ is a closed bounded interval, $a<b$. 

\begin{definition}
	A partition of $I$ is a \ul{finite}, \ul{ordered} set $P := (x_0,x_1,\dots,x_n)$ of points in $I$ such that $a = x_0 < x_1 < x_2 < \dots < x_n = b$.
\end{definition}
\begin{definition}
	The \ul{norm} of a partition $P$, denoted $||P||$, is equal to $P = \max\{x_i - x_{i-1}: 1\leq i \leq n\}$, i.e. $||P|| =$ the largest of the widths $x_i - x_{i-1}$.
\end{definition}
\begin{definition}
	A tagged partition $^tP$ of an interval $[a,b]$ consists of a partition $(x_0,x_1,\dots,x_n)$ of $[a,b]$ along with a set of points, known as \say{tags} that satisfy \ul{$x_{i-1} \leq t_i \leq x_i$} for $1\leq i \leq n$. Also, $||^tP|| = ||P||$.
\end{definition}
\begin{definition}
	Let $f:[a,b] \ra \mb R$ and let $^tP = \{(t_i,[x_{i-1},x_i]): 1\leq i \leq n\}$ be a tagged partition of $[a,b]$. The \say{Riemann Sum} $S(f,^tP)$ of $f$ associated to $^tP$ is defined by: $S(f,^tP) = \sum_{i=1}^n f(t_i)(x_i - x_{i-1})$.
\end{definition}
\begin{definition}
	A function $f:[a,b] \ra \mb R$ is said to be \say{Riemann Integrable} on $[a,b]$ if there exists a number $L\in\mb R$ such that $\forall \epsilon>0$ there exists $\delta>0$ such that if $^tP$ is \ul{any} tagged partition of $[a,b]$ with $||^tP|| < \delta$, then $|S(f,^tP)-L| <\epsilon$. 
	The set of all Riemann-Integrable functions on $[a,b]$ will be denoted by $\mc R[a,b]$.
\end{definition}
\begin{theorem}
	If $f\in\mc R[a,b]$, then the value $L$ above is uniquely determined.
\end{theorem}
\begin{proof}
	Assume $L'$ and $L''$ both satisfy the definition and let $\epsilon>0$. Then $\exists \delta'>0$ such that $|S(f,^tP_1) - L'|<\frac{\epsilon}{2}$ for all tagged partitions $^tP_1$ of $[a,b]$ with $||P||<\delta'$. Also $\exists \delta''>0$ such that $|S(f^tP_2)-L''|<\frac{\epsilon}{2}$ for all tagged partitions $^tP_2$ of $[a,b]$ with $||^tP_2||<\delta''$. Let $\delta = \min\{\delta',\delta''\}$ and pick a specific tagged partition $^tP$ with $||^tP||<\delta$. Since $||^tP|| < \delta'$ and $||^tP||<\delta''$, we have $|S(f,^tP) - L'|< \frac{\epsilon}{2}$ and $|S(f,^tP)-L''|<\frac{\epsilon}{2}$. 
	By triangle-inequality $|L'-L''| = L' - S(f,^tP) + S(f,^tP) - L''| \leq |L' S(f,^tP)| + |S(f,^tP)-L''| < \frac{\epsilon}{2} + \frac{\epsilon}{2} = \epsilon$. This uniquely \ul{determined} value \ul{$L$} is usualy denoted by $\int_a^b f(x) \, dx$ or $\int_a^b f$. 
\end{proof}
\begin{itemize}
	\item 26.67 points of 32 determined for HW.
	\item 51 points out of 68 determined for exams.
	\item Sum of those 2 is 77.67, so $\frac{26.67}{77.67} = 34.33$ \%; $\frac{51}{77.6} = 65.67 \ra$ divide by 3 $\ra$ 21.89.
	\item (.3433)(\ul{HW \%}) + (.2189)(Exam \#1 \%) + (.2189)(Exam \#2 \%) + (.2189)(.)
\end{itemize}

Let \ul{$I = [a,b]$, $a<b$}, $P = (x_0,x_1,\dots,x_n)$, $a = x_0<x_1<\dots < x_{n-1} < x_n = b$. Subintervals: (non-overlapping) $I_1 = [x_0,x_1], I_2 = [x_1,x_2], I_n = [x_{n-1},x_n]$. Tags are chosen with $t_i\in I_i$ for $i = 1,2,\dots,n$.

\begin{example}
	The function $f(x) = k = $ constant $\forall x\in[a,b]$ is in $\mc R[a,b]$. Guess: \ul{$\int_a^b f = k\ul{(b-a)}$}.
	If $^tP = \{(t_i,[x_{i-1},x_i])\}_{i=1}^n$ is \ul{any} partition of $[a,b]$, then $$S(f,^tP) = \sum_{i=1}^n f(t_i) (x_i - x_{i-1}) = k(x_1 - x_0) + k(x_2 - x_1) + \dots + k(x_n - x_{n-1}) = k[x_n - x_0] = k(b-a).$$ Hence for any $\epsilon>0$, we'll choose $\delta = 1$. Then if $||^tP||<1$, we have $| S(f,^tP) - k(b-a) | = 0 < \epsilon$. Since $\epsilon>0$ is arbitrary, we conclude that $f(x) = k$ is in $\mc R[a,b]$ 
\end{example}
\begin{example}
	Suppose that $c\leq d$ are points in $[a,b]$. If $\phi:[a,b] \ra \mb R$ satisfies \ul{$\phi(x) = 1$} for all $x\in[c,d]$ (if $c=d$, then $[c,d] = \{c\}$) and \ul{$\phi(x) = 0$} elsewhere in $[a,b]$, then we claim that \ul{$\phi \in \mc R[a,b]$} and that $\int_a^b \phi = \ul{(d-c)}$.
	\ul{Goal}: Given any \ul{$\epsilon>0$}, we choose \ul{$\delta = \frac{\epsilon}{4}$} and we will show that if \ul{$||^tP|| <\delta$}, then we have $$\ul{|S(\phi,^tP) - (d-c)| \leq 2\delta} = \frac{\epsilon}{2} < \epsilon.$$
	For a \ul{given tagged partition} $^tP$ with $||^tP||<\delta$, assume \ul{$t_i$ is the first tag} (reading from left to right) in $[c,d]$ and assume $t_{i+N}$ is the last tag in $[c,d]$, where \ul{$N\geq 0$}.
	Note, for any $x_0<x_1<x_2$, we could have $t_1 = t_2$.
	\ul{Case 1}: We could actually have \ul{no} tags in \ul{$[c,d]$}. In this case, note that $S(\phi,^tP) = \sum_{i=1}^n \ul{\phi(t_i)}(x_i-x_{i-1}) = 0$ since $\phi(t_i) = 0$ for \ul{all} of our tags. We \ul{claim} in this case that \ul{$d-c<2\delta$}.
	\begin{proof}
		[Proof of claim]  $t_j < c$ and $d<t_{j+1}$, $-c<-t_j$, so $d-c<t_{j+1} - t_j$. Then $x_{j-1} \leq t_J \leq x_j \leq t_{j+1} \leq x_{j+1}$. Since $||^tP|| < \delta$, we have $\begin{cases} x_j - x_{j-1} < \delta \\ x_{j+1} - x_j < \delta \end{cases} \implies d-c < t_{j+1} - t_j \leq x_{j+1} - x_{j-1} < \ul{2\delta}$.
	\end{proof}
\end{example}
\begin{theorem}
	Suppose that $\ul{f},g\in\ul{\mc R[a,b]}$ and that \ul{$k\in\mb R$} is a constant. Then
	\begin{enumerate}
		\item[a)] $k\cdot f \in \mc R[a,b)$ and $\int_a^b kf = k\int_a^b f$.
		\item[b)] $f+g\in\mc R[a,b]$ and $\int_a^b(f+g) = \ul{\int_a^b f} + \int_a^b g$.
		\item[c)] If $f(x) \leq g(x)$ for all $x\in[a,b]$, then $\int_a^b f \leq \int_a^b g$.
		\item[d)] If $|f(x)|\leq M\in \mb R^{\geq0}$ for all $x\in[a,b]$, then $\left|\int_a^b \right| \leq M(b-a)$.
	\end{enumerate}
\end{theorem}

\begin{theorem}
	[p.199 of Bartle \& Sherbert] Suppose \ul{$f$} and \ul{$g$} are in $\mc R[a,b]$.
	\begin{enumerate}
		\item[(a)] $k\cdot f \in\mc R[a,b]$
		\item[(b)] $f+g \in\mc R[a,b]$
		\item[(c)] If $f(x) \leq g(x)$ for all $x\in[a,b]$, then $\int_a^b f \leq \int_a^b g$.
		\item[(4)] $\left| S(f,^tP)-\int_a^bf \right| < \frac{\epsilon}{2}$, \ul{top of p.200}. $\left| \int_a^b f - S(f^tP) \right| < \ul{\frac{\epsilon}{2}} \overset{a9}{\implies} -\frac{\epsilon}{2} < \int_a^b f - S(f,^tP) < \frac{\epsilon}{2} \implies \int_a^b f - \frac{\epsilon}{2} < S(f,^tP)$. 
	\end{enumerate}
\end{theorem}
\begin{recall}
	Squeeze Theorem for infinite sequences: $\ul{a_n} \leq x_n \leq \ul{b_n}$ for all $n\geq 1$. Assume that $\lim_{n\ra\infty} a_n = \lim_{n\ra\infty} b_n = L$, then $\lim_{n\ra\infty} x_n = L$
\end{recall}

Bottom of p.1 on Squeeze Theorem of Integrals handout: (4) $\left| S(\omega_{\epsilon/3},^tP_2) - \int_a^b \omega_{\epsilon/3} \right| < \frac{\epsilon}{3}$.

\begin{theorem}
	[Big Theorem] If $f:[a,b]\ra\mb R$ is continuous on $[a,b]$ with \ul{$a<b$}, then $f\in\mc R[a,b]$.
\end{theorem}

\begin{proof}
	First of all, we know $f$ is \ul{uniformly} continuous on $[a,b]$, i.e. given \ul{$\epsilon>0$} there exists $\delta>0$ such that if $u,v\in[a,b]$ and $|u-v|<\delta$, then $|f(u) - f(v)| < \frac{\epsilon}{(b-a)}$. Let $P = (x_0,x_1,\dots,x_n)$ be a partition of $[a,b]$ such that $||P||<\delta$. Subintervals: $I_1 = [x_0,x_1], I_2 = [x_1,x_2], \dots, I_n = [x_{n-1},x_n]$. Since $||P||<\delta$, we have: $x_1-x_0<\delta, x_2-x_1<\delta, \dots, x_n-x_{n-1}<\delta$. Let $u_i \in I_i$ be a point where $f$ attains its absolute minimum value on $I_i$. Let $v_i \in I_i$ be a point where $f$ attains its absolute maximum value on $I_i$. We're using the Extreme Value Theorem. Let $\alpha_\epsilon(x)$ be the step function defined by $\alpha_\epsilon(x) = f(u_i)$ for $x\in[x_{i-1},x_i)$ for $i = 1,\dots, n-1$ and we set $\alpha_\epsilon(x) = f(u_n)$ for $x\in[x_{n-1},x_n]$. We define $\omega_\epsilon(x)$ the same way with $\omega_\epsilon(x) = f(v_i)$. Is $\alpha_\epsilon(x) \leq f(x)$ for all $x\in[a,b]$? Yes, also $f(x) \leq \omega_\epsilon(x)$ for all $x\in[a,b]$. We know $\alpha_\epsilon(x),\omega_\epsilon(x)\in\mc R[a,b]$. $0 = \int_a^b \ul{0} \leq \int_a^b (\ul{\omega_\epsilon - \alpha_\epsilon}) = (f(v_1) - f(u_1))\cdot(x_1 - x_0) + \dots = \sum_{i=1}^n (f(v_i)-f(u_i))\cdot(x_i-x_{i-1})$. Need to check $|u_i-v_i| \leq \ul{x_i - x_{i-1}} < \delta$. Then $\sum_{i=1}^n (f(v_i)-f(u_i))\cdot (x_i - x_{i-1}) < \sum_{i=1}^n \frac{\epsilon}{(b-a)} \cdot (x_i - x_{i-1}) = \frac{\epsilon}{(b-a)} \sum_{i=1}^n (x_i - x_{i-1}) = \frac{\epsilon}{(b-a)} \cdot (b-a) = \epsilon$.
\end{proof}

Theorems you should be able to prove on Exam \#4.
\begin{enumerate}
	\item[1.)] If $f:I\ra\mb R$ is differentiable at \ul{$c\in I$}, then $f$ is continuous at $c$. (\ul{$I = $ interval})
	\item[2.)] Product rule for derivatives: $(fg)'(c) = f'(c)\cdot g(c) + f(c)\cdot g'(c)$.
	\item[3.)] Suppose that $f$ is continuous on $I = [a,b]$ and that $f$ is differentiable on $(a,b)$, and that $f'(x) = 0$ for all $x\in(a,b)$. Then $f(x)$ is constant on $I = [a,b]$. (Corollary to MVT)
	\item[4.)] If $f\in\mc R[a,b]$, then the limit value $L$ is uniquely determined.
	\item[5.)] Theorem 7.1.4 parts (a), (b), (c) on p.199 of Bartle \& Sherbert. $f,g\in\mc R[a,b]$, $k$ is a constant.
	\begin{enumerate}
		\item[(a)] $\int_a^b kf = k\int_a^b f$
		\item[(b)] $\int_a^b(f+g) = \int_a^bf + \int_a^b g$
		\item[\ul{(c)}] If $f(x) \leq g(x)$ for all $x\in[a,b]$, then $\int_a^b f\leq \int_a^b g$.
	\end{enumerate}
\end{enumerate}

$\ul{S(kf,^tP)} = \sum_{i=1}^n k\cdot f(t_i)(x_i-x_{i-1}) = k\cdot \sum_{i=1}^n f(t_i)(x_i-x_{i-1}) = \ul{k\cdot S(f,^tP)}$.

\begin{proof}
	[Proof of (a)]: If $k=0$, then $\int_a^b0 =0 = 0\cdot \int_a^b f$. Since $f\in\mc R[a,b]$, \ul{given any $\epsilon>0$} \ul{there exists $\delta>0$} such that if $^tP$ is any tagged partition with \ul{$||^tP||<\delta$}, then \ul{$|S(f,^tP)-\int_a^b f|<\frac{\epsilon}{|k|}$}.
	Want $\ul{|S(kf,^tP)-k\cdot \int_a^b f}| = |k\cdot S(f,^tP)-k\cdot \int_a^b f| = |k|\cdot |S(f,^tP)-\int_a^b f| < |k|\cdot \frac{\epsilon}{|k|} = \epsilon$.
\end{proof}

\begin{recall}
	[Problem \#1 HW \#5] Let $f:[a,b]\ra\mb R$ be \ul{continuous} \ul{at} \ul{$c\in[a,b]$} and suppose that \ul{$f(c)>0$}. Then there exists a positive number $m\in\mb R^+$ and an interval $[u,v] \subset [a,b]$ such that $c\in[u,v]$ and $f(x)\geq m$ for all $x\in[u,v]$.
\end{recall}
\begin{example}
	[\ul{HW \# 6 \#3}] \ul{Suppose $f$ is continuous on $[a,b]$}, \ul{that $f(x)\geq 0$} \ul{for all $x\in[a,b]$}, and $\int_a^b f = 0$. Prove \ul{that $f(x) = 0$} for all $x\in[a,b]$.
		\begin{proof}
			[Proof (by contradiction)] Assume that there exists a point \ul{$c\in[a,b]$} such that $f(c)>0$. To get a contradiction, I want to show that $\int_a^b f>0$.
			
			$h(x) : \begin{cases} 0 & \text{on }[a,u) \\ m & \text{on }[u,v] \\ 0 & \text{on }(v,b] \end{cases}$. $h(x)\in\mc R[a,b]$ and $h(x)\leq f(x)$ for all $x\in[a,b]$. $f(x)\in\mc R[a,b]$. Then $0< m(v-u) = \int_a^b h(x) \leq \int_a^b f(x)$.
		\end{proof}
\end{example}

\begin{definition}
	A function \ul{$F$} such that $F'(x) = f(x)$ \ul{for all $x\in[a,b]$} is \ul{called} \ul{\say{antiderivative}} of $f$ on $[a,b]$.
	$\int_a^b f(x) \, dx = F(b) - F(a)$
\end{definition}
\begin{theorem}
	[\ul{Fundamental Theorem of Calculus (First Form)}] Suppose you have two functions $f,F$ on $[a,b]$ ($a<b$) such that
	\begin{enumerate}
		\item[(a)] $F$ is continuous on $[a,b]$.
		\item[(b)] $F'(x) - f(x)$ for all $x\in(a,b)$.
		\item[(c)] \ul{$f\in\mc R[a,b]$}.
	\end{enumerate}
	Then we have \ul{$\int_a^b f = F(b) - F(a)$}.
\end{theorem}
\begin{example}
	$\int_0^\pi \sin{x} \, dx$
\end{example}
\begin{proof}
	Given $\epsilon>0$. Since $f\in\mc R[a,b]$, there exists \ul{$\delta>0$} such that for any tagged partition $^tP$ with \ul{$||^tP|| < \delta$}, we have \ul{$|S(f,^tP) - \int_a^b f | < \epsilon$}. Take any partition $P$ with $||P||<\delta$.
	Assume, as usual, that $I_i = [x_{i-1},x_i]$ for $i = 1,\dots,n$. Apply the Mean Value Theorem to the function $F$ on $I_i$ ($i$ fixed in the range $1\leq i \leq n$) to conclude that $\exists u_i \in (x_{i-1},x_i)$ such that $F(x_i) - F(x_{i-1}) = F'(u_i)\cdot (x_i-x_{i-1})$. The tagged partition we'll use here is $^uP = \{(u_i,[x_{i-1},x_i])\}$.
	Then $F(b)-F(x) = \sum_{i=1}^n (F(x_i)-F(x_{i-1})$ because $(F(x_1)-F(x_0)) + (F(x_2)-F(x_1)) + \dots + (F(x_n) - F(x_{n-1})) = F(x_i) - F(x_{i-1})$. By \ul{MVT} and condition (b), $F(b)-F(a) = \sum_{i=1}^n F'(u_i) (x_i - x_{i-1}) = \sum_{i=1}^n f(u_i) (x_i - x_{i-1}) = S(f,^uP)$. Thus $|S(f,^uP)-\int_a^b| <\epsilon$ since $||^uP||<\delta$. $\implies |F(b) - F(a) - \int_a^b f | < \epsilon$. Since $\epsilon>0$ \ul{was} \ul{arbitrary}, we conclude that $F(b)-F(a) = \int_a^b f$.
\end{proof}
\begin{theorem}
	If $f\in\mc R[a,b]$ and if $[c,d]\subseteq[a,b]$, then the restriction \ul{of $f$ to $[c,d]$ is in $\mc R[c,d]$}.
\end{theorem}
\begin{definition}
	If \ul{$f\in\mc R[a,b]$}, then the function defined by \ul{$\ul{F(x)} := \int_a^x f$} for any \ul{$x\in[a,b]$} is called the \ul{indefinite} \ul{integral} of $f$, where $[c,d] = [a,x]$.
\end{definition}
\begin{theorem}
	The indefinite integral $F(x)$ defined above is continuous on $[a,b]$.
\end{theorem}
\begin{theorem}
	[\ul{Fundamental Theorem of Calculus (2nd Form)}] Let \ul{$f\in\mc R[a,b]$} and \ul{assume} \ul{$f$} is \ul{continuous} at a point $c\in[a,b]$. Then the indefinite integral $F(x)$ is \ul{differentiable} at $c$ and $F'(c) = f(c)$.
\end{theorem}
\begin{theorem}
	If \ul{$f$} is \ul{continuous} \ul{on $[a,b]$}, then the indefinite integral $F(x)$ is differentiable \ul{on $[a,b]$} and \ul{$F'(x) = f(x)$} for all $x\in[a,b]$, i.e. the indefinite integral $F(x)$ is an anti-derivative of $f(x)$.
\end{theorem}
\end{document}
